%\usepackage[paperheight=8.3in,paperwidth=5.8in,margin=1in]{geometry}
\usepackage[paperheight=9.8in,paperwidth=6.9in,margin=1in]{geometry}
\usepackage{fontspec,graphicx}
\usepackage{setspace}
\usepackage{xstring}
\usepackage{footmisc}
\usepackage{fancyhdr}
\usepackage{bigfoot}
\usepackage[normalem]{ulem}
\usepackage{longtable}
\usepackage{polyglossia}
\usepackage{parskip}

\textwidth=30pc
\textheight=50pc




\usepackage{nameref}
\usepackage[usenames,dvipsnames]{xcolor}
\usepackage{ulem}
\usepackage{calc}
\usepackage{enumitem}





\usepackage{tikz}
\usetikzlibrary{positioning,shapes,shadows,arrows}





\usepackage{pdfpages}
\usepackage[titles]{tocloft}
\usepackage{changepage}





\usepackage{fontspec}
\usepackage{metalogo}
\usepackage{array}
\usepackage{longtable}





%\usepackage{ragged2e}
%\usepackage{adjustbox}
%\usepackage{afterpage}
%\raggedbottom
%%----------------------------------------------------------------------------------------------------------------------------%%
%%Stop text-overflow to right margin
\usepackage{sectsty}
\allsectionsfont{\raggedright}

%%------------------------------------------------------------------------------------%%

%\newfontfamily{\devanagarifont}[Script=Devanagari,Ligatures=TeX,AutoFakeBold=3.5,AutoFakeSlant,WordSpace=1.2, Scale=1]{Kokila}
\newfontfamily{\devanagarifont}[Script=Devanagari,Ligatures=TeX,AutoFakeBold=3.5,AutoFakeSlant,WordSpace=1.2, Scale=1]{Kokila}

%%----------------------------------------------------------------------------------------------------------------------------%%

%%Set Main Font and it's features
%\setmainfont[Script=Devanagari,Ligatures=TeX,AutoFakeBold=3.5,AutoFakeSlant,WordSpace=1.2, Scale=1]{Kokila}
\setmainfont[Script=Devanagari,Ligatures=TeX,AutoFakeBold=3.5,AutoFakeSlant,WordSpace=1.2, Scale=1]{Kokila}

%%----------------------------------------------------------------------------------------------------------------------------%%

%%default language
\setdefaultlanguage{Sanskrit}
\setotherlanguage{hindi}
%%------------------------------------------------------------------------------------%%

%%Sets Parindent to 0
\setlength\parindent{0pt}

%%Change Paraskip/distance between paragraphs
\usepackage{parskip}
\setlength\parskip{1.2ex}
%%------------------------------------------------------------------------------------%%
%%define commands for different types of texts
\newcommand*{\pratIkam}[1]{\textbf{#1}}
\newcommand*{\janAH}[1]{\textbf{#1}}
\newcommand*{\uddharaNam}[1]{\textbf{`#1'}}
\newcommand*{\uddharaNashloka}[1]{\begin{center}\textbf{#1}\end{center}}
\newcommand*{\shlokaH}[1]{\begin{center}\textbf{#1}\end{center}}
\newcommand*{\puShpikA}[1]{\begin{center}\small{#1}\end{center}}
%\newcommand*{\sUtram}[1]{\begin{center}\textbf{\LARGE{#1}}\end{center}}
\newcommand*{\sUtram}[1]{\begin{center}\textbf{#1}\end{center}}
\newcommand*{\sUtramm}[1]{\textbf{#1}}
\newcommand*{\adhikaraNam}[1]{\begin{center}\textbf{\huge{#1}}\end{center}}
\newcommand*{\dIpikaa}[1]{\textbf{{\large#1}}}
\newcommand*{\TIkA}[1]{\textbf{#1}}
\newcommand*{\jnm}[1]{\begin{center}\textbf{#1}\end{center}} %%for jaiminIyanyAyamAlA
\newcommand*{\laxaNam}[1]{\textbf{#1}}
\newcommand*{\nyAyaH}[1]{\textbf{#1}}
\newcommand*{\anumAnam}[1]{\textbf{#1}}
\newcommand{\mds}[1]{{#1} - इति मणिशास्त्रिभिः शोधितमत्र ।}
\newcommand{\mdsf}[1]{\footnote{{#1} - इति मणिशास्त्रिभिः शोधितमत्र ।}}
\newcommand{\snp}[1]{{#1} - इति सम्पूर्णानन्दविश्वविद्यालयात् प्रकाशितपुस्तकस्य पाठोऽत्र ।}
\newcommand{\snpf}{ - इति सम्पूर्णानन्दविश्वविद्यालयात् प्रकाशितपुस्तकस्य टिप्पणम् ।}
\newcommand{\bpp}[1]{{#1} - इति भाण्डारकरसंस्थानात् प्रकाशितपुस्तकस्य पाठोऽत्र ।}
\newcommand{\bppf}{ - इति भाण्डारकरसंस्थानात् प्रकाशितपुस्तकस्य टिप्पणम् ।}



\newcommand{\printedtxt}[1]{{#1} - इति मुद्रितपाठोऽत्र ।}
%\newcommand{\printedtxtf}[1]{\footnote{{#1} - इति मुद्रितपाठोऽत्र ।}}
\newcommand{\printedfootnote}{ - इति मुद्रितसंस्करणस्य टिप्पणम् ।}
%\newcommand{\printedfootnotef}[1]{\footnote{{#1} - इति मुद्रितसंस्करणस्य टिप्पणम् ।}}
\newcommand{\myimagination}[1]{{#1} - इति ललितालालितस्य कल्पितः पाठः ।}
%\newcommand{\myimaginationf}[1]{\footnote{{#1} - इति ललितालालितस्य कल्पितः पाठः ।}}

\newcommand{\GOMLTR}[1]{{#1} - इति चेन्नैराजकीयमातृकागारस्थायाः D15363-SD1321-इतिसङ्ख्याकायाः पाठः ।}
\newcommand{\ALRCTRf}[1]{\footnote{{#1} - इति अड्यारस्थमातृकायाः TR-158 इतिसङ्ख्याकायाः पाठः ।}}
\newcommand{\drf}[1]{{#1} - इति कुत्रचित् पाठः ।}


\newcommand{\raw}[1]{{#1} - Typed with readings from multiple sources.}
%%------------------------------------------------------------------------------------%%

%%Define New environment

%%This make things bold and centered if things are put between \begin{shlokah} and \end{shlokah}
%\newenvironment{shlokah}{\center\bfseries}{\endcenter}
%\newenvironment{shlokaH}{%
%	\center\bfseries %
%}{%
%	\endcenter %
%}

\newenvironment{suutram}{
	\begin{center}\textbf{सूत्रम्}\end{center}
	\fontsize{18}{22}\selectfont}{\null\par\hrule\null\par}

\newenvironment{siddhAntakaumudI}{
	\begin{center}\textbf{सिद्धान्तकौमुदी}\end{center}
	\fontsize{14}{17}\selectfont}{\null\par\hrule\null\par}

\newenvironment{bAlamanoramA}{
	\begin{center}\textbf{बालमनोरमा}\end{center}
	\fontsize{16}{19}\selectfont}{\null\par\hrule\null\par}

\newenvironment{viTThaleshI}{
	\begin{center}\textbf{लघुचन्द्रिकाया विट्ठलेशी}\end{center}
	\fontsize{12}{15}\selectfont}{\null\par\hrule\null\par}
%%----------------------------------------------------------------------------------------------------------------------------%%
%%----------------------------------------------------------------------------------------------------------------------------%%
%%make normalfootnote size smaller thank bigfoot footnotes
%\renewcommand{\footnotesize}{\tiny}


%%Stop Footnote Indent which occurs by default
\setlength\footnotemargin{0em}

%% Restart footnote number each page instead of continuing through pages
\usepackage{perpage}
\MakePerPage{footnote}
%%------------------------------------------------------------------------------------%%
%%make all Arabic/Roman numerals Devanagari
\makeatletter
\def\devanagarinumber#1{\devanagaridigits{\number #1}}
\let\orig@arabic\@arabic
\let\@arabic\devanagarinumber
\makeatother
%%----------------------------------------------------------------------------------------------------------------------------%%
%%Play with title\usepackage{titlesec} %% needed for what we do next
%%newfont family for title
\usepackage{titlesec}
\newfontfamily\titlefont[Script=Devanagari,Ligatures=TeX,AutoFakeBold=3.5,AutoFakeSlant,WordSpace=1.2, Scale=1]{Siddhanta}

%%Rename Chapter to अध्यायः
\makeatletter
\renewcommand{\@chapapp}{अद्ध्यायः}
\makeatother

%% change TOC to विषयसूची
\renewcommand{\contentsname}{विषयसूची}

%% Redefine font of title and it's position
\titleformat{\chapter}[display]{\centering\titlefont\Huge\bfseries}{\chaptertitlename\ \thechapter}{10pt}{\Huge} %% this brought everything in center and made bold and changed font.

%From NityanandaMishra
\makeatother
\titleformat{\chapter}[display]
{\bfseries\Large}
{\centering\huge{\chaptertitlename}}
{1ex}
{\titlerule\vspace{1ex}\centering}
[\vspace{1ex}\titlerule]
\renewcommand\cftchapfont{\LARGE\bfseries}
\renewcommand\cftsecfont{\Large}
\renewcommand\cftchappagefont{\LARGE\bfseries}
\renewcommand\cftsecpagefont{\Large}
\setlength{\cftsecnumwidth}{4em}
\usepackage{footmisc}
\makeatletter


%%----------------------------------------------------------------------------------------------------------------------------%%
%%\usepackage{endnotes}
%%\let\footnote=\endnote
%%------------------------------------------------------------------------------------%%



%%------------------------------------------------------------------------------------%%

%%----------------------------------------------------------------------------------------------------------------------------%%

%%----------------------------------------------------------------------------------------------------------------------------%%



%%------------------------------------------------------------------------------------%%%----------------------------------------------------------------------------------------------------------------------------%%



%%----------------------------------------------------------------------------------------------------------------------------%%

%%start work on bigfoot%%

%% define font for different bigfoot footnotes
\newfontfamily{\footAfont}[Script=Devanagari]{Kokila}
\newfontfamily{\footBfont}[Script=Devanagari]{Kokila}
\newfontfamily{\footCfont}[Script=Devanagari]{Kokila}
%%\newfontfamily{\footDfont}[Script=Devanagari]{Arial Unicode MS}


%% set bigfoot footnotes' size
\def\vivaranam#1{{\fontsize{12pt}{14pt}\selectfont\footAfont #1}}
\def\mUlaTippaNam#1{{\fontsize{10pt}{12pt}\selectfont\footBfont #1}}
\def\TIkATippaNam#1{{\fontsize{10pt}{12pt}\selectfont\footCfont #1}}
%%\def\atiriktam#1{{\fontsize{8pt}{10pt}\selectfont\footDfont #1}}


%%footnote for vivaranam
\SelectFootnoteRule[1]{default}[\centerline{{\fontsize{7pt}{9pt}\selectfont अद्वैतदीपिकाविवरणम्}}\medskip] %this prints title of bigfoot footnote with specific size
\SetFootnoteHook{\footAfont}
\DeclareNewFootnote{A}
\renewcommand{\thefootnoteA}{} %enabling this hides footnote reference numbers

%%footnote for mUlTippaNam
%%\SelectFootnoteRule[1]{default}%%[\centerline{{\fontsize{10}{12}\selectfont भावद्योतनिका}}\medskip] %this prints title of bigfoot footnote with specific size
\SetFootnoteHook{\footBfont}
\DeclareNewFootnote[para]{B}
%%\renewcommand{\thefootnoteB}{} -- allow numbering by disabling it


%%footnote for TIkATippaNam
%%\SelectFootnoteRule[1]{default}[\centerline{{\fontsize{10}{12}\selectfont टीकापाठभेदाः}}\medskip] %this prints title of bigfoot footnote with specific size
\SetFootnoteHook{\footCfont}
\DeclareNewFootnote[para]{C}%%[roman] %% writes all footnotes in same para and marks alphabetically
%%\renewcommand{\thefootnoteC}{} -- allow numbering by disabling it

%%\SelectFootnoteRule[1]{default}[\centerline{{\fontsize{10}{12}\selectfont टीकापाठभेदाः}}\medskip]
%%\DeclareNewFootnote{D}


%% Grab data from Tex file for corresponding bigfoot footnotes
\newread\fntA
\openin\fntA=vivaranam %%%%%%%%%For Thika footnotes
\newread\fntB
\openin\fntB=mUlaTippaNam %%%%%%%%%For Vyakya footnotes
\newread\fntC
\openin\fntC=TIkATippaNam %%%%%%%%%For Hindi Arth footnotes
%%\newread\fntD
%%\openin\fntD=atiriktam %%%%%%%%%For Hindi Arth footnotes


\newif\iffntArem \fntAremtrue 
\newif\iffntBrem \fntBremtrue 
\newif\iffntCrem \fntCremtrue 
%%\newif\iffntDrem \fntDremtrue 


\long\def\footA{\read\fntA to\datafromA%
	\footnoteA{\vivaranam{\datafromA}}% 
}
\long\def\footB{\read\fntB to\datafromB%
	\footnoteB{\mUlaTippaNam{\datafromB}}% 
}
\long\def\footC{\read\fntC to\datafromC%
	\footnoteC{\TIkATippaNam{\datafromC}}% 
}
%%\long\def\footD{\read\fntD to\datafromD%
%%	\footnoteD{\atiriktam{\datafromD}}% 
%%}
%%----------------------------------------------------------------------------------------------------------------------------%%
%% Prints a horizontal line with 2pt skip both up and down which separates footnotes and chapter
\renewcommand\footnoterule{%
	\vskip 2pt%
	\hrule%
	\vskip 5pt%
}
%%----------------------------------------------------------------------------------------------------------------------------%%
%%Fancy Header
\pagestyle{fancy}
\fancyhead{}
\fancyhead[LE]{\nouppercase\leftmark}% LE -> Left part on Even pages
\fancyhead[RO]{\nouppercase\rightmark}% RO -> Right part on Odd pages
%%-----------------------------------------------------------------------------------------------------------------------------%%
\usepackage{hyperref,bookmark}
\hypersetup{
	pdftitle={अद्वैतिसिद्धिः सलघुचन्द्रिका},
	pdfauthor={स्वामिमधुसूदनसरस्वती},
	pdfkeywords={वेदान्तम्, अद्वैतवेदान्तम्, उपनिषद्, शङ्कराचार्यः, मधुसूदनसरस्वती},
	pdfsubject={द्वैतमित्थ्यात्वस्थापनपूर्व्वकोद्वैतनिश्चयः},
	bookmarksnumbered,
	pdfpagelayout=TwoPageRight,
	bookmarksopen=true,
	pdfstartview={FitH},
	colorlinks,
	urlcolor=cyan,
	linkcolor=blue
}
%%-----------------------------------------------------------------------------------------------------------------------------%%
\makeatletter
\renewcommand*{\cleardoublepage}{\clearpage\if@twoside \ifodd\c@page\else
	\hbox{}
	\thispagestyle{empty}
	\newpage
	\if@twocolumn\hbox{}\newpage\fi\fi\fi}
\makeatother

%%-----------------------------------------------------------------------------------------------------------------------------%%
%\interfootnotelinepenalty=100
\interfootnotelinepenalty=0
\clubpenalty=0
\widowpenalty=0
\displaywidowpenalty=0

\usepackage{tcolorbox}
\tcbuselibrary{breakable,skins}
\newtcolorbox{mybreak}{
	enhanced jigsaw,breakable,frame hidden,interior hidden,boxrule=0pt,boxsep=0pt,left=0pt,right=0pt,top=0pt,bottom=0pt,
	title after break={\color{black}--- --- अनुवर्त्तते --- ---\vspace*{\baselineskip}}
}

%%-----------------------------------------------------------------------------------------------------------------------------%%
%Debugging
\showboxdepth=5
\showboxbreadth=5

%\usepackage{fnbreak}	%To Detect Bad Footnote
%%-----------------------------------------------------------------------------------------------------------------------------%%
\setlength\headheight{16pt}
%%-----------------------------------------------------------------------------------------------------------------------------%%
% To Hide Elements Of An Environment
\usepackage{filecontents}
\begin{filecontents*}
	{PackageToHideEnvironment.sty}
	\NeedsTeXFormat{LaTeX2e}
	\ProvidesPackage{PackageToHideEnvironment}
	\RequirePackage{environ}
	\newif\if@hidden% \@hiddenfalse
	\DeclareOption{hide}{\global\@hiddentrue}
	\ProcessOptions\relax
	\NewEnviron{hidelaghuchandrikA}{
		{\null\par\hrule\null\par}%
		{\begin{center}
				\textbf{लघुचन्द्रिका}
		\end{center}}\par
		\fontsize{16}{19}\selectfont
		{\if@hidden\else\BODY\fi}
	} %hides anything which is in laghuchanfrikA Environment; To show the same remove [hide] option from \usepackage[hide]{PackageToHideEnvironment}
	
\end{filecontents*}

\usepackage[hide]{PackageToHideEnvironment}

%\newenvironment{laghuchandrikA}{
%	\begin{center}
%		\textbf{लघुचन्द्रिका}
%	\end{center}
%\fontsize{16}{19}\selectfont
%{\null\par\hrule\null\par}
%}
%%-----------------------------------------------------------------------------------------------------------------------------%%
%Increase Depth of List/Quote
\usepackage{enumitem}
\setlistdepth{99}
%%-----------------------------------------------------------------------------------------------------------------------------%%

%%-----------------------------------------------------------------------------------------------------------------------------%%