\textless{}\textgreater{} - निपात एकाच् ।प्रगृह्र॑मित्यनुवर्तते,
पुँल्लिङ्गतया च विपरिणम्यते । एकाश्चासावच्चेति कर्मधारयः । तदाह ---
एकोऽजित्यादिना । इ विस्मये इति । इ इति चादित्वान्निपातः । स च आश्चर्ये
वर्तत इत्यर्थः । इ इन्द्रः । उ उमेशः । इ इति उ इति निपातः । सम्बोधने
उभयोरपि एकाच्त्वान्निपातत्वाच्च प्रगृह्रत्वान्न सन्धिः ।अना॑ङित्यत्र
ङकारानुबन्धस्य प्रयोजनमाह - अनाङित्युक्तेरिति । आ एवमिति । पूर्व
प्रक्रान्तवाक्यार्थस्याऽन्यथात्वद्योतकोऽयमाकारः । पूर्वमित्थं नामंस्था
इदानीं त्वेवं मन्यसे इत्यर्थः । आ एवमिति । स्मरणद्योतकोऽयमाकारः ।
इहोभयत्रापि आकारस्य ङित्त्वाऽभावान्न पर्युदासः । ङित्त्विति । ङित्तु
आकारः प्रगृह्रो न भवति, अनाङिति पर्युदासादित्यर्थः । ओष्णमिति ।
आ-उष्णमित्यत्र आकारस्य ङित्त्वात्प्रगृह्रत्वाऽभावे सति आद्गुणः । ननु
प्रयोगदशायां ङकारस्याऽश्रवणाविशेषान्ङिदङिद्विवेकः कथमित्यत आह-वाक्येति ।
प्रक्रान्तवाक्यार्थस्यान्यथात्वे स्मरणे च अङित् । अन्यत्र=ईषदाद्यर्थे
गम्ये, ङिदिति विवेकः --- भेदोऽवगन्तव्य इत्यर्थः । तथाच भाष्यम्-॒ईषदर्थे
क्रियायोगे मर्यादाभिविधौ च यः । एतमातांङितं विद्याद्वाक्यस्मरणयोरङित्॥॑
इति॥एकोऽच् यस्ये॑ति बहुव्रीहिस्तु नाश्रितः, तथा
सतिप्रेद॑मित्यादाबतिप्रसङ्गात् ।
