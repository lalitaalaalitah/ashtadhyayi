\textless{}\textgreater{} - न लुमताङ्गस्य ।प्रत्ययलोपे
प्रत्ययलक्षण॑मित्यनुवर्तते ।लु॑इत्येकदेशोऽस्यास्तीति लुमान्=लुक्शब्दः
श्लुशब्दो सुप्शब्दश्च । तेन शब्देन प्रत्ययलोपे विहिते सति
प्रत्ययनिमित्तकमङ्गकार्यं न स्यादित्यर्थः । तदाह --- लुक्श्लु इत्यादिना
। अङ्गस्येत्यनुक्तौ पञ्च सप्तेत्यादौसुप्तिङन्त॑मिति पदसंज्ञा न स्यात्,
जश्शसोर्लुका लुप्तत्वात् । ततश्चन लोपः प्रातिपदिकान्तस्ये॑ति नलोपो न
स्यात् । अतोऽङ्गस्येत्युक्तम् । एवञ्च जसि लुका
लुप्तेप्रत्ययलक्षणाऽभावाज्जसि चे॑ति गुणो न भवतीत्यभिप्रेत्योदाहरति ---
कतीति । प्रसङ्गादाह --- अस्मदिति । त्रिष्विति ।
पुंस्त्रीनपुंसकेष्वित्यर्थः । सरूपा इति । समानानि रूपाणि येषामिति
विग्रहः । लिङ्गविशेषबोधकटाबाद्यभावादिति भावः । नचैवं सतिअलिह्गे
युष्मदस्मदी॑इतिसाम आक॑मिति सूत्रस्थभाष्यविरोध इति वाच्यं,
पदान्तरसंनिधानं विना लिङ्गविशेषो युष्मदस्मच्छब्दाभ्यां न प्रतीयते इति हि
तदर्थः । अत एवन षट्स्वरुआआदिभ्यः॑ इति
पञ्चनादिषट्संज्ञकेभ्यष्टाब्ङीब्निषेधः सङ्गच्छते । अन्यथा
स्त्रीत्वाऽभावादेव तदभावे सिद्धे किं तेन । अत एव चङे प्रथमयो॑रिति सूत्रे
भाष्ये युष्मानित्यत्रतस्माच्छसो नः पुंसी॑त्युपन्यासः सङ्गच्छते । अत एव
चनेतराच्छन्दसी॑ति सूत्रे शिशीनुमादिभिर्युष्मदस्मदाद्यादेशानां
विप्रतिषेधपरं वार्तिकं तद्भाष्ये च सङ्गच्छते । इति दिक् । त्रिशब्दे
विशेषमाह --- त्रिशब्द इति । त्रि-आमिति स्थिते नुटि दीर्घे णत्वे
त्रीणामिति प्राप्ते ।
