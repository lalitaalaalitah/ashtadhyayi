\textless{}\textgreater{} - तस्मिन्निति निर्दिष्टे ।इको यणची॑त्यत्र अचि
इको यण् स्यादित्यवगतम् । तत्राऽचो वर्णान्तराधिकरणत्वं न संभवतीति
सतिसप्तम्याश्रयणीया,-अचि सति इको यण् स्यादिति । तत्र व्यवहितेऽव्यवहिते च
इको यण् प्राप्तः । ततश्चसमिध॑मित्यत्र धकारल्यलहितेऽकारे सत्यरि
मकारादिकारस्य यण् स्यात् । तथा अचि सति पूर्वस्य परस्य वा इको यण्
प्राप्तः । ततश्च दध्युदकमित्यत्र इकारे अचि सति उकारस्य परस्यापि इको यण्
स्यात् । तत्राऽव्यवहित एव अचि भवति न व्यवहिते, पूर्वस्यैव भवति न
परस्येत्येतदर्थमिदमारभ्यते । तस्मिन्निति न तच्छब्दः स्वरूपपरः । तथा
सतितस्मिन्नणि च युष्माकास्माकौ॑ इत्यादावेव प्रवर्तेत, न त्विको
यणचीत्यादौ । किन्तु इको
यणचीत्यादिसूत्रगतस्याऽचीत्यादिसप्तम्यन्तपदस्यतस्मिन्नि॑त्यनुकरणम्
।इती॑त्यनन्तरं ``गम्येऽर्थे'' इति शेषः । निरिति नैरन्तर्ये ।
दिशिरुच्चारणे । एवं च इको यणचि रायो हलीत्यादिसूत्रेषु इचि हलि इत्येवं
सप्तम्यन्तपदगम्येऽर्तेऽकारादौ दध्यत्र सुध्युपास्य इत्यादिप्रयोगदशायां
निर्दिष्टेऽव्यवहितोच्चारिते सति पूर्वस्य कार्यं भवति, न तु
व्यवहितोच्चारिते नापि परस्येति फलितोऽर्थः । व्यवधान\#ं च वर्णान्तरकृतमेव
निषिध्यते, नतु कालकृतम् ।इको यणची॑त्यादौ कालकृतव्यवधानस्य
संहिताधिकारादेव निरासलाभात्, तत्र कालकृतव्यवधानस्याप्यनेनैव सूत्रेण
निरासे संहिताधिकारस्य वैयथ्र्यापातात् । एवं च ये संहिताधिकारबहिर्भूताः
``आनङृतो द्वन्द्वे''देवताद्वन्द्वे च॑ इत्यादय उत्तरपदे परत
आनङादिविधयस्ते सर्वे अगनाविष्णू इत्यग्नाविष्णू इत्याद्यवग्रहे
कालव्यवधानेऽपि भवन्ति । एतत्सर्वमभिप्रेत्य पर्यवसन्नार्थमाह ---
सप्तमीनिर्देशेनेत्यादिना । --- इति सूत्राक्षरानुयायी पन्थाः ।अतिशयने
तम॑बित्यत्र तु नेयं परिभाषा प्रवर्तते, सर्तम्यन्तातिशायनपदार्थस्य
शब्दरूपत्वाभावेनाऽव्यवहितोच्चारितत्वरूपनिर्धिष्टत्वाऽसंभवात् ।
नचैवमपिकर्तृकर्मणोः कृती॑त्यत्रापि अस्याः परिभाषायाः प्रवृत्तौकर्तृषष्ठी
कर्मणिषष्ठी च कृष्णस्य कृतिः जगतः कर्ता कृष्ण इत्यत्रैव स्यान्नतु
``कृतिः कृष्णस्य'' ``कर्ता जगत'' इत्यत्र इति वाच्यं, लक्ष्यानुरोधेन
क्वचिदेवंजातीयकेष्वस्याः परिभाषाया अप्रवृत्तिरिति ``श्नान्नलोप'' इति
सूत्रे भाष्ये प्रपञ्चितत्वात् । वस्तुतस्तु भाष्यानुसारेणाऽत्र सूत्रे
निर्दिष्टग्रहणं संहिताधिकारसूत्रं च विफलमेवेति इको यणचीत्यत्र वक्ष्यते ।
