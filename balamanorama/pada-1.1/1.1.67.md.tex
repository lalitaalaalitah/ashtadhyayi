\textless{}\textgreater{} - तस्मादित्युत्तरस्य । द्व्यन्तरुपसर्गेभ्योऽप
ईत्उदस्थारतम्भोः पूर्वस्ये॑त्यादिसूत्रगतपञ्चम्यन्तस्यानुकरणंतस्मा॑दिति ।
``इति'' शब्दानन्तरं ``गम्येऽर्थे'' इति शेषः । ``निर्दिष्टे''
इत्यनुवर्तते । निरिति नैरन्तर्ये ।
दिशिरुच्चारणेद्व्यन्त॑रित्यादिसूत्रेषु
पञ्चम्यन्तगम्येऽर्थे=द्व्यन्तरादिशब्दे निर्दिष्टे= अव्यवहितोच्चारिते
सत्येव ततः परस्यैव ईत्वं भवति, न तु व्यवहितोच्चारिते द्व्यादिशब्दे, नापि
ततः पूर्वस्य भवतीति नियमार्थमिदम् । तदाह --- पञ्चमीनिर्देशेनेत्यादिना ।
उत्तरस्य किम् तिङ्ङतिङ॑ इति निघात उत्तरस्यैव भवति --- अग्निमीले । नेह
ईले अग्निम् । अव्यवहिते किम्, उत्प्रस्थानम् । उदःस्थास्तम्भोरिति
पूर्वसवर्णो न भवति ।
