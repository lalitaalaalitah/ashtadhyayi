\textless{}\textgreater{} - तदेवमनुनासिका नव अचः, अननुनासिकाश्च
नवेत्यष्टादशविधत्वमेकैकस्याऽच इति स्थितम् । अथाऽनुनासिकसंज्ञामाह ---
मुखनासिका । मुखसहिता नासिका मुखनासिका । शाकपार्थिवादित्वात्सहितपदस्य
लोपः । उच्यते उच्चार्यते इति वचनः । कर्मणि ल्युट् । मुखनासिकया वचन
इतिकर्तृकरणे कृता बहुलम् इति तृतीयासमासः । तदेतदाह --- मुखसहितेत्यादिना
। मुखं च नासिका चेति द्वन्द्वस्तु न । तथा सतिद्वन्द्वश्च
प्राणितूर्यसेनाङ्गानाम् इति समाहारद्वन्द्वनियमात्स नपुंसक॑मिति
नपुंसकत्वेह्यस्वो नपुंसके प्रातिपदिकस्य॑ इति ह्यस्वत्वे ``मुखनासिकवचन''
इत्यापत्तेः । ननु अष्टादश भेदाः किं सर्वेषामचामविशिष्टाः, नेत्याह ---
तदित्थमिति । इयता प्रबन्धेन यत् अचां भेदप्रपञ्चनं तत् ---
इत्थं=वक्ष्यमाणप्रकारेण व्यवस्थितं वेदितव्यमित्यर्थः । अष्टादश भेदा इति
। अष्टादश प्रकारा इत्यर्थः । दीर्घाभावादिति । तथा च उदात्तलृकारदीर्घः,
अनुदात्तलृकारदीर्घः, स्वरितलृकारदीर्घः । ते च अनुनासिकास्त्रयः
अननुनासिकास्त्रय इति षड्भेदानामभावे सति ह्यस्वप्रपञ्चः प्लुतप्रपञ्चश्च
,ड्विध इति लृकारस्य द्वादशविधत्वमेवेति भावः । लृकारस्य दीर्घाऽभावे होतृ
लृकार इत्यत्र सवर्णदीर्घे कृते होतृकार इति ॠकारस्यैव अकस्सवर्ण इति
सूत्रे च भाष्योदाहरणमेव प्रमाणम् । ह्यस्वाभावादिति । यद्येचो ह्यस्वाः
स्युस्तर्हि वर्गसमाम्नाये त एव लाघवात् अ इ उ इत्यादिवत् पठएरन्, नतु
दीर्घाः, गौरवात् । अत एचो ह्यस्वा न सन्तीति विज्ञायते । एवं च
ह्यस्वप्रपञ्चषड्भेदाभावाद्द्वादशविधत्वमेवैचामिति भावः ।
