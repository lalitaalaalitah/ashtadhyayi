\textless{}\textgreater{} - स्वमज्ञाति । अत्रापि सर्वनामानीति विभाषा
जसीति चानुवर्तते । ज्ञातिश्च धनं च ज्ञातिधने, तयोराख्या-ज्ञातिधनाख्या, न
ज्ञातिधनाख्या अज्ञातिधनाख्या, तस्याम् --- अज्ञातिधनाख्यायाम् ।स्व॑मिति
शब्दस्वरूपापेक्षया नपुंसकत्वम् । तदाह --- ज्ञातिधनान्येति । स्वेस्वा इति
। सर्वनामत्वे शीभावः, तदभावे तदभाव इति भावः । आत्मा आत्मीयं ज्ञातिः धनं
चेति स्वशब्दस्य चत्वारोऽर्थाः ।स्वो ज्ञातावात्मनि स्वं त्रिष्वात्मीये
स्वोऽस्त्रियां धने॑ इत्यमरः । अत्रस्वो ज्ञातावात्मनी॑त्येकं वाक्यम् ।
ज्ञातावात्मनि च स्वशब्दः पुँल्लिङ्ग इत्यर्थः ।स्वं त्रिष्वात्मीये॑इति
द्वितीयं वाक्यम् । आत्मीये स्वशब्दो विशेष्यनिघ्न इत्यर्थः
।स्वोऽस्त्रियां धने॑ इति तृतीयं वाक्यम् । धने स्वशब्दः पुंनपुंसक
इत्यर्थः ।स्वः स्यात्पुंस्यात्मनि ज्ञातौ, त्रिष्वात्मीये ।ञस्त्रियां
धने॑ इति मेदनीकोशः । तत्र ज्ञातिधनयोः पर्युदासादात्मनि आत्मीये च
सर्वनामता जसि विकल्प्यत इत्यबिप्रेत्य व्याचष्टे --- आत्मीया इत्यर्थः ।
आत्मान इति वेति । ज्ञातिधनपर्युदासस्य प्रयोजनमाह ---
ज्ञातिधनवाचिनस्त्विति । ज्ञातिवाचिनो धनवाचिनश्च सर्वनामत्वपर्युदासाज्जसि
``स्वाः'' इत्येव रूपमित्यर्थः । नच ज्ञातिधनयोरप्यात्मीयत्वपुरस्कारे
सर्वनामत्वं न स्यादिति वाच्यम्, आख्याग्रहणबलेन ज्ञातित्वधनेत्वपुरस्कार
एव पर्युदासप्रवृत्तेः ।
