\textless{}\textgreater{} - आद्यन्तौ टकितौ । आदिश्च अन्तश्च आद्यन्तौ,
टश्च कच् टकौ । टकारादकार उच्चारणार्थः । टकौ इतौ ययोस्तौ टकितौ ।
द्वन्द्वान्त इच्छब्दः प्रत्येकं संबध्यते । टित्कितावाद्यन्तावयवौ स्तः ।
कस्येत्याकाङ्क्षायां यस्य तौ विहितौ तयोरित्यर्थाल्लभ्यते । तदाह ---
टित्कितावित्यादिना । क्रमादिति यथासङ्ख्य सूत्रलभ्यम् । टित् आद्यवयवः,
किदन्तावयव इत्यर्थः । नचैवं सति मिलितयोरेकत्रान्वयाऽभावात्कथमिह द्वन्द्व
इति वाच्यम्, प्रथमतः समुदायरूपेण परस्परं युगलयोरन्वयबोधमादाय
द्वन्द्वप्रवृत्तौ सत्यां यथासंख्यसूत्रपर्यालोचनया पुनः
प्रत्येकान्वयोपपत्तेः । ``एचोऽयवायावः'' इत्यादावप्येषैव गतिः । लोके
त्वेवंजातीयकप्रयोगोऽसाधुरेवेति भाष्यादिषु स्पष्टम् । अत्रैव
यथासङ्ख्यसूत्रोमन्यासो युक्तः ।आर्धधातुकस्येड्वलादेः॑ भविता ।ङ्णोः कुक्
टुक् शरि॑, प्राङ्क्षष्ट इत्याद्युदाहरणम् ।पुरस्तादपवादा अनन्तरान्विधीन्
बाधन्ते नोत्तरान् इतिषष्ठी स्थानेयोगे॑त्यस्यानन्तरस्यैवायमपवादः ।
``प्रत्ययः'' परश्चेत्यनेन तु परत्वादिदं बाध्यते । तेन चरेष्टः
गापोष्टगित्यादयः परा एव भवन्ति ।
