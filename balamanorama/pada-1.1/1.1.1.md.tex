इदानीं संज्ञान्तराणि विधास्यन् वृद्धिसंज्ञां तावदाह --- --- \pratIkam{वृद्धिरादैच्} ।\\
\textbf{यद्यपि} --- पाणिनीयाष्टाध्याय्यामिदमादिमं सूत्रं , \textbf{तथापि} --- नेदमादावुपन्यस्तम् , अस्य सूत्रस्य तपरकरणेन प्रत्याहारगर्भितत्वेन \textbf{ग्रहणकशास्त्र}नियमार्थ\textbf{तपरसूत्र}-\textbf{प्रत्याहारसूत्र}प्रवृत्त्युत्तरप्रवृत्तिकतया प्रत्याहारशास्त्रप्रपञ्चनिरूपणात् प्रागुपन्यासानर्हत्वात् ।
\begin{quote}
	\textbf{न च} --- \textbf{सूत्रकृता} अयमेव पाठक्रमः कुतो नाद्रियत --- \textbf{इति वाच्यं} ;\\
	स्वतन्त्रेच्छस्य महर्षेर्नियन्तुमशक्यत्वात् ।
\end{quote}

आच् च ऐच् च इति समाहारद्वन्द्वः ।\\
``द्वन्द्वाच्चुदषहान्तात्'' - इति समासान्तस्तु न, अत एव निर्देशेन
समासान्तविधेरनित्यत्वात् । ``चोः कुः'' इति पदान्ते विहितं कुत्वमपि न ,
``अयस्मयादीनि छन्दसि॑'' - इति भत्वात् । ``वृद्धिरादैजदेङ्'' इति
संहितापाठपक्षे चकारस्य ``झलां जशोऽन्ते'' इति पदान्ते विहितजश्त्वं तु
भवत्येव , ``उभयसंज्ञान्यपि छन्दसि दृश्यन्ते'' - इतिवचनात् ,
``छन्दोवत्सूत्राणि भवन्ति'' - इति छान्दसविधीनां सूत्रेष्वपि प्रवृत्तेः ।
न च एवमपि पदत्वात् कुत्वं , भत्वाज् जश्त्वाऽभावश्च कुतो न स्याद् इति
वाच्यं ; ``छन्दसि दृष्टानुविधिः'' - इति वचनाद् । --- इत्यलम् । आच् च ऐच्
च इति इतरेतरयोगद्वन्द्वो वा । तथा सति सौत्रमेकवचनम् ।
आचार्यपारंपर्योपदेशसिद्धसंज्ञाधिकारात् ``सञ्ज्ञा'' - इति लभ्यते ।
तदेतदाह --- आदैच्चेत्यादिना ।
