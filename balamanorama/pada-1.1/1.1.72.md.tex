\textless{}\textgreater{} - यच्छब्दस्वरूपमुपादाय यो विधिरारभ्यते स तस्य
तदन्तस्य च भवतीत्येतद्वक्तुमाह-येन विधिः । विधीयत इति विधिः ।उपसर्गे घोः
कि॑रिति धाधातोर्भावे किप्रत्ययः । येनेति करणे तृतीया ।
शास्त्राकृद्विधाने कर्ता । धातोरित्यधिकृत्य \#एरजिति इकारेण करणेन
धातोरच्प्रत्ययं विधत्ते पाणिनिः । करणं च व्यापारवत् । एरजित्यत्र
विशेषणस्य इकारस्य पाणिनिकर्तृकविधानक्रियायां करणस्य इतरव्यावर्तनमेव
व्यापारः । ततश्च विशेषणमेवात्र येनेति तृतीयान्तेनोच्यते । स्वं
रूपमित्यतः स्वमित्यनुवर्तते षष्ठन्ततया च विपरिणम्यते । एवं
चविशेषणसमर्पकः शब्दस्तदन्तस्य स्वस्य च प्रत्यायक॑ इति फलति । तदाह ---
विशेषणमित्यादि । विशेषणसमर्पकः शब्दस्तदन्तस्य शब्दस्य, विशेषण
समर्पकशब्दस्य च बोधकः स्यादिति यावत् । ततश्च एरजित्यत्र
इकारान्ताद्धातोरच्प्रत्ययः स्यात्, इकाररूपाद्धातोश्चेति फलति । यथा-चयः,
अयः । केचित्तु करणं कर्तृपरतन्त्रमिति तृतीयया पारतन्त्र्यं लक्ष्यते,
तच्च शब्दानां विशेषणत्वेनेति विशेषणपरत्वं यच्छब्दस्य लभ्यत इत्याहुः ।
तत्तु शब्देन्दुशेखरे दूषितम् । समासेति । वार्तिकमेतत् । समासविधौ
प्रत्ययविधौ च तदन्तविधेः प्रतिषेधो वाच्य इत्यर्थः । तेन कृष्णं परमश्रित
इत्यत्रद्वितीया श्रिते॑ति समासो न भवति । सूत्रनडस्य गोत्रापत्यं
सौत्रनाडिः ।अत इञ् । अनुशतिकादीनां चेत्युभयपदवृद्धिः । अत्र नडादिभ्यः
फगिति फग्नभवति । नन्वेवं सति पचन्तमतिक्रान्ता अतिपचन्तीत्यत्रउगितश्चे॑ति
उगिदन्तात्प्रातिपदिकाद्विहितो ङीप् न स्यात्, प्रत्ययविधौ तदन्तविधेः
प्रतिषेधात् । तथा दाक्षिरित्यत्र अत इञिति इञ् न स्यात् । अस्यापत्यं
इरित्यत्रैव इञ्स्यादित्यत आह --- उगिदिति । इदमपि वार्तिकम् ।द्वितीयायां
चे॑ति वर्जयतेर्णमुल् ।उगिद्ग्रहणं वर्णग्रहणं च वर्जयित्या
समासप्रत्ययविधावित्युक्तः प्रतिषेधो भवती॑त्यर्थः । उगिद्वर्णग्रहणे तु
येन विधिरिति तदन्तविधिरस्त्येव, ततश्च अतिपचन्तीत्यत्र
उगिदन्तप्रातिपदिकान्तादुगितश्चेति ङीप्, दाक्षिरित्यत्र अवर्णान्तादिञ् च
सिध्यति ।
