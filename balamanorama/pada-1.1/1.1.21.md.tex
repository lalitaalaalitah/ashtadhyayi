\textless{}\textgreater{} - ननु इदम् भ्यामिति स्थिते त्यदाद्यत्वे पररूपे
इदो लोपे च कृते अ-भ्याम् इति स्थिते
अङ्गस्याकारात्मकत्वाददन्तत्वाभावात्कथंसुपि चे॑ति दीर्घ इत्यत
आह-आद्यन्तवदे । आदित्वान्तत्वयोर्नित्यमन्यसापेक्षत्वादेकस्मिन्
तत्प्रयुक्तकार्याणामप्राप्तौ तत्पाप्त्यर्थमिदमारभ्यते ।एकशब्दोऽसहायवाची
।एके मुख्यान्यकेवलाः॑ इत्यमरः । सप्तम्यन्ता॒त्तत्र तस्येवे॑ति वतिः,
एकस्मिन्नित्युपमेये सप्तमीदर्शनात् । वतिश्च द्वन्द्वान्ते
श्रूयमाणत्वात्प्रत्येकं संबध्यते । तदाह-एकस्मिन्नित्यादि । तदादितदन्तयोः
क्रियमाणं कार्यं तदादौ तदन्त इव च असहायेऽपि स्यादित्यर्थः । एकस्मिन्निति
किम् । दरिद्रातेरेरजिति न । आदिवत्त्वफलम्-औपगव इत्यादौ
अण्प्रत्ययाद्युदात्तत्वम् । आभ्यामित्यादौ अन्तवत्त्वाद्दीर्घादिर्भवति ।
भाष्ये तु आद्यन्तवदित्यपनीय व्यपदेशिवदेकस्मिन्निति सूत्रपाठः शिक्षितः ।
तेन इयायेत्यादौएकाचो द्वे प्रथमस्ये॑ति द्विर्भावः, धुगित्यत्र
व्यपदेशिवत्त्वेन धात्ववयवत्वाद्भष्भावश्च सिध्यति ।
विशिष्टोऽपदेशो-व्यपदेश=मुख्यो व्यवहारः । सोऽस्यास्तीति व्यपदेशी । मुख्य
इति यावत् । एकस्मिन् तदादित्वतदन्तत्वतदवयवत्वादिप्रयुक्तकार्यं स्यादिति
फलितम् । ()इदमोऽन्वादेशेऽशनुदात्तस्तृतीयादौ ।२.४.३२ ।इदम् --- भिसिति
स्थिते त्यदाद्यत्वे पररूपे ``हलि लोपः'' इति इदो लोपे अ-भिस् इति
स्थितेअतो भिस ऐ॑सिति प्राप्ते --- नेदमदसोरकोः ।अतो भिस ऐ॑सित्यतो भिस
ऐसित्यनुवर्तते ।अको॑रिति षष्ठो । न विद्यते ककारो ययोरिति बहुव्रीहिः ।
तदाह --- अककारयोरित्यादिना । एत्त्वमिति ।बहुवचने झलीत्यनेने॑ति शेषः ।
ङयि विशेषमाह-अत्वमित्यादि । अत्वं ङेः स्मै इत्यन्वयः । इदम् एव इति
स्थिते स्मैभावात्परत्वादनादेशेविप्रतिषेधे यद्बाधितं तद्बाधितमेवे॑ति
न्यायेन पुनः स्मैभावो न स्यादित्यत आह --- नित्यत्वादिति ।
कृतेऽकृतेऽप्यनादेशे प्रवृत्तियोग्यतया स्मैभावस्य
नित्यत्वादनादेशात्प्रागेव स्मैभावे कृतेऽनादेशस्य हलि लोपेन बाध इति भावः
। आभ्यामिति । पूर्ववत् । एभ्य इति । त्यदाद्यत्वं, पररूपत्वं, हलि
लोपः,बहुवचने झल्ये॑दित्येत्त्वंचेति भावः । अस्मादिति । त्यदाद्यत्वं,
पररूपत्वं हलि लोपःस ``ङसिङ्योः'' इति स्मादिति भावः । अस्येति ।
त्यदाद्यत्वं, पररूपत्वं, स्यादेशः, हलि लोपश्चेति भावः । अनयोरिति ।
त्यदाद्यत्वंस पररूपत्वम्, ``अनाप्यकः''ओसि चे॑त्येत्त्वम्, अयादेशश्चेति
भावः । एषामिति । आमि त्यदाद्यत्वं, पररूपत्वं, हलि लोपः, एत्त्वषत्वे इति
भावः । अस्मिन्निति । अत्वं, पररूपत्वं, स्मिन्, हलि लोपश्चेति भावः ।
एइआति अत्वं, पररूपत्वं, हलि लोपः, एत्त्वषत्वे इति भावः । ककारयोगे त्विति
।अव्ययसर्वनाम्नामकच् प्राक् टे॑रित्यनेन इदंशब्दस्य, अकचि सतीत्यर्थः ।
अयकमिति । अकचि सति निष्पन्नस्य इदकम्शब्दस्यतन्मध्यपतिन्यायेन ``इदमो मः''
इत्यादाविदंग्रहणेन ग्रहणान्मत्वादिकमिति भावः । ``अनाप्यकः'' इति, ``हलि
लोपः'' इति,नेदमदसोरको॑रिति च नेह प्रवर्तते । ककारयोगे
तन्निषेधादित्याशयेनाह --- इमकेन इमकाभ्यामिति । इत्यादीति । इमकैः ।
इमकस्यै । इमकेभ्यः । इमकस्मात् । इमकस्य, इमकयोः २, इमकेषाम् । इमकस्मिन्
इमकेषु ।
