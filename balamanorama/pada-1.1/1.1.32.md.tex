\textless{}\textgreater{} - विभाषा जसि । सर्वनामग्रहणमनुवर्तते, द्वन्द्व
इति च ।जसी॑त्यविभक्तिको निर्देशः । जस इः जशिः । आर्षः सप्तम्या लुक्
।इ॑शब्द इवर्णपरः सन्शी॑तीकारमाचष्टे । ततश्च जसादेशे शीभावे कर्तव्ये इति
फलितम् । तदाह --- जसाधारमिति । जस् आधारो यस्येति बहुव्रीहिः ।
जस्स्थानकमित्यर्थः । ननुजसि परतो द्वन्द्वे सर्वानामसंज्ञा वा
स्या॑दित्येव कुतो न व्याख्यायत इत्यत आह --- शीभावं प्रत्येवेत्यादिना ।
यदि त्वकच्स्यात्तर्हि तस्याऽव्यवधायकत्वाच्छीभावः प्रसज्येत । कप्रत्यये
तु सति तेन व्यवधानान्नोक्तदोष इत्याह --- वर्णाश्रमेतरका इति । नचाऽकचि
कर्तव्ये विकल्पाऽभावे ।ञपि सर्वादीनी॑ति नित्या सर्वनाम संज्ञा कुतोऽत्र न
स्यादिति वाच्यं,द्वन्द्वे चे॑ति तस्या नित्यानिषेधात् । नचद्वन्द्वे॑चेति
निषेधस्योक्तरीत्याऽवयवेषु प्रवृत्त्यभावाद्वर्णाश्रमेतरशब्दे समुदाये
इतरशब्दस्याऽवयवस्य सर्वनामत्वाऽनपायादकज्दुर्वार इति वाच्यं,
द्वन्द्वावयवमात्रे सुन्दरादिविशेषणान्वयाऽबाववत्कुत्सादिविवक्षाया अभावात्
। समुदाये तद्विवक्षायां समुदायोत्तरप्रत्ययेनाऽवयवगतकुत्सादेरपि
बोधेनोक्तार्थत्वादवयवेभ्यः पृथक् तदनुत्पत्तेः । अन्यथा अवयवेभ्यः
प्रत्येकं कप्रत्ययापत्तेः । एतदेवाभिप्रेत्योक्तं भाष्ये ---
॒वर्णाश्रमेतरशब्दे अकच् न भवती॑ति । एवंच यदा इतरशब्देन द्वन्द्वं कृत्वा
कुत्सिता वर्णाश्रमेतरा इति कुत्सायोगः क्रियते तदा कप्रत्यये सति
``वर्णाश्रमेतरका'' इत्येव रूपम् । यदा तु कुत्सित इतरः-इतरक इत्यकचं
कृत्वा वर्णाश्च आश्रमाश्च इतरकश्चेति द्वन्द्वः क्रियते, तदा शीभावविकल्पः
स्यादेव॥
