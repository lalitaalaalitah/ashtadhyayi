\textless{}\textgreater{} - अतो नैवमर्थः, किं तु लुक्श्लुलुप इत्यावर्तते
। ततश्च लुक् श्लु लुप इत्युच्चार्य विहितं प्रत्ययाऽदर्शनं
यथासङ्ख्यपरिभाषया क्रमाल्लुगादिसंज्ञं स्यादिति लभ्यते, अतो नोक्तसङ्कर
इत्यभिप्रेत्याह-लुक्श्लुलुप्शब्दैरित्यादिना ।फले लुक्,जुहोत्यादिभ्यः
श्लुः॑ ``जनपदे लुप्'' इत्यादिविधिप्रदेशेषुअस्य सूत्रस्य शाटकं
वये॑तिवद्भाविज्ञानान्नान्योन्याश्रयः । तदेवं कतिशब्दस्य षट्संज्ञायाम्
--- ।
