\textless{}\textgreater{} - लृकारेण तु स्थानिना न कस्यापि स्थानसाम्यं,
तत्र कतमो गुणो भवतीत्याकाङ्क्षायामिदमारभ्यते --- उरण्रपरः । इत्युक्तमिति
। ``अणुदित्सूत्रे'' इति शेषः । उरितिऋ॑इत्यस्य षष्ठएकवचनम् । ``षष्ठी
स्थाने'' इति परिभाषया स्थाने इति लभ्यते । अनुवादे तत्परिभाषानुपस्थितावपि
स्थानेग्रहणं ततो ।ञनुवर्तते । तदाह --- तत्स्थाने योऽणिति ।
``स्थानेऽन्तरतम'' इत्यतो ।ञपि स्थानेग्रहणमनुवर्तते । स्थानं प्रसह्ग
इत्युक्तम् । प्रसङ्गावस्थायामित्यर्थो विवक्षितः । तदाह --- रपरः सन्नेव
प्रवर्तत इति । अत्र ``र'' इति प्रत्याहारो विवक्षितः । ततश्च रेफशिरस्को
लकारशिरस्कश्च प्रवर्तत इति लभ्यते । तयोव्र्यवस्थां दर्शयति --- तत्रेति ।
रेफलकारशिरस्कयोर्मध्ये कृष्णर्दिंधरित्यत्र अर्, तवल्कार इत्यत्र
अलित्यन्वयः । कुत इयं व्यवस्थेत्यत आह --- आन्तरतम्यादिति । त्रिषु गुणेषु
प्रसज्यमानेषु अकारस्याऽणो रेफलकारशिरस्कतया तस्य अर् अलित्येवमात्मकस्य
अकारांशे स्थानीभूतेन अकारेण रेफांशे ऋकारेण, लकारांशे लृकारेण च
स्थानसाम्यादकारञकारयोः स्याने अरेव भवति । अकारलृकारयोः स्थानेऽलेव भवति ।
एकारोकारौ तु गुणौ न भवत एव, तयोरृकारेण लृकारेण च
स्थानसाम्या.ञभावादित्यर्थः । नच एकार ओकारश्च कथं रपरो न स्यातामिति
वाच्यं, पूर्वेणैव णकारेण ह्रत्राण् गृह्रते,
प्रशास्तृणामित्यादिनिर्देशादित्यलम् ।पक्षे द्वित्वमिति । ऋधधातोः क्तिनि
झषस्तथोरिति तकारस्य धत्वे ऋद्धिरिति द्विधकारं रूपं स्वाभाविकम् । तत्र
अरादेशे रेफात्परस्य धकारस्याऽचो रहाभ्यामिति कदाचिद्द्वित्वमित्यर्थः ।
