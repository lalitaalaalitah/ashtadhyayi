\textless{}\textgreater{} - अथ
सर्वादिगणान्तर्गततिसूत्रीसमानाकारामष्टाध्यायीपठितां
पूर्वपरेत्यादित्रिसूत्रीं पुनरुक्तिशङ्कां व्युदस्यन् व्याचष्टे ---
पूर्वपरा ।सर्वानामानी॑ति॒विभाषा जसी॑ति चानुवर्तते । तदाह --- एतेषामिति ।
पूर्वादिसप्तानामित्यर्थः । गण इति । सर्वादिगण इत्यर्थः । या प्राप्तेति
।सर्वादीनि॑त्यनेन नित्या संज्ञाया प्राप्तेत्यर्थः । अनेन पूर्वपरेति
सूत्रं गणपठितं जसोऽन्यत्र नित्यतया सर्वनामसंज्ञार्थम् । अष्टाध्यायीपठितं
तु जसि तद्विकल्पार्थमिति न पौनरुक्त्यमिति सूचितम् । स्वाभिधेयेति ।
अपेक्ष्यत इत्यपेक्षः । कर्मणि घ । स्वस्य=ततश्चनियमेनावधिसापेक्षार्थे
वर्तमानानां पूर्वादिशब्दानां जसि सर्वनामसंज्ञाविकल्प इति फलति ।
व्यवस्थायां किमिति । पूर्वादिशब्दानां नियमेनावधिसापेक्ष एवार्थे
विद्यमानत्वादिति प्रश्नः । दक्षिणा गाथका इति । अत्र दक्षिणशब्दो
नावध्यपेक्ष इति भावः । दक्षिणपार्ावर्तिनो गाथका इत्यत्र
कस्मादित्यवध्यपेक्षा अस्त्येवेत्यत आह --- कुशला इत्यर्थ इति । यद्यपि
प्रावीण्यमपि कस्मादित्यवध्यपेक्षमेव, तथापिउत्तरे प्रत्युत्तरे च शक्त॑
इत्यादि प्रत्युदाहरणं बोध्यमित्याहुः । असंज्ञायां किमिति
।संज्ञोपसर्जनीभूतास्तु न सर्वादयः॑ इति वक्ष्यमाणतया संज्ञायां
सर्वनामत्वस्याऽप्रसक्तेरिति प्रश्नः । उत्तराः कुरव इति । कुरुशब्दो
दोशविशेषे नित्यं बहुवचनान्तः । सुमेरमवधीकृत्य तत्रोत्तरशब्दो वर्तत
इत्यस्तीह व्यवस्था । किं तु संज्ञाशब्दत्वान्नास्य सर्वनामता ।
पूर्वादिशब्दानां तु दिक्षु अनादिस्सङ्केत इति न ते संज्ञाशब्दाः । कुरुष
तूत्तरशब्दस्याधुनिकस्सङ्केत इति भवत्ययं संज्ञाशब्द इति मन्यते ।
केचित्त्वसंज्ञायामित्यस्या.ञभावे संज्ञायामेव
पूर्वादिशब्दानामप्राप्तविभाषा स्यादित्याहुः ।
