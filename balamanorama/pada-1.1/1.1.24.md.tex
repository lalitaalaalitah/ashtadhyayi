\textless{}\textgreater{} - पञ्चन्शब्दो नित्यं बहुवचनान्तः । तस्य
षट्संभाकार्यं लुकं विधास्यन् षट्संज्ञामाह --- ष्णान्ता षट् । ष्च नश्च
ष्णौ । ष्टुत्वेन णः । ष्णौ अन्तौ यस्याः सा ष्णान्ता ।बहुगणवतुडति
सङ्ख्या॑ इत्यतः सङ्ख्येत्यनुवर्तते । तच्च पूर्वसूत्रे बहुगणवतुडतिपरमपि
शब्दाधिकारादिह पञ्च षडित्यादिप्रसिद्धसङ्ख्याबोधकशब्दपरमाश्रीयते,
बहुगणवतुडतिषु ष्णान्तत्वाऽसम्भवात् । तदाह --- षान्तेत्यादिना । षड्भ्यो
लुगिति ।अनेन जश्शसोर्लु॑गिति शेषः । पञ्च पञ्चेति । जश्शसोर्लुकि नलोप इति
भावः । सङ्ख्या किमिति । सङ्ख्याग्रहणानुवृत्तेः किं फलमिति प्रश्नः ।
विप्रुषः पामान इति विप्रुष्शब्दस्य पामन्शब्दस्य च ष्णान्तत्वेऽपि
सङ्ख्यावाचकत्वाऽभावेन षट्संज्ञाविरहात्ततः परस्य जसो लुगिति भावः । ननु
शतशब्दाज्जश्शसोः शिभावे ``नपुंसकस्य झलचः'' इति नुमिसर्वनामस्थाने चे॑ति
दीर्घे शतानीति रूपम् । एवं सहरुआआणीत्यपि रूपम् । तत्रअट्कुप्वाङि॑ति
णत्वं विशेषः । इह ``तदागमाः'' इति न्यायेन
नुमोऽङ्गभक्तत्वाच्छतन्शब्दसहरुआन्शब्दयोर्नान्तसङ्ख्याशब्दत्वात्
षट्संज्ञायां सत्यांषड्भ्यो लु॑गिति जश्शसोर्लुक् स्यादत आह ---
शतानीत्यादि । सर्वनामेति । सर्वनामस्थानं परत्वेन उपजीव्य प्रवृत्तस्य
नुमः सन्निपातपरिभाषया सर्वनामस्थानभूतजश्शसोर्लुकं प्रति
निमित्तत्वाऽभावादित्यर्थः । पञ्चभिः पञ्चभ्य इति । नलोपे रूपम् ।
