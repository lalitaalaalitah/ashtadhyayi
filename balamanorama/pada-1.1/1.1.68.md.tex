\textless{}\textgreater{} - ॒अग्नेर्ढक्॒वाय्वृतुपित्रुषसो यत्राज्ञो
य॑दित्यादौ लौकिकव्युत्त्पत्या उपस्थितानां वह्निवातादीनामर्थानां
ढगादिप्रत्ययैः पौर्वापर्यासम्भवात्प्रातिपदिकादित्यनेनान्वयासंभवाच्च
तत्तदर्थकपर्यायशब्दानां ग्रहणापत्तौ तन्नियमार्थमिदं सूत्रमारभ्यते ---
स्वं रूपम् ।अग्नेर्ढगित्यादौ अग्न्यादिशब्दस्य यत्स्वरूपं श्रुतं तदेव
अग्न्यादिशब्दै प्रत्येतव्यं, नतु तदन्यस्तत्तत्पर्यायोऽपि । शब्दशास्त्रे
संकेतिता वृद्धिगुणादिसंज्ञा शब्दसंज्ञा, तत्र नायं नियम॑ इत्यर्थः । तदाह
--- शब्दस्य स्व रूपं संज्ञीति । बोध्यमित्यर्थः । न च वृद्धिर्गुण
इत्यादिसंज्ञाविधिबलादेव तत्र तदर्थग्रहणं भविष्यतीति
किमशब्दसंज्ञेत्यनेनेति वाच्यम्,उपसर्ग घोः कि॑रित्यत्र ``घु शब्दे'' इति
घुधातुनिवृत्त्यर्थत्वात् ``दाधा ध्वदाप्'' इति
संज्ञाकरणस्यघुमास्थागापाजहातिसां हलि॑ इत्यादौ आवश्यकतया संज्ञाकरणस्य
सामर्थ्योपक्षयादित्यन्यत्र विस्तरः । इदं सूत्रं भाष्ये प्रत्याख्यातम् ।
