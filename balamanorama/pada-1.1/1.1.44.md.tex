\textless{}\textgreater{} - न वेति विभाषा ।मेध्यः पशुरनड्वान्विभाषितः॑
इत्यादियाज्ञिकप्रयोगे विभाषाशब्दः केवलविकल्पे दृष्टः । इह तु शास्त्रे
निषेधो विकल्पश्चेति द्वयं मिलितं विभाषाशब्दार्थ इति बोधयितुमिदमारभ्यते ।
इतिशब्दाभावे, स्वं रूपं शब्दस्येति नवाशब्दयो
स्वरूपपरत्वान्नवाशब्दयोर्विभाषासंज्ञेत्यर्थः स्यात् । ततश्च ``विभाषा ओ''
रित्यादौ नवाशब्दा४वादेशौ स्याताम् । इतिकरणे तु नायं दोषः । इतिर्हि
प्रत्येकं संबध्यते । ततश्च ``न'' इति शब्देन योऽर्थो गम्यते निषेधः,
``वा'' इति शब्देन योऽर्थो गम्यते विकल्पस्तदुभयस्य मिलितस्य विभाषासंज्ञा
स्यादित्यर्थः फलति । एवंच नवाशब्दार्थयोरेव संज्ञित्वं लभ्यते, न तु
नवाशब्दस्वरूपयोरिति नोक्तदोषः । तदाह-निषेधविकल्पयोरित्यादि ।
उभयत्रविभाषार्थमिदं सूत्रम् । तथाहि --- (प्राप्तविभाषा अप्राप्तविभाषा
उभयत्रविभाषेति) त्रिविधा विभाषा । प्राप्तविभाषा यथा-विभाषा जसीति ।
वर्णाश्रमेतरे वर्णाश्रमेतरा इत्यत्र हि द्वन्द्वे चेति नित्यतया
सर्वनामसंज्ञानिषेधे प्राप्ते विभाषेयम् । अप्राप्तविभाषा यथा --- तीयस्य
ङित्सु विभाषेति । द्वितीयस्मै द्वितीयायेत्यादौ तीयप्रत्ययस्य सर्वादिगणे
पाठाऽभावादप्राप्ताया\#ं सर्वनामसंज्ञायां विभाषेयम् । उभयत्रविभाषा यथा
--- विभाषा ओरिति । आयतेर्लिटि यङि च संप्रसारणविभाषेयम् । तत्र लिटि
शुशाव, शिआआय, शुशुवतुःशिआयतुरित्याद्युदाहरणम् । यङि तु शोशूयते इति ।
अत्र यङंशेऽप्राप्तविभाषैवेयम् । लिटि तु
द्विवचनबहुवचनेष्वपित्सुवचिस्वपियजादीनां किति॑ इति नित्यतया सम्प्रसारणं
प्राप्तम्, पित्सु त्वेकवचनेषु सम्प्रसारणं न प्राप्तमेव ।असंयोगाल्लिट्
कित् इति कित्त्वस्य अपित्स्वेव प्रवृत्तेः । एवं च प्राप्तेऽप्राप्ते
चारम्भात्विभाषा ओ॑रित्युभयत्रविभाषेति स्तितिः । तत्र यदि ``नवेति
विभाषा'' इति सूत्रं नारभ्येत, तर्हि ``अनड्वान्विभाषितः''
इत्यादियाज्ञिकप्रयोग इव विभाषा ओरित्यत्रापि केवलविकल्पः प्रतीयेत ।
भावोऽभावश्चेति द्वयं तावद्विकल्पः । ततश्च विभाषाश्रुतौ
प्रवृत्तिस्दभावश्चेति द्वयमपि विधेयमिति लभ्यते । तत्र यदिविभाषा
श्वेः॑इति विकल्पो विधिमुखः --- ॒लिटि आयतेः सम्प्रसारणं भवति न भवती-ति
तर्हि पित्स्वेव विकल्पस्य प्रवृत्तिः स्यात् । तत्र हि सम्प्रसारणस्य
वचिस्वपीति किति विहितस्य अप्राप्तत्वेन प्रथमं भवनांशो विधेयः, तस्य
पाक्षिकत्वाय न भवतीत्यपि विधयम् । कित्सु तु प्रवृत्तिर्न स्यात् । तेषु
हि वचिस्व पियजादीनां कितीति प्राप्तत्वात् प्रथमं भवनांशो न विधेयः । न
भवतीत्यंश एव विधेयः । एवं च उभयांशविधेयत्वाऽलाभात्तत्र विकल्पविधिरयं न
प्रवर्तेत, तत्र नित्यमेव सम्प्रसारणं स्यात् । यदि तु विकल्पो निषेधमुखः
--- लिटि आयतेः सम्प्रसारणं न भवति भवती॑ति, तर्हि कित्स्वेव प्रवृत्तिः
स्यात् । तत्र हि वचिस्वपीति प्राप्तत्वान्न भवतीति प्रथमं विधेयम् ।
अभवनस्य पाक्षिकत्वलाभाय भवतीत्यपि विधेयम् । पित्सु तु प्रवृत्तिर्न
स्यात् । तत्र सम्प्रसारणस्याऽप्राप्ततया न भवतीत्यंशस्य प्रथमं
विध्यनर्हत्वात् । न च पित्सु विधिमुखः कित्सु निषेधमुख इत्युभयथापि
प्रवृत्तिरिति वाच्यम् । सकृच्छतस्य विभाषाशब्दस्य
क्वचिद्विधिमुखविकल्पबोधने क्वचिन्निषेधमुखविकल्पबोधने च असामथ्र्यात् ।
आवृत्त्या तद्बोधने तु स एव दोषः ।नवेति विभाषे॑त्यारम्भे तु
श्रुतक्रमानुरोधेन बोधान्नेत्यंशेन कित्सु पूर्वं निषेधः प्रवर्तते । ततः
किदकिद्रूपे सर्वस्मिन् लिटि निःसम्प्रसारणतया ऐकरूप्यं प्रापिते सति, भवति
न भवतीत्येकरूपेण विधिमुख एव विकल्पः प्रवर्तते । तदेवमुभयत्रविभाषार्थमिदं
सूत्रम् । प्राप्तविभाषायां तु नास्योपयोगः, तत्र भवनांशस्य प्राप्तत्वेन
विध्यनर्हत्वात् । अप्राप्तविभाषायामपि न तस्योपयोगः, तत्र अभवनांशस्य
सिद्धत्वेन विध्यनर्हत्वात् । नचैवमपिउणादयो
बहुलं॒॑ह्मकोरन्यतरस्याम्॒छन्दस्युभयथा॑ ``अनुपसर्गाद्वा'' इत्यादिविधिषु
विभाषाशब्दाऽभावात् केवलविकल्पविधौ वैरूप्यं दुर्वारमिति वाच्यम्,
विभाषाशब्दस्याऽत्र सूत्रे विकल्पवाचकशब्दोपलक्षणत्वात् । एवं च लोके ये
विकल्पपर्यायाः शब्दास्ते सर्वेऽस्मिन् शास्त्रे निषेधविकल्पयोः प्रत्यायका
इति सूत्रार्थपर्यवसानं बोध्यम् । भाष्ये तु विभाषादिशब्दानां लोकवदेव
केवलविकल्पपरत्वेऽपि लक्ष्यानुरोधेनैव क्वचिद्विधिमुखेन क्वचिन्निषेधमुखेन
विकल्पस्य प्रवृत्युपपत्तेरेतत्सूत्रं प्रत्याख्यातमित्यलं बहुना ।
