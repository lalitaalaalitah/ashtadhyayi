\textless{}\textgreater{} - ननु सुध् य् इत्यत्र ईकारस्थानिकस्य यकारस्य
स्थानिवद्भावेनाच्त्वाद्च्परकत्वात्कथं धकारस्य द्वित्वमिति शङ्का ह्मदि
निधाय तस्य स्थानिवद्भावप्रापकसूत्रमाह-स्थानिवदादेशः । गुरुस्थानापन्ने
गुरुपुत्रादौ स्थानापत्त्या तद्धर्मलाभो लोकतः सिद्धः ।
कुशादिस्थानापन्नेषु शरादिषु च वैदिकन्यायसिद्धः । इह तु शास्त्रे स्वं
रूपं शब्दस्येति वचनात्स्थानिधर्मा आदेशेषु न प्राप्नुयुरिति
तत्प्राप्त्यर्थं स्थानिवदादेश इत्यारब्धम् । ``स्थानं प्रसङ्ग''
इत्युक्तम् । यस्य स्थानेऽन्यद्विधीयते तत्स्थानि । येन विधीयमानेन
अन्यत्प्रसक्तं निवर्तते स आदेशः । स्थानिना तुल्यः स्थानिवत् ।तेन
तुल्य॑मिति वतिप्रत्ययः । आदेशः स्थानिना तुल्यो भति । स्थानिधर्मको भवतीति
यावत् । अलिति वर्णपर्यायः । विधीयत इति विधिः=कार्यम् । अलाश्रयो विधि
अल्विधिः अनल्विधिः । अलाश्रयभिन्ने कार्ये कर्तव्ये इति प्रतीयमानोऽर्थः ।
अलाश्रयकार्ये कर्तव्ये स्थानिवन्न भवतीति फलितम् । अलाश्रयेति
सामान्यवचनात् अला विधिः, अलः परस्य विधिः, अलो विधिः लि विधिश्चेति
सर्वसंग्रहः । अला विधौ यथा-व्यूढोर स्केन । अत्र विसर्जनीयस्य सः॑ इति
विसर्गस्थानिकस्य सकारस्य व\#इसर्गत्वमाश्रित्य अड्व्यवाय इति णत्वं
प्राप्तं न भवति । अलः परस्य विधौ यथा-द्यौः ।दिव औ॑दिति वकारस्थानिकस्य
औकारस्य स्थानिवद्भावेन हल्त्वात्ततः परस्य सोर्हल्ङ्यादिलोपः प्राप्तो न
भवति । अलो विधौ यता द्युकामः ।दिव उ॑दिति वकारस्थानिकस्य उकारस्य
स्थानिवद्भावेन वकारत्वात्लोपो व्योर्वली॑ति लोपः प्राप्तो न भवति । अलि
विधौ यथा-क इष्टः । यजेः क्तः । अत्र यकारस्थानिकसंप्रसारणस्य इकारहस्य
स्थानिवद्भावेन हश्त्वात् हशि चे॑त्युत्वं प्राप्तं न भवति । अल् चेह
स्थानिभूतः, स्थान्यवयवश्च गृह्रते । ततश्च आदेशस्य स्थानिभूतो योऽल्,
स्थान्यवयवश्च योऽल्, तदाश्रयविधौ न स्थानिवदिति फलति । तत्र
स्थानीभूताऽल्विधौ व्यूढोरस्केनेत्युदाह्मतमेव । यथा वा-धिवि प्रीणन इति
धातोर्लटि प्रथमपुरुषस्य झेरन्तादेशे ``धिन्विकृण्व्योर च'' इति विकरणस्य
उकारस्य यणि वकारे सति तस्य स्थानिवद्भावेनार्धधातुकत्वात् स्वतो
वलादित्वाच्च इडागमः प्राप्तो न भवति, वकारस्य स्थानिभूतो योऽल् उकारः,
तदादेशं वकारमार्धधातुकत्वेनाश्रित्य प्रवर्तमानस्य इटः
स्थान्यलाश्रयत्वात् । स्थान्यवयवालाश्रयविधौ यथा प्रतिदीव्य । इह
क्त्वादेशस्य य इत्यस्य स्थानिवद्भावेन बलाद्यार्धधातुकत्वादिडागमः
प्राप्तो न भवति । इडागमस्य वलादित्वविषये स्थान्यवयवभूतालाश्रयत्वात् ।
तदेतदाह-आदेशः स्थानिवत्स्यान्नतु स्थान्यलाश्रयविधाविति ।
स्थान्यलाश्रयेत्यत्र स्थानीति किम् । रामाय । इहसुपि चे॑ति दीर्घस्य
यञादिसुबाश्रयस्य आदेशगतयकाररूपालाश्रयत्वेऽपि तस्मिन् कर्तव्ये यादेशस्य
स्थानिवद्भावेन सुप्त्वं भवत्येव, दीर्घस्य आदेशगतयकाररूपालाश्रयत्वेऽपि
स्थान्यलाश्रयत्वाऽभावात् । न च नीञ्धातोर्ण्बुलि अकादेशे वृद्धौ नै-अक इति
स्थिते ऐकारस्य स्थानिवद्भावेन ईकारधर्मकत्वादायादेशो न स्यात्, ईकारस्य
आयादेशाभावादिति वाच्यम्, इह हि स्थानिप्रयुक्तं यत् कार्यं शास्त्रीयं
तदेवातिदिश्यते । ईकारस्य च आयादेशभावो न शास्त्रविहित इति न तस्य
ईकारस्थानिके ऐकारे अतिदेश इत्यास्तां तावत् । अनेनेति । उदाह्मतेन
स्थानिवत्सूत्रेण इह=सुध्य् इत्यत्र ईकारस्थानिकस्य यकारस्य स्थानिवद्भावेन
अच्कार्यकारित्वमाश्रित्य अचि न द्वित्वमित्यर्थकेन अनचीत्यनेन धकारस्य
द्वित्वनिषेधो न शङ्कनीय इत्यर्थः । कुत इत्यत आह --- अनल्विधाविति
तन्निषेधादिति । स्थानिवत्त्वनिषेधादित्यर्थः । यकारादेशस्थानीभूतो योऽल्
ईकारः तद्गतमच्त्वं यकारे आश्रित्य प्रवर्तमानस्य यकारद्वित्वनिषेधस्य
स्थान्यलाश्रयत्वादिति भावः ।
