\textless{}\textgreater{} - ऊँ । इदमप्येकपदं सूत्रम् । ऊँ इति
दीर्घस्याऽनुनासिकस्य ऊकारस्य लुप्तप्रथमाविभक्तिकस्य निर्देशः । उञ
इत्यनुवर्तते, इतौ शाकल्यस्य प्रगृह्रमिति च । तदाह --- उञ इताविति । ऊँ
ईतीति । उक्तविधे ऊँकारादेशे रूपम् । प्रगृह्रत्वात्प्रकृतिभावः ।
एतदादेशाऽबावपक्षे पूर्वसूत्रेण प्रगृह्रत्वे सतिउ इती॑ति रूपम् ।
प्रगृह्रत्वस्याप्यभावे सति यणादेशेविती॑ति रूपमिति त्रीणि रूपाणि फलितानि
। तदेवमुञ ऊँ इत्येकमेव सूत्रं विभज्य व्याख्यातम् । एकसूत्रत्वे तु उञ इतौ
परे ``ऊँ'' इत्ययं दीर्घोऽनुनासिकः प्रगृह्रश्चादेशः शाकल्यामते स्यात् ।
तदभावपक्षे तुनिपात एका॑जिति नित्यं प्रगृह्रत्वमित्येतावल्लभ्येत । ततश्च
``ऊँ इति''उ इती॑ति रूपद्वयमेव स्यात्,विती॑ति रूपं न लभ्येत । अतो विभज्य
व्याख्यातम् ।
