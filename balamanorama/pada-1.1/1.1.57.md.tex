\textless{}\textgreater{} - ननु सुध् य् इत्यत्र मास्तु स्थानिवदिति
सूत्रेण स्थानिवद्भावः । तद्गुत्तरसूत्रेण तु स्थानिवत्त्वं स्यादेवेति
शङ्कामुद्भावयिष्यंस्तथाविधमुत्तरसूत्रमाह-अचः परस्मिन् ।
स्थानिवत्सूत्रेणैव सिद्धे किमर्थमिदं सूत्रमित्यत आह-अल्विध्यर्थमिति ।
अलाश्रयविधावपि स्थानिवद्भावार्थमित्यर्थः । तेन वव्रश्चेति सिध्यति ।ओ
व्रश्चू च्छेदने॑ । लिटि तिपि णलि द्वित्वम् ।लिठभ्यासस्ये॑ति अभ्यचासे
रेफस्य सम्प्रसारणम् । ऋकारः । पूर्वरूपम् । उरदत्वम् । रपरत्वम् ।
हलादिश्शेषः । तत्राभ्यासे वकारस्य पुनः सम्प्रसारणं न, ञकारस्थानिकस्य
उरदत्वस्य स्थानिवद्भावेन सम्प्रसारणतयान सम्प्रसारणे सम्प्रसारणम् इति
निषेधात् । पूर्वसूत्रेण त्वत्र स्थानिवद्भावो न सम्भवति ---
सम्प्रसारणनिषेधस्य स्थान्यलाश्रयत्वादिति भावः । पूर्वसूत्रादिह
स्थानिवदादेश इत्यनुवर्तते । अच इत्येतदादेश इत्यनेनान्वेति-अच आदेश इति ।
परहस्मिन्निति सति सप्तमी । ततश्च परनिमित्तक इति लभ्यते ।
तच्चादेशविशेषणम् । तदाह --- परनिमित्तोऽजादेशैति । विधीयत इति
बिधिः=कार्यम् । पूर्वस्य विधिः पूर्वविधिः । पूर्वत्वं च यद्यपि सावधिकम्,
त्रयं चात्र संनिहितं-स्थानी आदेशः परनिमित्तं चेति । तत्र स्थानी
तावन्नावधिर्भवितुमर्हति, तस्यादेशेनापहारात् । नाप्यादेशः, नापि
परनिमित्तम्, वैयाकरण इत्यत्र इकारस्थानिकयणादेशात्तत्परनिमित्तादाकाराच्च
पूर्वस्य न य्वाभ्यामित्यैकारस्य आयटादेशे कर्तव्ये यणादेशस्य
स्थानिवद्भावेनाऽच्त्वापत्तेः । तथापि स्थान्यपेक्षयैवाऽत्र पूर्वत्वं
विवक्षितं, स्थानिन आदेशे नापह्मतत्वेऽपि भूतपूर्वगत्या तत्पूर्वत्वस्य
सम्भवात् । तदेतदाह --- स्थानिभूतादचः पूर्वत्वेनेत्यादि । अत्र स्थानिनि
सति यद्भवति तदादेशे ।ञपि भवति, यन्न भवति तदादेसेऽपि न
भवतीत्येवमशास्त्रीयस्यापि कार्याऽभावस्य अतिदेशो भवति । तत्राद्ये
वव्रश्चेत्युदाह्मतमेव, तत्रन सम्प्रसारणे सम्प्रसारण॑मिति निषेधकार्यस्य
शास्त्रीयत्वात् । द्वितीये तु गणयतीत्युदाहरणम् । गण संख्यान इति चुरादौ
अदन्तधातुः । तस्माण्णिच् । अतो लोपः । तिप् शप् णेर्गुणः, अयादेशः ।
गणयतीति रूपम् । अत्र णिचि परत उपधाभूतस्य गकारादकारस्य ``अत उपधायाः'' इति
न भवति, प्रकृतसूत्रेणाऽल्लोपस्य स्थानिवद्भावात् । अकारे स्थानिनि सति
गकारादकारस्य उपधात्वभङ्गादुपधावृद्धिर्न प्रवृत्तिमर्हति ।
वृद्ध्यभावस्याशास्त्रीयत्वेऽपि अल्लोपे अतिदेशात् । न चात्र गकारादकारस्य
उपधात्वभङ्गादुपधावृद्धिर्न प्रवृत्तिमर्हति । वृद्ध्य
भावस्याशास्त्रीयत्वेऽपि अल्लोपे अतिदेशात् । न चात्र गकारादकारस्य
स्थान्यकारान्न पूर्वत्वम्, णकारेण व्यवधानादिति वाच्यम्, पूर्वत्वं ह्रत्र
व्यवहिताऽव्यवहितसाधारणम्, उत्तरसूत्रे स्वरे निषेधाल्लिङ्गात् । तच्च
तत्रैव स्पष्टीभविष्यतीत्यलम् । इति यण इति । अनेन सूत्रेण सुध् य् इत्यत्र
धकारस्य द्वित्वनिषेधे कर्तव्ये ईकारस्थानिकस्य यकारस्य स्थानिद्भावे
प्राप्ते तत्प्रतिषेधसूत्रमारभ्यत इत्यर्थः ।
