\textless{}\textgreater{} - अणुदित्सवर्णस्य । प्रत्ययशब्दस्य
अणादिप्रत्ययपरत्वे ``त्यदादीनामः'' ``इदमैश्'' इत्यादीनां पर्युदासो न
स्यादित्यतो व्याचष्टे --- प्रतीयत इति । उत् इत् यस्य सः उदित=कु चु टु तु
पु इत्यादिः । चकारात्स्वं रूपमित्यतः स्वमित्यनुवर्तते । तच्च षष्टन्ततया
विपरिणम्यते । तदाह-अविधीयमान इत्यादिना । अणिति पूर्वेण परेण वा
प्रत्याहार इति संशये निर्धारयति-अत्रेति । अस्मिन्नेव सूत्रे अण् परेण
णकारेण, इतरत्र तुअणोऽप्रगृह्रस्ये॑त्यादौ पूर्वेणैवेत्यर्थः । अत्र च
आचार्यपारंपर्यौपदेशरूपं व्याख्यानमेव शरणम् । एवं चाणुदित्सूत्रेणानेन
अकारादिभिश्चतुर्भिदीर्घप्लुतानामिव सवर्णभूतहकारादीनामपि
ग्रहणादच्त्वातेषु परेषु इकारस्य यणादिकं स्यादिति नाज्झलाविति प्रतिषेध
आवश्यक इति स्थितम् । अणुदित्सूत्रस्य फलमाह --- तदेवमिति ।
तत्णुदित्सूत्रम् । एवं=वक्ष्यमाणप्रकारेण फलतीत्यर्थः । तिंरशत इति ।
ऋलृवर्णयोर्मिथः । सवर्णतया ऋकारेण स्वाष्टादशभेदानाम्लृकारीयद्वादशभेदानां
च ग्रहणादिति भावः । एवम्लृकारोऽपीति । ऋकारस्यापि लृकारसवर्णत्वादिति भावः
। नन्वेकारेण ऐकारप्रपञ्चोऽपि गृह्रते, ऐकारेण एकारप्रपञ्चश्च । तथा ओकारेण
औकारप्रपञ्चो गृह्रेत, औकारेण ओकारप्रपञ्चश्च । ततस्च एचश्चतुवशतेः संज्ञा
स्युरित्येवं वक्तव्यं नतु द्वादशानामित्यत आह --- एदेतोरिति । कुतो न
सावण्र्यमित्यत आह --- ऐऔजिति । यदि ह्रेदैतोः ओदौतोश्च परस्परं सावण्र्यं
स्यात्तर्हि एकारेण ऐकारप्रपञ्चस्य, ओकारेण औकारप्रपञ्चस्य च
अकारादिभिदीर्घप्लुतानामिव ग्रहणसम्भवात्ऐऔ॑जिति सूत्रं नारभ्येत । आरभ्यते
च (सूत्रकृता) । अत एदैतोरोदौतोश्च न परस्परं सावण्र्यमिति विज्ञायत
इत्यर्थः । अच् इच् एच् इत्यादि प्रत्याहारास्तु ङकारेणैव निर्वाह्राः । नच
``एचोऽयवायाव'' इत्यत्र यथासंख्यार्थमैऔजिति सूत्रमस्त्विति वाच्यम् । तत्र
स्थानेऽन्तरतमः॑ इति सूत्रेणैव निर्वाहस्य वक्ष्यमाणत्वादिति भावः ।
वस्तुतस्तुऐऔ॑जिति सूत्राऽभावे ``वृद्धिरादैच्''न त्वाभ्यां पदान्ताभ्यां
पूर्वौ तु ताभ्यामैच् ``प्लुतावैच् इदुतौ'' इत्यादौ एङ्ग्रहणापत्तौ
एदोतोरपि ग्रहणे प्रसक्ते तन्निवृत्त्यर्थमैच्प्रत्याहार आवश्यक इति
तदर्थमैऔजित्यारम्भणमीयमेव । अत ऐऔजिति सूत्रारम्भस्य
चरितार्थत्वादेदैतोरोदौतोश्च मिथऋ सावण्र्याऽभावसाधकत्वकथनमनुपपन्नमेव ।
एदैतोरोदौतोश्च मिथः सावण्र्याभावस्तु वृद्दिरादैजित्यादौ
क्वचिदैज्ग्रहणात् ``अदेङ्गुणः'' इत्यादौ क्वचिदेङ्ग्रहणाच्च सुनिर्वाहः ।
अन्यथा सर्वत्र एङ्ग्रहणमेव ऐज्ग्रहणमेव वा कुर्यात् । तावतैव चतुर्णां
ग्रहणसम्भवात् । अत ऐच् एङिति प्रत्याहारद्वयग्रहणसामथ्र्यादेदैतोरोदौतोश्च
न मिथः सावण्र्यम् । ``प्लुतावैच इदुतौ''एचोऽप्रगृस्ये॑ति
प्रत्याहारद्वयग्रहणवैयथ्र्याच्चेति शब्देन्दुशेखरे प्रपञ्चितम् । तेनेति ।
एदैतौरोदौतोश्च मिथस्सावण्र्याभावेनेत्यर्थः । नापादनीयमिति ।
नाशङ्कनीयमित्यर्थः । एवं च एकारेण सह वर्तत इति सैः, हे सैरित्यत्र
``एङ्ह्यस्वात्'' इति संबुद्धिलोपो न । ग्लावं ग्लाव इत्यत्र
``औतोऽम्शसोः'' इत्यात्वं च न । स्यादेतत् । हकारस्य आकारस्य च सवर्णसंज्ञा
स्यात् , स्थानप्रयत्नसाम्यात्, अज्झलामेव सावण्र्यनिषेधात्,
वार्णसमाम्नायिकानामेन वर्णानामज्झलशब्दवाच्यत्वात्, आकारप्रश्लेषे च
प्रमाणाऽभावात् । न चाकारस्याच्त्वात्तेन आकारस्यापि अणौदित्सूत्रेण
ग्रहणादाकारहकारयोर्न सावण्र्यमिति वाच्यं, ग्रहणकसूत्रे हि लब्धात्मकमेव
सत् ``अस्य च्वौ'' इत्यादौ प्रवृत्तिमर्हति । नाज्झलाविति प्रवृत्तिदशायां
च ग्रहणकशास्त्रं न लब्धात्मकम् । तद्धि सवर्णपदघटितं,
सवर्णपदार्थावगमोत्तरम\#एव लब्धात्मकम् । सवर्णसंज्ञाविधायकं च
तुस्यास्यसूत्रं सामान्यतः त्वर्थं बोधयदपि नाज्झलावित्यपवादविषयं
परिह्मत्य तदन्तत्रैव पर्यवसन्नं स्वकार्यक्षमम् । तदुक्तम् ---
॒प्रकल्प्यपवादविषयमुत्सर्गोऽभिनिविशते॑ इति । उक्तं च
भाष्ये-वर्णानामुपदेशस्तावत्, उपदेशोत्तरकाला इत्संज्ञा, इत्संज्ञोत्तरकाल
आदिरन्त्येनेति प्रत्याहारः, प्रत्याहारोत्तरकाला सवर्णसंज्ञा,
तदुत्तरकालमणुदित्सूत्रमित्येतेन समुदितेन वाक्येनात्यत्र सवर्णानां ग्रहणं
भवती॑ति । अन्यत्र=॒अस्य च्वौ॑ इत्यादावित्यर्थः । अत्र
भाष्येप्रत्याहारोत्तरकाला सवर्णसंज्ञे॑त्यनेन नाज्झलाविति निषेधसहितः
सावण्र्यविधिर्विवक्षितः , केवलसावण्र्यविधेः प्रत्याहारानपेक्षत्वेन
प्रत्याहारोत्तरकालिकत्वनियमाऽसम्भवात् । तथा चाणुदित्सूत्रस्य नाज्झलाविति
निषेधसहिततुल्यास्यसूत्रप्रवृत्तेः प्रागलब्धात्मकत्वात्तेन
नाज्झलावित्यत्र अज्ग्रहणेन सवर्णानां
ग्रहणाऽभावात्सावण्र्यविधिनिषेधाभावादकारहकारयोः सावण्र्यं स्यादिति शङ्कते
--- नाज्झलाविति सावण्र्येत्यादिना । यद्यपीति सम्भावनायाम् ।
अक्षरसमाम्नायः=चतुर्दशसूत्री । तत्र भवा आक्षरसमाम्नायिकाः
।बह्वचोऽन्तोदात्तात् इति ठञ् । न च नाज्झलविति
प्रवृत्तिदशायामणुदित्सूत्रप्रवृत्त्यभावेऽपि तत्र अजित्यनेन लक्षणया
दीर्घप्लुतानां ग्रहणमस्तु प्रत्याहाराणां स्ववाच्यवाच्येषु लक्षणाया
अनुपदमेव प्रपञ्चितत्वादिति वाच्यम् । स्ववाच्यवाच्येषु हि प्रत्याहाराणां
लक्षणा, न चात्राच्छब्दवाच्याकारादिवाच्यता दीर्घप्लुतानामस्ति ।
अमुदित्सूत्रस्येदानीमप्रवृत्तेरिति भावः । परिहरति --- तथापीति ।
वार्णसमाम्नायिकानामेव नाज्झलाविति निषेध इत्यभ्युपगमेऽपि हकारस्य आकारो न
सवर्ण इत्यर्थः । कुत इत्यत आह --- तत्रापीति । अपिशब्दो व्युत्क्रमः ।
तत्र=नाज्झलाविति सूत्रे, आसहितोऽच् आचित्याकारस्यापि सवर्णदीर्घेण
प्रश्लिष्टत्वादित्यर्थः ।नन्वस्तु हकारस्य आकारस्य च सावण्र्यं, किं
तत्प्रतिषेधार्थेन आकारप्रश्लेषेणेत्यत आह --- तेनेति । तेन=हकारस्य
आकारस्य च सावण्र्यप्रतिषेधेन, हकारेण आकारस्य
ग्रहणाऽभावाद्विआपाभिरित्यत्र ``हो ढः'' इति हकारस्य विधीयमानं ढत्वं
पकारादाकारस्य न भवति । आकारप्रश्लेषाऽभावे तु, तस्य हकारस्य च
सावण्र्यसत्वाद्धकारेण आकास्य च ग्रहणात्तस्य ढत्वं स्यादित्यर्थः । अत्र
ढत्वस्यासिद्धत्वात्संयोगान्तलोप एवापादनीय \#इति नवीनाः ।कालसमयवेलासु
तुमुन् इति सूत्रे वेलास्विति लकारादाकारस्य निर्देशो नाज्झलावित्यत्र
आकारप्रश्लेषे प्रमाणम् । अन्यथा तत्र ढत्वस्य संयोगान्तलोपस्य
वाऽ‌ऽपत्तावाकारो न श्रूयेतेत्यलम् । ननु ग्रहणकसूत्रेऽज्ग्रहणमेव
क्रियताम्, अणुदित्सवर्णस्येति किमण्ग्रहणेन, हयवरलानां सवर्णाऽभावेन तेषु
ग्रहणकशास्त्रस्य व्यर्थत्वादित्यत आह --- अनुनासिकेति । तेनेति ।
उक्तद्वैविध्येन सवर्णत्वात्-अननुनासिकास्ते यवलाः प्रत्येकं द्वयोद्र्वयोः
संज्ञाः । अनुनासिकास्तु यवला अननुनासिकानामपि न संज्ञाः, ``भेदको गुणा''
इत्याश्रयणात्, वर्णसमाम्नायेऽननुनासिकानामेव तेषां पाठाच्च । एवं च
यवलसंग्रहार्थं ग्रहणकसूत्रेऽज्ग्रहणमपह\#आय अण्ग्रहणमिति भावः ।
