\textless{}\textgreater{} - मिदचोऽन्त्यात्परः । मकार इत् यस्य स मित्
अन्त्यादचः परो भवतीत्यर्थे ``शेमुचादीनाम्'' इत्यादाविदं न
प्रवर्तेत,तत्रान्त्यस्याचोऽभावादत आह --- अच इति षष्ठन्तमिति ।यतश्च
निर्धारणमित्यनेने॑ति शेषः । ``अच'' इत्येकत्वमविवक्षितं, तदाह --- अचां
मध्य इत्यादिना । अन्तावयव इति । एतच्च
आद्यान्तावित्यतोऽन्तग्रहणानुवृत्त्या लभ्यते,
आद्यन्तशब्दैकदेशस्यान्तशब्दस्य तन्मात्रे
स्वरितत्वप्रतिज्ञाबलेनाऽनुवृत्तिसंभवात् । आदिग्रहणमनुवत्र्य
परादित्वाभ्युपगमे तु वारीणीति बहुवचने सर्वनामस्थाने चासंम्बुद्धौइति
नान्ताङ्गस्य विहितो दीर्घो न सिध्येत् । अभक्तत्वे तु वहंलिह इत्यत्र
``वहाभ्रे लिहः'' इति खशि,अरुर्द्विष॑दिति मुमि तस्य ``मोऽनुस्वारः'' इति
मान्तस्य पदस्य विहितोऽनुस्वारो न स्यात् । वस्तुतस्तु यस्य समुदायस्य
मिद्विहितस्तस्याऽचां मध्ये योऽन्त्यस्तस्य समुदायस्याऽन्तावयवय इति
व्याख्येयम् । अत एव ``समुदायभक्तो मित्'' इति भाष्यं सङ्गच्छते ।
समासाश्रयविधौ मूकारश्च वक्ष्यति --- ॒अङ्गस्य नुम्विधानात्तद्भक्तो हि
नु॑मिति ।
