\textless{}\textgreater{} - अनेकाल्शित्सर्वस्य । न एकोऽनेकः, अनेकोऽल्
यस्य सोऽनेकाल्, शकार इत् यस्य स शित्, अनेकाल्व शिच्चेति समाहारद्वन्द्वः
। स्पष्टमिति । अनुवर्तनीयपदान्तराऽभावादिति भावः ।
अस्तेर्भूरित्याद्युदाहरणम् । नन्वस्धातोर्भूर्भवतीत्युक्ते
कृत्स्नस्यैवादेशः प्राप्त इति किमर्थमिदं सूत्रमारभ्यत इत्यत
आह-अलोऽन्त्यसूत्रापवाद इति ।अलोन्त्येति॑ति सूत्रैकदेशानुकरणम् ।
अनुकरणत्वादेव नापशब्दः, अधिरीआर इति सूत्रैकदेशस्य प्राग्रीआरान्निपाता
इति ग्रहणाल्लिङ्गात् । स्यादेतत् । अष्टन्शब्दाज्जसि शसि चअष्टन आ
विभक्ता॑विति आत्वे अष्टा अस् इति स्थिते अष्टाभ्य औशिति कृताकारादष्टनः
परयोर्जश्शसोर्विधीयमान औशादेशोऽलोऽन्त्यस्येति बाधित्वा आदेः
परस्येत्यादेरकारस्य प्राप्तः । अनेकाल्त्वाच्च सर्वादेशः प्राप्तः ।
एवम्अतो भिस ऐ॑सित्यादावपि । तत्र कतरच्छास्त्रं बाध्यं कतरच्च प्रवर्तत
इत्यत्र किं विनिगमकम्रित्यत आह-अष्टाभ्य औशित्यादाविति । आदिनाअतो भिस
ऐ॑सित्यादिसंग्रहः । अष्टाभ्य औ॑ शित्यादावादेः परस्येत्येतदपि परत्वादनेन
बाध्यत इत्यन्वयः । अस्तेर्भूरित्यादौ अनेकाल्शिदित्यनेन
यथाऽलोऽन्त्यस्येति बाध्यते तथ\#आऽष्टाभ्या औशित्यादावादेः परस्येत्येतदपि
बाध्यत इत्यर्थः । नन्वस्तेर्भूरित्यादावलोऽन्त्यस्येति प्राप्ते
सत्येवानेकाल्शित्सर्वस्येति नापवादः । अस्तेर्भूरित्यादावादेः
परस्येत्यप्राप्तावपि तत्प्रवृत्तेरित्यत आह-परत्वादिति ।विप्रतिषेधे परं
कार्य॑मिति तुल्यबलविरोधे परप्राबल्यस्य वक्ष्यमाणत्वादिति भावः ।आदेः
परस्ये॑त्यस्यावकाशोद्व्यन्तरुपसर्गेभ्योऽप ई॑दित्यादिः
।अनेकाल्शित्सर्वस्ये॑त्यस्यावकाशः-॒अस्तेर्भूः॒॑इदम इ॑शित्यादिः ।
अतस्तुल्यबलत्वमुभयोः ।
