\textless{}\textgreater{} - अथ स्वरादिचादिभिन्नान्यव्ययान्याह ---
तद्धितश्चासर्व । असर्वविभक्तिरिति बहुर्वीहिः । तत्र सर्वा विभक्तयो
यस्मान्न भवन्तीति बहुवचनान्तविग्रहो न संभवति, अव्ययेभ्यः सप्तानां
विभक्तीनामुत्पत्त्यभ्युपगमात् । तथाहितद्धितश्चे॑ति प्रकृतसूत्रे भाष्ये
तावत्द्व्येकयोर्द्विवचनैकवचने॑बहुषु बहुवचन॑मिति सूत्रविन्यासं भङ्क्त्वा
``एकवचनं''द्वयोर्द्विवचने॒॑बहुषु बहुवचन॑मिति सूत्रन्यासं कृत्वा
एकवचनमुत्सर्गतः करिष्यते, द्विबह्वोरर्थयोस्तस्य द्विवचनबहुवचने बाधके
इत्यादि स्थितम् । ततश्च एकवचन॑मित्यनेन ङ्याप्प्रातिपदिकादेकवचनं भवतीति
सामान्यविधिना द्वित्वबहुत्वाऽभावे एकवचनमिति लभ्यते । एवं च
द्विबहुत्वाऽभावे सति एकत्वे तदभावे च एकवचनमिति फलति । तत्र
द्वित्वबहुत्वयोर्द्विवचनबहुवचनोक्त्यैव ततोऽन्यत्र एकवचनस्य सिद्धत्वात्
``एकवचनम्'' इति सूत्रं कर्मत्वाद्यभावेऽपि प्रापणार्थं संपद्यते । तथाच
अलिङ्गसंख्येभ्योऽव्ययेभ्य एकवचनं प्रवर्तमानं
विनिगमनाविरहात्सर्वविभक्त्येकवचनं भवति । अत एव ``अव्ययादाप्सुपः''
इत्यत्र प्रत्याहारग्रहणमर्थवत् । तस्मात्सर्वा विभक्तियो यस्मादिति न
विग्रहः, किन्तु सर्वशब्दोऽत्र सर्व पटो दग्ध इतिवदवयवकार्त्स्न्ये वर्तते
। एवंच सर्वा वचनत्रयात्मिका विभक्तिर्यस्मान्नोत्पद्यते ।
किन्त्वेकवचनान्येवोत्पद्यन्ते, स तद्धितान्तोऽव्ययसंज्ञः स्यादिति
फलतीत्यभिप्रेत्याह --- यस्मादिति । सर्वेति । वचनत्रयात्मिकेत्यर्थः ।
नोत्पद्यत इति ।किन्त्वेकवचनान्येवोत्पद्यन्ते॑ इति शेषः ।
स्यादेतत्-तिङ्श्चे-त्यनुवृत्तौप्रशंसायां रूप॑मिति
रूपप्प्रत्ययेईषदसमाप्तौ कल्प॑बिति कल्पप्प्रत्यये च पचतिरूपं पचतिकल्पमिति
रूपम् । प्रशस्तं पचति, ईषत् पचतीत्यर्थः । अत्राप्यव्ययत्वं स्यात्,
अरुआवविभक्तितद्धितान्तत्वात् । किञ्च उभयशब्देऽतिव्याप्तिः,
तस्याप्यसर्वविभक्तितद्धितान्तत्वादित्यत आह-परिगणनमिति । वार्तिकमेतत् ।
तसिलादय इति । ``पञ्चम्यास्तसिल्'' इत्यारभ्यद्वित्र्योश्च धमु॑ञित्यर्थः ।
शस्प्रभृतय इति ।बह्वल्पार्था॑दित्यरभ्यअव्यक्तानुकरणा॑दिति डाजन्ता
इत्यर्थः । अम् आमिति ।अमु च च्छन्दसी॑त्यम्,किमेत्तिङव्यये॑त्यमा च
गृह्रते । कृत्वोऽर्था इति ।सङ्ख्यायाः क्रियाभ्यावृत्तिगणने
कृत्वसुच्,द्वित्रिचतुभ्र्यःसुच्,विभाषा बहोर्थे॑ति त्रय इत्यर्थः । तसिवती
इति ।तेनैकदिक्, ``तसिश्च'' इति तसि\#ः ``तेन तुल्यम्'' इत्यादिविहितो
वतिश्च गृह्रते ।प्रतियोगे पञ्चम्यास्तसिः॑ इत्यस्य तु शस्प्रभृतित्वादेव
सिद्धम् । एवं च स्वरादिषु वदित्यस्य प्रयोजने चिन्त्यम् । नानाञाविति
।निनञ्भ्यां नानाञौ न सहे॑ति विहातौ नानाञौ । इति परिगणनंकर्तव्यमित्यन्वयः
। परिगणनेनैव सिद्धेतद्धितश्चे॑ति सूत्रं न कर्तव्यमिति भावः ।
