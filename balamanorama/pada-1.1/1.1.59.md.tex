\textless{}\textgreater{} - द्विर्वचनेऽचि । द्विरुच्यते येन परनिमित्तेन
तद्द्वर्वचनं । द्वित्वनिमित्तमिति यावत् । अचीत्यस्य विशेषणमिदम् ।अचः
परस्मिन्नित्यतोऽच इति, ``स्थानिवदादेश'' इत्यत आदेश इति, न पदान्तेत्यतो
नेति चानुवर्तते । द्विर्वचन इत्यावर्तते । एवं चद्वित्वे कर्तव्ये सती॑
त्यपि लभ्यते । तदाह --- द्वित्वनिमित्तेऽचीत्यादिना ।द्वित्वे कर्तव्ये
सती॑त्युक्तेः कृते द्वित्वे ``चक्रे'' इत्यादौ यणादयो भवन्ति । अन्यता तु
न स्युः, द्वित्वनिमित्तस्य अचः सत्त्वात् । द्वित्वनिमित्त इति किम् ।
दुद्यूषति । दिव्धातोः सनि द्वित्वात्परत्वादूठि कृते द्वित्वात्प्राग्यण्
भवत्येव । तथा च ``द्यू'' इत्यस्य द्वित्वे दुद्यूषतीति सिध्यति । द्वित्वे
कर्तव्ये यणो निषेधे तु दिद्यूषतीत्यभ्यासे इकार एव श्रूयेत । न तूकारः ।
``द्वित्वनिमित्ते'' इत्युक्तौ तूठि परे द्वित्वात्प्राग्यण्यो न निषेधः,
ऊठो द्वित्वनिमित्तत्वाऽभावात् । अचीति किम् । जेघ्रीयते । अत्र
घ्राधातोर्यङि द्वित्वात्प्राकई घ्राध्मो॑रितीकारादेशो न निषिध्यते ।
ईत्वस्य द्वित्वनिमित्त यङ्निमित्तकत्वेऽपि
द्वित्वनिमित्ताऽज्निमित्तकत्वाऽभावात् । अचः किम् । असूषुपत् । इह
स्वापेश्चङि द्वित्वात्प्राक्स्वापेश्चङी॑तिवकारस्य सम्प्रसारणं न
निषिध्यते, तस्याऽजादेशत्वाऽभावात् । ततश्च कृते सम्रसारणे सुप् इत्यस्य
द्वित्वेऽब्यासे उकारस्य श्रवणं संभवति । संप्रसारणे निषिद्दे तु ``स्वप्''
इत्यस्य द्वित्वेऽभ्यासे उकारो न श्रूयेत । एवं च प्रकृते
यणादेशात्प्राग्लिटि धातो॑रिति द्वित्वे कृ कृ ए इति स्थिते --- ।
