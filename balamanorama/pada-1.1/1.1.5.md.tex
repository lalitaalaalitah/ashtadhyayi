\textless{}\textgreater{} - क्ङिति च । ग् क् ङ् एषां समाहारद्वन्द्वः,
कात्पूर्वगकारस्य चर्त्वेन निर्देशात् । ग् क् ङ् च इत् यस्येति विग्रहः ।
द्वन्द्वान्ते श्रूयमाण इत्यर्थ । इक इत्युचार्य विहिते इति लभ्यते ।न
धातुलोप आद्र्धधातुके॑ इत्यतो नेत्यनुवर्तते । तदाह ---
गित्किन्ङिन्नमित्ते इत्यादिना । गितीत्यनुक्तौ तुग्लाजिस्थश्च ग्स्नु॑रिति
ग्स्नुप्रत्ययेजिष्णु॑रित्यत्र गुणनिषेधो न स्यात् । न च ग्स्नुप्रत्ययः
किदेव क्रियतामिति वाच्यं, तथा सतिघुमास्थे॑ति किति विहितस्य ईत्वस्य
प्रसङ्गात् । यदि तु गिति ङिति परतो गुणवृद्धी न स्त इति व्याख्यायेत, तदा
च्छिन्नं भिन्नमित्यत्र क्तप्रत्यये परे लघुपधगुणनिषेधो न स्यात् ।
स्थानिभूतस्येको हला व्यवधानात् । न च येन नाव्यवधानन्यायः शङ्क्यः । चितं
स्तुतमित्यादावव्यवहिते चरितार्थत्वात् । यदि च ``इको गुणवृद्धी'' इत्येव
व्याख्यायेत, न त्विग्लक्षणे इति, तदा लिगोरपत्यं लैगवायनः, नडादित्वात्
फक्, इह आदिवृद्धेरोर्गुणस्य च वस्तुगत्या इक्स्थानिकत्वान्निषेधः
स्यादित्यलम् । भूयादिति । इहाद्र्धधातुकत्वाल्लिङ सलोप
इत्यस्याऽप्रवृत्तेः स्कोरिति सलोप इत्युक्तं न विस्मर्तव्यम् । न चैवमपि
संयोगादिलोपस्याऽसिद्धत्वाद्धल्ङ्यादिलोपः स्यादिति वाच्यं, सुटि यासुटि च
सति ताभ्यां विशिष्टस्यैव प्रत्ययत्वेनाऽपृक्तत्वाऽभावादित्यलम् ।
भूयास्तामिति । आशिषि लिङस्तस्तामादेशे आद्र्धधातुकत्वाच्छभावे
यासुडागमेऽतः परत्वाऽभावादियादेशाऽभावे सुटि झल्परसंयोगदित्वेन यासुटः
सकारस्य लोपो, गुणनिषेधश्च । भूयासुरिति । झेर्जुसि यासुडागमे गुणनिषेधे
रूपम् । भूया इति । आशीर्लिङः सिपि इतश्चेतीकारलोपः । यासुटःस्को॑रिति
सलोपः, गुणनिषेधः, रुत्वविसर्गौ । भूयास्तमिति । थसस्तमादेशे यासुटि
गुणनिषेधः । एवं थस्य तादेशेऽपि भूयास्तेति रूपम् । भूयासमिति ।
मिपोऽमादेशे यासुटि गुणनिषेधः । भूयास्वेति । लिङो वस् । ``नित्यं ङित''
इति सकारलोपः । यासुट् । गुणनिषेधः । एवं मसि भूयास्मेति रूपम् ।
इत्याशीर्लिङ्प्रक्रिया ।
