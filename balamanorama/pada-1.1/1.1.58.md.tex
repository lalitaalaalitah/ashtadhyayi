\textless{}\textgreater{} - न पदान्तद्विर्वचन । स्थानिवदादेश॑ इतिअचः
परस्मि॑न्निति चानुवर्तते । परनिमित्तकोऽजादेशो न स्थानिवदित्यन्वयः ।
पदान्तश्च द्विर्वचनं च वरे च यलोपश्च स्वरश्च सवर्णश्च अनुस्वारश्च
दीर्घश्च जश्च चर्चेति द्वन्द्वः । तेषां विधयः=विधानानि । कर्मणि षष्ठआ
समासः । ततश्च पदान्तादिषु विधेयेषु इति लभ्यते । वर इत्यनेन वरे योऽजादेशः
स विवक्षितः । आर्षो द्वन्द्वः । सप्तम्या अलुक्च । विधिशब्दः
प्रत्येकमन्वेति --- पदान्तविधौ द्विर्वचनविधावित्यादि ।
पदस्यान्तः=चरमावयवः । पदान्तस्य विधाने=पदान्तकर्मके विधाने । पदस्य
चरमावयवे कार्ये द्विर्वचनादौ च कार्ये इति यावत् । तदाह --- पदस्य
चरमेत्यादिना । पदान्तस्य स्थाने विधाविति तु न व्याख्यातम् , तथा सति एषो
यन् हसतीत्यसिद्धेः । तथाहि --- एषः यन् इति छेदः । इण्धातोर्लटः शतरि शपि
लुकि इकारस्य इणो यणिति यण् । अत्र एतत्तदोरिति सुलोपो न भवति, तस्य हलि
परतो विधानात्, इह च तस्मिन् कर्तव्ये इकारस्थानिकस्य यकारस्य
स्थानिवद्भावेनाऽच्त्वात् । नच न पदान्तेति निषेधः शङ्ख्यः, यो विधीयमानः
पदस्य चरमावयवः संपद्यते तत्रैव तन्निषेधात्, इह च विधेयस्य सुलोपस्य
पदानवयवत्वात् ।पदान्तस्य स्थाने विधा॑विति व्याख्याने तु इह यकारस्य
स्थानिवद्भावो न सिध्येत्, लोपस्य पदान्तसकारस्य स्थाने विधानात् । अत्र
हशि चेत्युत्वे तु कर्तव्ये यकारो न स्थानिवद्भवति, उकारस्य विधीयमानस्य
पदचरमावयवत्वात् । एवं च पदान्तविधावित्यस्य एषो यनित्येतदुत्वविषये
उदाहरणम् । सुलोपविषये तु प्रत्युदाहरणमिति भाष्ये स्पष्टम् ।
बाष्यप्रदीपोद्द्योते स्पश्टतरमेतत् । पदान्तविधौ कानि
सन्तीत्याद्युदाहरणमनुपदमेव मूले स्पष्टीभविष्यति । द्विर्वचने सुध् य्
इतयुदाहरणम् । नचेह द्वित्वे कर्तव्ये यकारस्य स्थानिवद्भावविरहेऽपि
तन्निषेधे स्थानिवद्भावः स्यादेवेति वाच्यम्, अनचि चेति
द्वित्वस्याऽनैमित्तिकतयातद्विषये यकारस्थानिवद्भावस्यानपेक्षितत्वेन तत्र
तन्निषेधस्य वैयथ्र्यापत्त्या द्विर्वचनशब्देनात्र अचि नेति
द्वित्वनिषेधस्यैव विवक्षितत्वादिति भावः । वरे यथा --- यायावरः । ``यश्च
यङ'' इति याधातोर्यङन्ताद्वरच् ।सन्यङो॑रिति द्वित्वम् । यायाय वर इति स्थि
अतो लोप इति यङोऽकारस्य लोपः ।लोपो व्यो॑रिति यकारलोपः । अत्र
अजाद्यार्धधातुकमाश्रित्य ``अतो लोप इटि च'' इत्याकारलोपे कर्तव्ये अल्लोपो
न स्थानिवत् । यलोपे यता --- यातिः । याध\#आतोर्यङि द्वित्वं । क्तिच् ।
यायाय ति इति स्थिते-अतो लोपः ।लोपो व्यो॑रिति यलोपः । अल्लोपस्य
स्थानिवत्त्वादातो लोपः । लोपो व्योरिति यलोपः । यातिरिति रूपम् । अत्र
अल्लोपो यलोपे कर्तव्ये न स्थानिवत् । न च वाय्वोरित्यत्रापि लोपो व्योरिति
यलोपे कर्तव्ये उकारादेशस्य वकारस्य स्थानिवत्त्वनिषेधः स्यादिति वाच्यं,
स्वरदीर्घयलोपेषु लोप एवाजादेशो न स्थानिवदिति वार्तिके परिगणनात्, इह च
वकारस्य लोपरूपत्वाऽभावात् । स्वरविधौ यथा --- चिकीर्षकः । चिकीर्ष इति
सन्नन्तात् ण्वुल् । अकादेशः । सनोऽकारस्य अतो लोपः । अत्र ईकारस्य
वितीत्युदात्तत्वे कर्तव्ये अल्लोपो न स्थानिवत् । यद्यपि
ईकारोऽल्लोपस्थानीभूतादकारान्नाव्यवहितपूर्व इति स्थानिवद्भावस्य
प्राप्तिरिह नास्ति, तथाप्यस्मादेव ज्ञापकात् पूर्वसूत्रे पूर्वत्वं
व्यवहिताव्यवहितसाधारणम् । तत्प्रयोजनं तु पूर्वसूत्र एवोक्तम् ।सवर्णविधो
यथा --- शिण्ड्ढि । शिष् इति धातो रौधादिकाल्लोण्मध्यमपुरुषैकवचनम् । सिप्
श्नम् । शिनष् सि । हित्वम् । धित्वम् । ष्टुत्वम् । षस्य जश्त्वं डकारः ।
शिनड्ढि । श्नसोरल्लोपः । नश्चापदान्तस्येत्यनुस्वारः । अनुस्वारस्य ययीति
तस्य परसवर्णो णकारः । शिण्ड्ढि इति रूपम् । अत्र परसवर्णे कर्तव्ये
अल्लोपो न स्थानिवत् । वस्तुतस्तु सवर्णविधौ नेदमुदाहरणम् । श्नमकारस्य
लोपोऽत्र अजादेशः । तत्स्थानीभूतः श्नमकार एव । तस्मिन् सति
नकारस्यानुस्वारप्रसक्तिरेव नास्ति । तथा चानुस्वारस्य स्थानीबूतादचः
पूर्वत्वेन कदाप्यदृष्टत्वात्तस्य परसवर्णे कर्तव्ये अचः
परस्मिन्नित्यलोपस्य स्थानिवत्त्वं न प्रसक्तमिति किं तत्प्रतिषेधेन ।
यत्तु तत्वबोधिन्याम --- अनुस्वारस्य स्थानिभूतो नकारः श्नमकारात्
पूर्वत्वेन दृष्ट इति तत्स्थानिकानुस्वारस्यापि तत्पूर्वत्वेन दृष्टत्वं,
स्थानिद्वारापि पूर्वत्वाभ्युपगमादित्युक्तम् । एवं सति तितौमाचष्टे
तितापयतीत्यत्र पुगागमो न स्यात् । ``तत्करोति तदाचष्टे'' इति णिच् ।
इष्ठवद्भावादुकारस्य टेरिति लोपः । अचो ञ्णितीति तकारादकारस्य वृद्धिराकारः
। पुगागमः । तितापीत्यस्मात् लट् तिप् शप् । गुणः । अयादेश\#ः । तितापयतीति
रूपम् । अत्र अचो ञ्णितीति वृद्ध्या लब्धस्य आकारस्य पुगागमे कर्तव्ये
उकारलोपस्य स्थानिवद्भावे सति उकारेण व्यवधाने णिपरकत्वाऽबावात्पुगागमो न
स्यात् । आकारस्य लोपस्थानिभूतादुकारात्पूर्वत्वस्य स्वतोऽभावेऽपि
स्थानिद्वारा सत्त्वादिति सिद्धान्तरत्नाकरे दूषितम् ।
प्रौढमनोरमाव्याख्याने तु शब्दरत्ने पादमाचष्टे पादयति, ततः क्विप् पात्
हसतीत्यादौ ``झयो हः'' इति पूर्वसवर्णे कर्तव्ये पूर्वस्मात्परस्य विधिरिति
पञ्चमीसमासप्राप्तस्थानिवद्भावनिषेधार्थमिह सूत्रे सवर्णग्रहणिति
प्रपञ्चितम् । अनुस्वारविधौ यथा --- शिंषन्ति । शिष्धातोर्लटि झिः ।
झोऽन्तः । श्नसोरल्लोपः । नस्चापदान्तस्येत्यनुस्वारः । शिंषन्ति । इह तु न
परसवर्णः, षकारस्य यय्त्वाभावात् । अत्र अनुस्वारे कर्तव्ये अल्लोपो न
स्थानिवत् । अत्र दीर्घविधौ यथा-प्रतिदीव्ना । हलि चेति दकारादिकारस्य
दीर्घे कर्तव्ये अल्लोपो न स्थानिवत् । जइआधौ यथा --- ॒सग्धिश्च मे॑ । अद
भक्षणे क्तिन् । बहुलं छन्दसीति घस्लादेशः । ``घसिभसोर्हलि च''
इत्युपधालोपः ।झलो झली॑ति सलोपः । झषस्तथोरिति तकारस्य धत्वम् । ``झलाञ्जश्
झशि'' इति जश्त्वेन घकारस्य गकारः । समाना ग्धिः=अदनं-सग्धिः । समानस्य
छन्दसीति संभावः । अत्र जश्त्वे कर्तव्ये उपधालोपो न स्थानिवत् । चर्विधौ
यथा --- जक्षतुः । घसेर्लिटि अतुस् । द्वित्वम् । ``हलादिश्शेषः''अभ्यासे
चर्चे॑ति जश्त्वम् । ``कुहोश्चुः'' इति जकारः ।गमहने॑त्युपधालोपः । ``खरि
च'' इति चर्त्वं ककारः ।शासिवसी॑ति षः । अत्र चर्त्वे कार्ये उपधालोपो न
स्थानिवत् । भाष्ये तुपूर्वत्रासिद्धे न श्थानिव॑दित्यवष्टभ्य
द्विर्वचनसवर्णानुसारदीर्घजश्चरः प्रत्याख्याताः । किञ्चदीर्घा
दाचार्याणा॑मित्युत्तरम् --- ॒अनुस्वारस्य ययि परसवर्णः॑ ``वा पदान्तस्य''
``खरि च'' ``वावसाने''अणोऽप्रगृह्रस्यानुनासिकः॑ इति पञ्चसूत्रीपाठ इति
``हलो यमां यमि'' इति सूत्रस्थबाष्यसंमतः सूत्रक्रमः ।एवं चशिण्ड्ढी॑त्यत्र
न परसवर्णप्रसक्तिः । परसवर्णे कर्तव्ये षकारस्थानिकस्य
जश्त्वस्यासिद्धत्वेन यय्परकत्वाऽभावादिति सर्वथापि सवर्णविधौ शिण्ड्ढीति
नोदाहसरणमित्यास्तां तावत् । इति स्थानीति । अनेन सूत्रेण सुध् य् इत्यत्र
द्वित्वनिषेधे कर्तव्ये यकारस्य स्थानिवत्त्वनिषेध इत्यर्थः । एवं
चाच्परकत्वाभावेन अनचि चेति निषेधाभावाद्धकारस्य द्वित्वं निर्बाधमिति भावः
। तथाच सुध् ध् यिति स्थितम् ।
