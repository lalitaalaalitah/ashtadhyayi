\textless{}\textgreater{} - ईदूतौ च ।प्रगृह्र॑मित्यनुवर्तते । तच्च
द्विवचनान्ततया विपरिणम्यते । शब्दस्वरूपस्य विशेष्यत्वात्तदन्तविधिः
।ईदूतौ च सप्तम्या॑वित्येव सिद्धेऽर्थग्रहणाद्यत्र सप्तम्या लुकियः शिष्यते
स लुप्यमानार्थाभिधायी॑ इति न्यायेन प्रकृतेरेव सप्तम्यर्थे पर्यवसानं
तथाविधत्वमीदूदन्तयोर्गम्यते । तथाच सप्तम्यर्थे पर्यवसन्नावीदूदन्तौ शब्दौ
प्रगृह्रौ स्त इत्यक्षरार्थः । फलितमाह --- सप्तम्यर्थ इत्यादिना । सोमो
गौरी इति । गौर्यामित्यर्थः ।सुपां सुलु॑गिति सप्तम्या लुक् । प्रगृह्रत्वे
प्रकृतिभावान्न यण् । वातप्रमीत्यादिसप्तम्यन्तं तु नात्रोदाहरणम्, तत्र
सप्तम्या लुप्तत्वाऽभावेन प्रकृतेः सप्तम्यर्थेऽप्रवृत्तेः । मामकी तनू इति
। मामक्यां तन्वामित्यर्थः ।सुपां सुलु॑गिति सप्तम्यालुक् । प्रगृह्रेभ्यः
परत इतिशब्दप्रयोगस्य पदकारैर्नियमितत्वात् पदपाठे ``मामकी इति''तनू
इती॑त्यत्र प्रगृह्रत्वफलं बोध्यम् । ननु ``ईदूतौ च सप्तम्याः'' इत्येव
सूत्र्यताम् । षटआ चाऽर्थद्वारा संबन्धो विवक्ष्यतां, ततश्च सप्तम्यर्थे
विद्यमानमीदूदन्तमित्यर्थस्यार्थग्रहणं विनैव लाभादर्थग्रहणं किमर्थमिति
पृच्छति --- अर्थग्रहणं किमिति । कस्म\#ऐ प्रयोजनायेत्यर्थः
।कि॑मित्यव्ययम् । वृत्ताविति । अर्थग्रहणसामथ्र्याल्लुप्तसप्तम्यर्थमात्रे
पर्यवसन्नमित्यर्थो विवक्षितः । ततश्च समासवृत्तौ लुप्तसप्तमीके
ईदूदन्तपूर्वपदे सप्तम्यर्थमतिलङ्घ्य उत्तरपदार्थे प्रवृत्ते सति
प्रगृह्रसंज्ञा न भवति । मा भूदिति । ``माङि लुङ्'' सर्वलकारापवादः ।
वाप्यआ इति । ``वाप्याम् --- अआ'' इति विग्रहे सुप्सुपेति समासे ``वाप्यआ''
इति रूपमित्यर्थः । अत्र वाप्यामिति सप्तम्या अधिकरणत्वमवगतं,
तच्चाधिकरणकारकं क्रियापेक्षं । तत्र वाप्यामआओ वर्तत इति क्रियाध्याहारे
वर्तमानक्रियायां वाप्या विद्यमानेऽओ लक्षमया प्रवृतिंत पुरस्कृत्य समासो
वक्तव्यः । एवं च समासे लुप्तसप्तमीकस्य वापीशब्दस्य सप्तम्यर्थमतिलङ्घ्य
तत्संसृष्टे आधेयभूतेऽओऽपि प्रवृत्तेः सप्तम्यर्थमात्राविश्रान्त्यभावान्न
प्रगृह्रत्वमिति भावः ।
