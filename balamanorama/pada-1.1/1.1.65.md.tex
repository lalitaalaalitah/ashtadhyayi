\textless{}\textgreater{} - सखन्-स् इति स्थिते उपधाकार्यं
वक्ष्यन्नुपधासंज्ञामाह --- अलोन्त्यात् । ``अल'' इति पञ्चमी,अन्त्या॑दिति
सामानाधिकरण्यात् । अल्प्रत्याहारो वर्णपर्यायः । पूर्वोऽप्यलेव गृह्रते,
साजात्यादित्याह --- अन्त्यादल इत्यादिना । अलः किम् । ``शिष्ट'' इत्यत्र
शास्धातौ आसिति संघातात्पूर्वशकारस्योपधात्वं न भवति । अन्यथाशास
इदङ्हलो॑रिति शकारस्येकारप्रसङ्गः । वर्णग्रहणं किम् । शास्धातौ ``शा'' इति
समुदायस्य उपधात्वं न भवति । अन्यथा ``शा'' इति समुदायस्य इकारः स्यात् । न
चालोन्त्यपरिभाषया आकारस्यैव इकारो भवतीति
वाच्यं,नानर्थकेऽलोन्त्यविधि॑रिति तन्निषेधात् ।
