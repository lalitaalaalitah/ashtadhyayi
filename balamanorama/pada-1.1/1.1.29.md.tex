\textless{}\textgreater{} - तत्रेदमारभ्यते --- ॒न बहुव्रीहौ॑ ।सर्वादीनि
सर्वनामानी॑त्यनुवर्तते । बहुव्रीहौ सर्वादीनि सर्वनामानि न स्युरित्यर्थः
प्रतीयते । एवं सति सूत्रमिदं व्यर्थं,प्रियसर्वाये॑त्यादीनां
बहुव्रीहिवर्तिनां सर्वादीनां
स्वार्थोपसङ्क्रान्तार्थान्तप्रधानकतयोपसर्जनत्वादेव
सर्वनामत्वनिषेधसिद्धेः,संज्ञोपसर्जनीभूतास्तु न सर्वादयः॑ इति
वक्ष्यटमाणत्वात् । अतो व्याचष्टे --- चिकीर्षित इति । बहुव्रीहाविति
विषयसप्तम्याश्रयणादयमर्थो लभ्यते । तथाच बहुव्रीहौ प्रसक्ते सति ततः
प्रागेव विग्रहवाक्येऽयं निषेधोर्थवान् । एकार्थीभावात्मकसामथ्र्यस्य
समासदशायामेव सत्त्वेन विग्रहवाक्ये तदभावेन
तदानीमुक्तोपसर्जनत्वस्याऽभावादिति भावः । अथ लौकिकविग्रहवाक्यं
दर्शयल्लँक्ष्यभूतं बहुव्रीहिं दर्शयति --- त्वकं पितेत्यादिना ।
सर्वनामत्वाऽभावात्कप्रत्ययेप्रत्ययोत्तरपदयोश्चे॑ति त्वमादेशे त्वत्को
मत्क इति च रूपम् । ननु बहुव्रीहिप्रवृत्तेः प्रागलौकिकविग्रहवाक्ये
सर्वनामत्वनिषेधात्त्वं पितेति कथं लोकिकविग्रहवाक्यप्रदर्शनमित्यत आह ---
इहेति ।न बहुव्रीहा॑वित्यस्मिन्सूत्र इत्यर्थः । प्रक्रियावाक्य इति ।
युष्मद् स्-पितृ स्, अस्मद् स्-पित\#ऋ स् --- इत्यलौकिकविग्रहवाक्य
एवेत्यर्थः । लोकिकविग्रहवाक्ये तु नायं निषेधः, बहुव्रीहिवत्तस्य स्वार्थे
परिनिष्ठितत्वेन स्वतन्त्रप्रयोगाह्र्मतया बहुव्रीहेस्तत्र
चिकीर्षितत्वाऽभावादलौकिकविग्रहात्मके प्रक्रियावाक्य एव तस्य
चिकीर्षितत्वात् । यथा चैतत्तथा समासनिरूपणे वक्ष्यते
नन्वलौकिकविग्रहवाक्ये मास्तु सर्वनामतानिषेधः, को दोषः । तत्राह-अन्यथेति
। न बहुव्रीहावित्यलौकिकविग्रहवाक्ये निषेधाऽभावे सतीत्यर्थः । तत्रापीति ।
अलौकिकविग्रहवाक्येऽपीत्यर्थः ।नन्वलौकिकविग्रहवाक्ये भवत्कच् । सत्यप्यकचि
तस्य प्रयोगानर्हत्वेन बाधकाऽभावादित्यत आह --- स चेति अलौकिकविग्रहवाक्ये
श्रुतस्य लौकिकविग्रहवाक्ये समासे च श्रवणनियमादिति भावः । उभयत्रापि
तन्नियमे दृष्टान्तद्वयमाह --- अतिक्रान्तो भवकन्तमित्यादि । भवच्छबन्दस्य
सर्वादिगणे पाठात्सर्वनामत्वादलौकिकविग्रहदशायामकच् । ततश्च भवकत् अम् अति
इत्यलौकिकविग्रहवाक्यं संपद्यते । तत्रअत्यादयः क्रान्ताद्यर्थे
द्वितीयये॑ति समासे सतिसुपो धातुप्रातिपदिकयो॑रिति सुब्लुकि
अतिभवकच्छब्दात्प्रथमैकवचने ।ञतिभवकानिति रूपम् । समासाऽभावपक्षे तु
भवकन्तमतिक्रान्त इति लौकिकविग्रहवाक्यं भवति । तत्र समासदशायां
भवच्छब्दार्थस्य स्वोपसङ्कान्तार्थान्तरप्रधानतयोपसर्जनत्वे
।ञप्यलौकिकविग्रहदशायां भवच्छब्दस्यानुपसर्जनत्वात्सर्वनामत्वे सति
प्रवृत्तोऽकच् अतिक्रान्तो भवकन्तमिति लौकिकविग्रहवाक्येऽभवकानिति समासे
चानुवर्तते, लौकिकविग्रहदसायां भवच्छब्दस्योक्तरीत्याऽनुपसर्जनत्वात् ।
समासे तस्योपसर्जनत्वेऽपि योनिभूताऽलौकिकविग्रहदशायां प्रवृत्तस्याऽकचो
निवर्तकाऽभावात् । नच भवत् अम् इत्यलौकिकविग्रहदशायां
सतोऽप्यनुपसर्जनत्वस्य समासदशायां विनाशं प्राप्स्यमानतया
विनाशोन्मुखत्वादकृतव्यूहपरिभाषयाऽलौकिकविग्रहवाक्येऽपि
सर्वनामत्वाऽभावादकज्दुर्लभः । ततश्चाऽतिक्रान्तो भवकन्तमिति
लोकिकविग्रहवाक्येऽतिभवकानिति समासे च कथमकच्प्रसक्त इति
दृष्टान्तऽसिद्धिरिति वाच्यम् । एवञ्जातीयकाऽलौकिकविग्रहवाक्ये
सर्वनामत्वप्रवृत्तावकृतव्यूहपरिभाषाया अनित्यत्वेनाऽनाप्रवृत्तेः ।
तदनित्यत्वे च न बहुव्रीहाविति सूत्रमेव ज्ञापकम् । तथा हि ---
यद्यकृतव्यूहपरिभाषा सार्वत्रिकी स्यात्, तर्हि बहुव्रीहिविषयेऽपि युष्मद्
स् पितृस् इत्याद्यलौकिकविग्रहवाक्येऽनुपसर्जनत्वस्य
बहुव्रीहिकालिकविनाशोन्मुखतया सर्वनामत्वस्याऽप्रसक्तत्वात्न
बहुव्रीहा॑विति नारभ्येत । अकृतव्यूहपरिभाषायास्तत्र
प्रवृत्तेर्भविष्यद्बहुव्रीहिकालिकविनाशोन्मुखमनुपसर्जनत्वं पुरस्कृत्य
तदलौकिकविग्रहवाक्ये सर्वनामत्वस्याऽप्रसक्तत्वान्न तन्निषेधाय न
बहुव्रीहावित्यर्थवत् ।
नचोदाह्मतबहुव्रीहिविषयाऽसोकिकविग्रवाक्येऽकृतव्यूहपरिभाषामाश्रित्यैव
सर्वनामत्वाऽभाव आश्रीयतां, किं न बहुव्रीहाविति सूत्रेणेति वाच्यम् । एवं
सत्यतिक्रान्तो भवकन्तमतिभवकनित्यादि न सिध्येत् । अकृतव्यूहपरिभाषया
तदलौकिकविग्रहवाक्येऽपि सर्वनामत्वाऽभावेनाकचः प्रवृत्त्यभावे
तस्यातिक्रान्तो भवकन्तमित्यादिलौकिकविग्रहवाक्येऽतिभवकानिति समासेऽपि च
श्रवणं न स्यात् । एवञ्च
बहुव्रीहिविषयेऽलौकिकविग्रहवाक्येऽकृतव्यूहपरिभाषाया
अप्रवृत्त्यसिद्धवत्कृत्य
सर्वनामत्वनिषेधात्तदितरसमासविषयेऽप्यलौकिकविग्रहवाक्येऽकृतव्यूहपरिभाषाया
अप्रवृत्त्या सर्वनामत्वं विज्ञायते । एतदर्थमेव न बहुव्रीहाविति
सूत्रमित्यन्यत्र विस्तरः । प्रत्याचख्याविति । निराकृतवानित्यर्थः ।
सूत्रभाष्ययोरुभयोरपि स्मृतित्वाऽविशेषाद्विकल्पमाशङ्कयाह --- यथोत्तरमिति
। सूत्रकाराद्वार्तिककारस्य, उभाभ्यामपि भाष्यकृत इत्येवं
मुनीनामुत्तरोत्तरस्य ग्रन्थस्य प्रामाण्यं, पूर्वपूर्वस्याऽप्रामाण्यमिति
वैयाकरणसमय इति भावः । न चाऽकृतव्यूहपरिभाषाया उक्तरीत्या
अनित्यत्वज्ञापनार्थमेतत्सूत्रमिति वाच्यम्, अकृतव्यूहपरिभाषाया
निमूर्लत्वस्य निष्फलत्वस्य च हलन्तादिकारे सेदिवस्शब्दनिरूपणे,समर्थानां
प्रथमाद्रे॑त्यत्र च वक्ष्यमाणत्वात् । संज्ञोपसर्जनीभूता इति ।
आधुनिकसङ्केतः संज्ञा । अन्यविशेषणत्वेन स्वार्थोपस्थापकम् --- उपसर्जनम् ।
न सर्वादय इति । सर्वादिगणे पठिता न भवन्तीत्यर्थः ।महासंज्ञेति ।
टिघुभादिवदेकाक्षरसंज्ञामकृत्वा सर्वेषां नामानीत्यन्वर्थसंज्ञाकरणबलेन
प्राधान्येनोपस्थितस्वीयसर्वार्थवाचकत्वस्य
सर्वनामशब्दप्रवृत्तिनिमित्तत्वमित्यवगततया तथाविधानामेव सर्वादिगणे
पाठानुमानादित्यर्थः ।प्राधान्येनोपस्थिते॑त्यनेन उपसर्जनव्यावृत्तिः
।प्राधान्येनोपस्थितसर्वार्थवाचकत्व॑मित्युक्ते
पूर्वादिशब्देष्वव्याप्तिरतः-॒स्वीये॑ति । सर्वार्थे॑त्यनेन
संज्ञाशब्दव्यावृत्तिः, संज्ञाशब्दानामेकैकव्यक्तिविषयकत्वात् ।
संज्ञाकार्यमिति । सर्वनामसंज्ञाकार्यं शीस्मायादिकमित्यर्थः । अन्तर्गणेति
। सर्वादिगणेऽन्तर्गतो गणः-अन्तर्गणः । तदीयं कार्यम् =॒अद्ड्डतरादिभ्यः॑
त्यदादीनामः॑ इत्यादिकमित्यर्थः । सर्वाय देहीति ।
संज्ञाशब्दत्वात्समायादेशो न । अतिकतरमिति । कतरमतिक्रान्तं कुलम्
अतिकतरमित्यत्र कतरशब्दस्य उपसर्जनत्वान्नाऽद्डादेशः । अतितदिति ।
तमतिक्रान्तो ब्राआहृणोऽतितदित्यत्र त्यदाद्यत्वं, ``तदोः सः सौ'' इति च न
भवति ।
