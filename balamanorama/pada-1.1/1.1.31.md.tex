\textless{}\textgreater{} - तदाह --- द्वन्द्वे उक्तेति । सर्वनामसंज्ञा
नेत्यर्थः । वर्णाश्रमेतराणामिति ।वर्णाश्रमेतराणां नो ब्राऊहि
धर्मानशेषतः॑ इति याज्ञवल्क्यस्मृतिः । वर्णाश्चाअश्रमाश्च इतरे चेति
द्वन्द्वः । अत्र सर्वनामत्वाऽभावादामि सर्वनाम्न इति न सुट् ।
समुदायस्यैवेति । द्वन्द्वे विद्यमानानि यानि सर्वादीनि तानि सर्वनामानि न
स्युरिति नाऽर्थः, विद्यतिक्रियाध्याहारे गौरवात् । किन्तु द्वन्द्वे
सर्वनामसंज्ञा न भवतीति प्रधानभूतया निषेध्यवनक्रिययैव
द्वन्द्वस्याधारतयाऽन्वयः । द्वन्द्वाधारा सर्वनामसंज्ञा न
भवतीत्यक्षरार्थः । द्वन्द्वस्य सर्वनामसंज्ञा नेति फलितम् ।
वर्णाश्रमतरेत्यादिसमुदायस्यैव द्वन्द्वता नतु तदवयवानाम् । एवंच
वर्णाश्रमेतरेत्यादिसमुदायस्यैव सर्वनामत्वनिषेधो नतु तदवयवानामिति
वस्तुस्तितिकथनम् । ननु द्वन्दावयवानां सर्वनामत्वनिषेधाऽभावे
वर्णाश्रमेतरशब्दे इतरशब्दस्य सर्वनामतया ततः परस्यामः सुटि
वर्णाश्रमेतरेषामिति स्यात् । नच अवर्णान्तात्सर्वनाम्नोऽह्गात्परस्यामः
सुड्विधीयते । ततश्च वर्णाश्रमेतरशब्दस्य समुदायस्य द्वन्द्वतया
सर्वनामत्वनिषेधेऽपि तदवयवस्य इतरशब्दस्य सर्वनामतया
तदन्ताङ्गात्परत्वादामः स\#उट् स्यात् । न चैवं सति द्वन्द्वस्य
तन्निषेधात् । इतरशब्दस्तु सर्वनामसंज्ञकः,न ततो विहित आम्, आमः समुदायादेव
विधानात् । अतो न सुडिति भावः ।अवर्णान्तादङ्गात्सर्वनाम्नो विहितस्यामः
सु॑डिति व्याख्याने तु येषां तेषामित्यत्राऽव्याप्तिः ।
अतोऽवर्णान्तादङ्गात्परस्य सर्वनाम्नो विहितस्यामः सुडित्येव व्याख्येयम् ।
