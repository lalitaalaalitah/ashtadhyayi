\textless{}\textgreater{} - अदसो मात् । ``अदस'' इत्यवयवषष्ठी ।
अदश्शब्दावयवान्मकारादित्यर्थः ।ईदू॑दितिप्रगृह्र॑मिति चानुवर्तते ।मा॑दिति
दिग्योगे पञ्चमी । परसभ्दोऽध्याहार्यः । तदाह --- अस्मात्पराविति ।
अदश्शब्दावयवमकारात्परावित्यर्थः ।ए॑दिति नानुवर्तते, अदश्शब्दे
मकारात्परस्यैकारस्याऽसंभवात् ।द्विवचन॑मित्यपि नानुवर्तते, अदश्शब्दे
मकारात्परस्य ईकारस्य ``अमी'' इति बहुवचनत्वात्, ऊकारस्य च मकारात्परस्य
तत्र द्विवचनान्तेष्वेव सत्त्वेन व्यावर्त्त्याऽभावात् । अमी ईशा इति ।
अदश्शब्दाज्जसि त्यदाद्यत्वं पररूपत्वं । जसश्शी । आद्गुणः । ``अदे'' इति
स्थिते एकारस्यएत ई॑दिति ईत्त्वं, दस्य च मत्त्वम् । तदेवम्-॒अमी॑\#इति
रूपम् । अत्र ईकारस्य द्विवचनत्वा.ञभावात्पूर्वसूत्रेण प्रगृह्रसंज्ञा न
प्राप्तेत्यनेन सा विधीयते । रामकृष्णावमू इति ।
पुँल्लिङ्गाददश्शब्दात्प्रथमाद्विवचने औङि, त्यदाद्यत्वं, पररूपत्वं ।
वृद्धिरेचि । अदौ इति स्थिते, अदसोऽसेरित्यौकारस्य ऊत्वं, दस्य मत्वम् ।
अमू इति रूपम् । यद्यप्ययमूकारो द्विवचनं भवति, तथापि पूर्वसूत्रेण
प्रगृह्रत्वे कर्तव्ये उत्वमत्वयोरसिद्धतया दकारादौकारस्यैव शास्त्रदृष्टआ
सत्त्वात्पूर्वस\#ऊत्रेण तस्य प्रगृह्रत्वं न प्राप्तमित्यनेन विधीयते
।अदसो मा॑दिति सूत्रं प्रति तु उत्वमत्वे नासिद्धे, आरम्भसामथ्र्यात् ।
पूर्वसूत्रस्य तु तत्र न सामथ्र्यं, हरी एतौ, विष्णू इमावित्यादौ
चरितार्थत्वात् ।स्त्रियौ फले वा अमू आसाते॑ इति स्त्रीलिङ्गो
नपुंसकलिङ्गश्च अदश्शब्दो नात्रोदाहरणम् । तथाहि स्त्रीलिङ्गददश्शब्दादौङि,
त्यदाद्यत्वे, पररूपत्वे, टापि, ``ओङ आप'' इति शीभावे, आद्गुणे,
उत्वमत्वयोरमू इत्येव रूपम् । तथा नपुंसलिङ्गात्तस्मादौङि, त्यदाद्यत्वे,
पररूपत्वे, नपुंसकाच्चेति शीभावे, आद्गुणे, उत्वमत्वयोरमू \#इत्येव रूपम् ।
अत्र पूर्वसूत्रेणैव प्रगृह्रत्वं सिद्धम् ।
उत्वमत्वयोरसिद्धत्वेऽप्येकारस्य द्विवचनस्य सत्त्वात् । अतः पुँल्लिङ्ग एव
अदश्शब्दोऽत्रोदाहरणमिति प्रदर्शयितुं ``रामकृष्णा'' वित्युक्तम् ।
मात्किमिति । ``अदस'' इत्येव सूत्रमस्तु, माद्ग्रहणस्य किं प्रयोजनमिति
प्रश्नः । अमुकेत्रेतिअव्ययसर्वनाम्नामकच् प्राक्टेः॑ \#इत्यकचि
अदकश्शब्दाज्जसि, त्यदाद्यत्वं, पररूपत्वम्, जसश्शी, आद्गुणः उत्वमत्वे ।
``अमुके'' इति रूपम् । अत्र एकारस्य प्रगृह्रत्वनिवृत्त्यर्थं माद्ग्रहणम्
। कृते च तस्मिन्नेकारस्य म\#आत्परत्वाऽभावान्न प्रगृह्रत्वमिति भावः ।
नन्वेवमपि माद्ग्रहणं व्यर्थम्, एद्ग्रहणमननुवर्त्त्य ईदूतोरेवाऽत्र
प्रगृह्रत्वविधानाब्युपगमेन ``अमुके'' इत्यत्र
प्रगृह्रत्वप्रसक्तेरेवाऽभावादित्यत आह --- असतीति ।अदसो मा॑दित्यत्र
ईदूदेतामेकसमासपदोपात्तानां मध्ये ईदूतोद्र्वयोरनुवृत्तौ
एतोऽप्यनुवृत्तिप्रसक्तौ माद्ग्रहणादेतोऽनुवृत्तिः प्रतिबद्धा ।
माद्ग्रहणाऽभावे तु बाधकाऽभावादेतोऽप्यनुवृत्तिः स्यात् । ततश्च ``अमुके''
इत्यत्रापि एकारस्य प्रगृह्रत्वप्रसक्तौ तन्निवृत्त्यर्थं माद्ग्रहणम् ।
कृते तु तस्मिन्नेतोऽनुवृत्तिप्रतिबन्धादमुके इत्यत्र न प्रगृह्रत्वम् ।
तथाच एकाराननुवृत्तिफलसकं माद्ग्रहणमिति भावः ।
