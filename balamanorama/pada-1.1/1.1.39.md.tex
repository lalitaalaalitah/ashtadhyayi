\textless{}\textgreater{} - कृन्मेजन्तः । कृत्-मेजन्त इति छेदः । म् च
एच्च मेचौ , तौ अन्ते यस्येति बहुव्रीहिः । तदाह --- कृद्योमान्त इति ।
तदन्तमिति । केवलस्य कृतः प्रयोगाऽनर्हत्वात्संज्ञाविधावपि तदन्तविधिरिति
भावः । स्मारं स्मारमिति ।आभीक्ष्ण्ये णमुल्चे॑ति स्मृधातोर्णमुल्,अचो
ञ्णिती ति वृद्धिः, रपरत्वम्,नित्यवीप्सयो॑रिति द्वित्वं,
मान्तकृदन्तत्वादव्ययत्वम् । जीवसे इति । ``तुमर्थे सेसेनसे''
इत्यादिनाऽसेप्रत्ययः । पिबध्यै इति । ``तुमर्थे से'' इत्यादिना
शध्यैप्रत्ययः । शित्त्वात्सार्वधातुकत्वम् ।पाघ्राध्मे॑ति पिबादेश इति
भावः । शप्तु न, कत्र्रर्थे सार्वधातुके तद्विधेः, ``अव्ययकृतो भावे'' इति
सिद्धान्तादित्याहुः ।
