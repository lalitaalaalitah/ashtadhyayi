\textless{}\textgreater{} - सुद् ध् य् इत्यत्र यकारस्य संयोगान्तलोपं
शङ्कितुं लोपसंज्ञासूत्रमाह --- अदर्शनं लोपः ।
शब्दानुशासनप्रस्तावाच्छब्दविषयकश्रवणंमिह दर्शनं विवक्षितम् ।
दर्शनस्याभावोऽदर्शनम् । अर्थाभावेऽव्ययीभावः । ``स्थानेऽन्तरतमः'' इत्यतः
स्थां इत्यनुवर्तते । स्थानं प्रसङ्ग इत्युक्तम् । शास्त्रतः शब्दस्य
कस्यचिच्छ्रवणप्रसङ्गे सति यदश्रवणं तल्लोपसंज्ञं भवतीत्यर्थः ।
