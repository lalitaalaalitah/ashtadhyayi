\textless{}\textgreater{} - स्थानेऽन्तरतमः । ``स्थानं प्रसङ्ग''
इत्युक्तम् । अन्तरशब्दोऽत्र सदृशपर्यायः, अतिशयतोऽन्तरोऽन्तरतमः । तदाह
--- प्रसङ्गे सतीत्यादिना । एकस्य स्थानिनोऽनेकादेशप्रसङ्गे सति यः
स्थानार्थगुणप्रमाणतः स्थानिना सदृशतमः स एवादेशो भवतीत्यर्थः । अत्र
स्थानशब्देन ताल्वादिस्थानं विवक्षितम् । गुणशब्देन प्रयत्नः ।
प्रमाणशब्देन एकद्विमात्रादिपरिमाणम् । तत्र स्थानतो यथा --- दध्यत्र ।
तालुस्थानकस्य इकारस्य तालुस्थानको यकारः । अर्थतो यथा ---
तृज्वत्क्रोष्टुरिति क्रोष्टुशब्दस्य उकारान्तस्य तृजन्त आदेशो
भवन्नर्थसाम्यात् क्रोष्टृशब्द एव तृजन्त आदेशो भवति । गुणतो यथा ---
वाग्घरिः । अत्र हकारः स्थानी घोषनादसंवारमहाप्राणप्रयत्नवान् । तस्य
गकारसवर्णो भवंश्चतुर्थो घकारो भवति, तस्य हकारेण स्थानिना
घोषनादसंवारमहाप्राणप्रयत्नसाम्यात् । ककारस्तु न भवति, तस्य
आआसाऽघोषविवाराल्पप्राणप्रयत्नकत्वात् । तथा खकारोऽपि द्वितीयो न भवति,
तस्य महाप्राणप्रयत्नसाम्येऽपि आआसाऽघोषविवारप्रयत्नभेदात् । तथा तृतीयो
।ञपि गकारो न भवति, तस्य घोषनादसंवारप्रयत्नसाम्येऽपि निना हकारेण
आआसाऽघोषविवारप्रयत्नबेदे सत्यपि महाप्राणप्रयत्नसाम्यसत्त्वात् , तथा
तृतीयो वा गकारः कुतो न स्यात्, तस्य स्थानिना हकारेण
अल्पप्राणप्रयत्नभेदेऽपि घोषनादसंवारप्रयत्नसाम्यसत्त्वात् । अत एव ङकारो
वा कुतो न स्यादिति चेन्न ,तमब्ग्रहणेन उक्तातिप्रसङ्गनिरासात् । अतिशयितो
ह्रन्तरोऽन्तरतमः । अतिशयितं च प्रयत्नतः सादृश्यं हकारेण घकारस्यैव,
उभयोरपि घोषनादसंवारमहाप्राणात्मकप्रयत्नचतुष्टयसाम्येन
सादृश्यातिशयसत्त्वात् । खकारस्य महाप्राणप्रयत्नसाम्येऽपि
घोषनादसंवारप्रयत्नविरहात् । गङयोः घोषनादसंवारप्रयत्नसाम्येऽपि
महाप्राणप्रयत्नविरहात् । प्रमाणतो यथा --- ॒अदसोऽसोर्दादुदोमः॑ इति ।
ह्यस्वस्य उकारो दीर्घस्य ऊकारः । नन्वेवमपि चेता स्तोतेत्यत्र इकारस्य
उकारस्य च सार्वाधातुकार्धधातुकयोरिति गुणो भवन् प्रमाणत आन्तर्यवानकारः
कुतो न स्यादित्यत आह --- यत्रेति । तेन इकारस्य एकार उकारस्य ओकारश्च गुणो
भवति, स्थानसाम्यात्, न त्वकारः, स्थानभेदात् । नच इकारेण एकारस्य, उकारेण
ओकारस्य कथं स्थानसाम्यम् । एकारस्य ओकारस्य च कण्ठस्थानाधिक्यादिति
वाच्यं, यावत्स्थानसाम्यस्य सावण्र्यप्रयोजकत्वेऽपि आन्तरतम्यपरीक्षायां
कथञ्चित्स्थानसाम्यस्यैव प्रयोजकत्वात् । अत्र सूत्रे
पूर्वसूत्रात्स्थानेग्रहणमनुवर्तते, एकदेशे स्वरितत्वप्रतिज्ञाबलात् ।
तृतीयान्तं च विपरिणम्यते । अनुवर्त्त्यमानस्चायंस्थान॑शब्दः पूर्वंसूत्रे
प्रसङ्गपरोऽप्यत्र ताल्वाद्यन्तयतमस्थानपरः, शब्दाधिकाराश्रयणात् ।
``अन्तरतम'' इत्यपि तेन संबध्यते । ततश्च ``स्थानेनाऽन्तरतम'' इति
वाक्यान्तरं संपद्यते । सति संभवे ताल्वादिस्थानत एवान्तरतमो भवतीत्यर्थः ।
ततश्चयत्रानेकविधमान्तर्यं तत्र स्थानत एवान्तर्यं बलीय॑ इति (सिद्धं) भवति
।
