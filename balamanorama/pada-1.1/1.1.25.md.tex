\textless{}\textgreater{} - डति च ।डती॑त्यविभक्तिको निर्देशः ।
प्रत्ययत्वात्तदन्तग्रहणम् । पूर्वसूत्रात्संख्ये॑त्यनुवर्तते । ``ष्णान्ता
षट्'' इत्यतःष॑डिति च । तदाह --- डत्यन्तेति । संक्येति किम् । पतिः । अथ
षट्संज्ञाकार्यं लुकं वक्ष्यन्नाह-प्रत्ययस्य लुक् । ``अदर्शनं लोप''
इत्यतोऽदर्शनमित्यनुवर्तते । प्रत्ययस्याऽदर्शनं लुक्श्लुलुप्संज्ञकं
स्यादित्यर्थः प्रतीयते । एवं सति एकस्यैव प्रत्ययाऽदर्शनस्य तिरुआओऽपि
संज्ञाः स्युः । ततश्चहन्ती॑त्यत्र शब्लुकि ``श्लौ'' इति द्वित्वं स्यात्
।जुहोती॑त्यत्र श्लौ सतिउतो वृद्धिर्लुकि हली॑ति वृद्धिः स्यात् ।
