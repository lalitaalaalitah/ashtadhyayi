\textless{}\textgreater{} - प्रथमचरम ।विभाषा
जसी॑त्यनुवर्तते,सर्वनामानी॑ति च । तदाह --- एते इति । प्रतमादय इत्यर्थः ।
उक्तसंज्ञा इति । सर्वनामसंज्ञका इत्यर्थः । तत्र नेमशब्दस्य जसि
सर्वनामसंज्ञा गणे पाठान्नित्या प्राप्ता । तद्विकल्पोऽत्र विधीयते ।
नेमशब्दव्यतिरिक्तानां प्रथमादिशब्दानां तु गणे पाठाऽभावादप्राप्तैव
सर्वनामसंज्ञा जसि विकल्पेन विधीयते । अतो नेमशब्दव्यतिरिक्तानां
प्रथमादिशब्दानां जसोऽन्यत्र न सर्वनामकार्यमित्याह-शेषं रामवदिति । ॒तय॑
शब्दो न प्रातिपदिकमित्याह --- तयप्प्रत्यय इति ।संख्याया अवयवे तयबिति
विहित॑ इति शेषः । तत इति । तस्मात् प्रत्ययत्वाद्धेतोः
प्रत्ययग्रहणपरिभाषया, तदन्ताः=तयबन्ता ग्राह्रा इत्यर्थः । द्वितये
द्वितया इति । द्ववयववावस्येत्यर्थे तयप् । यद्यप्यवयवसमुदायोऽवयवी
तयवर्थः, तस्य चैकत्वादेकवचनमेव युक्तन्तथापि यदोध्भूतावयवभेदः
समुदायस्तयबर्थः, उद्भूतत्वं च विवक्षितसङ्ख्याकत्वं,
तदाऽवयवबहुत्वाभिप्रायमवयविनोऽवयवाऽभेदाभिप्रायं वा बहु वचनमिति न दोषः ।
अत्र च तयब्ग्रहणमेव प्रमाणम् । अन्यथा तयबन्ताज्जस एवाऽभावात्कतेन । चरमे
चरमाः । अल्पे अल्पाः । अर्धे अर्धाः । कतिपये कतिपयाः-इत्यपि
प्रथमशब्दवदुदाहार्यम् । अर्धशब्दस्त्वेकदेशवाची पुंलिङ्गः । समांशवाची तु
नपुंसकलिङ्गः ।वा पुंस्यर्धोऽर्धं समेंऽशके॑ इति कोशात् । शेषं सर्ववदिति ।
नेमशब्दस्य सर्वादिगणे पाठादिति भावः । विभाषाप्रकरण इति ।विभाषा
जसी॑त्यधिकारे तीयान्तस्य ङे-ङसि-ङस्\#Hङि-इत्येतेषु ङित्सु परेषु
सर्वनामसंज्ञावचनं कर्तव्यमित्यर्थः । द्वितीयस्मै द्वितीयायेति । द्वयोः
पूरणो द्वितीयः । ``द्वेस्तीय'' इति पूरणे तीयप्रत्ययः । इत्यादीति ।
द्वितीयस्मात्, द्वितीयात् । द्वितीयस्मिन्, द्वितीये इत्यादिशब्दार्थः ।
एवं तृतीय इति । ``ङित्सूदाहार्य'' इति शेषः ।त्रेः संप्रसारणं चे॑ति पूरणे
तीयप्रत्ययः । रेफस्य संप्रसारणमृकारः ।संप्रसारणाच्चे॑ति पूर्वरूपम् ।
ननुप्रकारवचने जातीयर् इति पटुशब्दाज्जातीयरिपटुजातीय॑शब्दः, तस्यापि
तीयान्तत्वान्ङित्सु सर्वनामत्वविकल्पः स्यादित्यत आह --- अर्थवदिति
।अर्थवद्ग्रहणे नानर्थकस्ये॑ति परिभाषयाऽर्थवानेन तीयोऽत्र गृह्रते ।
जातीयरि तु समुदायस्यैवार्थवत्त्वं न तु तदेकदेशस्येति भावः । निष्क्रान्तो
जराया निर्जरः ।निरादयः क्रान्ताद्यर्थे॑ इति समासः ``गोस्त्रियोः'' इति
ह्यस्वत्वम् । निर्जरा जरा यस्मादिति बहुव्रीहिर्वा ।
