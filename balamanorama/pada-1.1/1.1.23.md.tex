\textless{}\textgreater{} - अथ षट्संज्ञाकार्यं वक्ष्यन्
षट्संज्ञोपयोगिनीं सङ्ख्यासंज्ञामाह-बहुगण । बहुश्च गणस्च वतुश्च डतिश्चेति
समाहारद्वन्द्वः । एतत्सङ्ख्यासंज्ञं स्यादित्यर्थः । फलितमाह-एते इति ।
बह्वादय इत्यर्थः । बहुगणशब्दाविह
त्रित्वादिपरार्धान्तशङ्ख्याव्यापकधर्मविशेषवाचिनौ गृह्रेते । न तु
वैपुल्यसङ्घवाचिनौ, सङ्ख्यायतेऽनयेति अन्वयर्थसंज्ञाविज्ञानात् । वतुडती
प्रत्ययौ । संज्ञाविधावपीह तदन्तग्रहणं, केवलयोः प्रयोगानर्हत्वात् ।
वतुरिहयत्तदेतेभ्यः परिमाणे वतु॑विति तद्धितप्रत्ययो गृह्रते, न तुतेन
तुल्यं क्रिया चेद्वति॑रिति वतिरपि, उकारानुबन्धात् । डतिरपिकिमः
सङ्ख्यापरिमाणे डति च ॑ इति विहिस्ततद्धित एव गृह्रते, वतुना साहचर्यात् ।
न तु भातेर्डवतुरिति विहितः कृदपि । ननुसङ्ख्यायाः क्रियाभ्यावृत्तिगणने
कृत्वसु॑जित्यादिसङ्ख्याप्रदेशेषु बह्वादीनामेव चतुर्णां ग्रहणं स्यात् । न
तु लोकप्रसिद्धसङ्ख्यावाचकानामपि,कृत्रिमाकृत्रिमयोः कृत्रिमे
कार्यसंप्रत्ययः॑ इति न्यायात् । ततश्च ``पञ्चकृत्वः'' इत्यादि न स्यादिति
चेन्न,सङ्ख्याया अतिशदन्तायाः कनि॑त्यत्र तिशदन्तपर्युदासबलेन
सङ्ख्याप्रदेशेषु कृत्रिमाऽकृत्रिमन्यायाऽप्रवृत्तिज्ञापनात् । नहि
विंशतितिंरशदादिशब्दानां कृत्रिमा सङ्ख्यासंज्ञाऽस्ति । नचैवं सति
बहुगणग्रहणवैयथ्र्यं शङ्क्यं, तयोर्नियतविषयपरिच्छेदकत्वाऽभावेन
लोकसिद्धसङ्ख्यात्वाऽभावात् । अत एव भाष्येएतत्सूत्रमतिदेशार्थं
यदयमसङ्ख्यां संख्येत्याह॑ इत्युक्तं सङ्गच्छते इत्यस्तां तावत् ।
