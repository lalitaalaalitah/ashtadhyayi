\textless{}\textgreater{} - तत्र विशेषं दर्शयितुमाह --- विभाषा दिक्समासे
।सर्वादीनी॑त्यतःसर्वनाम॑ग्रहणमनुवर्तते । तदाह --- अत्रेति । दिक्समासे
इत्यर्थः । ``न बहुव्रीहौ'' इत्यलौकिकविग्रहवाक्ये नित्यनिषेधे प्राप्ते
विकल्पार्थमिति केचित् । गौणत्वादप्राप्ते विभाषेयमित्यन्ये ।
सर्वनामत्वपक्षे उदाहरति --- उत्तरपूर्वस्यै इति । स्याड्ढ्रस्वौ ।
उत्तरपूर्वायै इति । सर्वनामत्वाऽभावपक्षे याट् ।
उत्तरपूर्वायाः-उत्तरपूर्वस्याः । उत्तरपूर्वासाम् --- उत्त्रपूर्वाणाम् ।
उत्तरपूर्वस्याम् --- उत्तरपूर्वायाम् ।सर्वनाम्नो वृत्तिमात्रे
पुंवद्भावः॑ इति मात्रग्रहणात्संप्रत्यसर्वनामत्वेऽपि पूर्वपदस्य
पुंवत्त्वम् । ननूत्तरा दिगिति गत्वा मोहवशात्पूर्वा दिक् यस्याः सा
उत्तरपूर्वा । ``अनेकमन्यपदार्थ'' इति बहुव्रीहिः । अत्रापि
दिक्शब्दघटितसमासत्वात्सर्वनामताविकल्पे उत्तरपूर्वस्यै --- उत्तरपूर्वायै
इति रूपद्वयं स्यात् । स्याडागमस्तु नेष्यते । तत्राह --- दिङ्नामानीति
।दिङ्नामान्यन्तराले॑ इति बहुव्रीहिः प्रतिपदोक्तो दिक्समासः,
दिक्शब्दमुच्चार्यं विहितत्वात् । नतु ``अनेकमन्यपदार्थे'' इति
बहुव्रीहिरपि । ततश्च लक्षणप्रतिपदोक्तपरिभाषया न तस्येह ग्रहणमित्यर्थः ।
योत्तरेति । उत्तराशब्दस्य पूर्वाशब्दस्य च सामानाधिकरण्यं द्योतयितुं
यत्तच्छब्दौ । सामानाधिकरण्याऽभावे ``अनेकमन्यपदार्थे'' इति
बुहुव्रीहेरसंभवात्,बहुव्रीहिः समानाधिकरणानां वक्तव्य)॑ इति वचनात् ।
उन्मुग्धाया इति । तेन पूर्वोत्तरयोर्विरोधात्कथं सामानादिकरण्यमिति शङ्का
निरस्ता । ननु ``विभाषा दिक्समासे'' इत्येवास्तु, बहुव्रीहिग्रहणं न
कर्तव्यं, प्रतिपदोक्तत्वेनदिङ्नामानी॑ति बहुव्रीहेरेव ग्रहणसिद्धेरित्यत
आह --- बहुव्रीहिग्रहणं स्पष्टार्थंमिति । बाह्रायै इत्यर्थ इति
।अन्तरमवकाशावधिपरिधानान्तर्धिभेदतादर्थ्ये ।
छिद्रात्मीयविनाबहिरवसरमध्येऽन्तरात्मनि चे॑ति कोशात् । अर्थान्तरपरत्वेन
तु सर्वनामत्वाऽभावान्न स्यादिति भावः । अपुरीत्युक्तेरिति ।अन्तरं
बहिर्योगेति गणसूत्रे॑ इति शेषः ।
