\textless{}\textgreater{} - विआवागम्, विआवाहौ इति सुटि रूपाणि
सुगमत्वादुपेक्ष्य शसादावचि संप्रसारणकार्यं वक्ष्यन् संप्रसरणसंज्ञा
दर्शयति --- इग्यणः यणः स्थाने इति । व्याख्यानात्स्थानार्थलाभः ।षष्ठी
स्थानेयोगे॑ति तु नेह भवति, अनुवादे परिभाषामनुपस्थितेः । षष्ठीश्रुतौ
सर्वत्र व्याख्यानादेव स्थानार्थलाभसंभवात् ``षष्ठीस्थानेयोगा''
इत्येत॒न्निर्दिश्यमानस्यादेशा भवन्ती॑त्येतदर्थमिति भाष्ये
सिद्धान्तितत्वाच्च । संप्रसारणसंज्ञ इति । ततश्च ``वसोः
संप्रसारणं''वचिस्वपियजादीना॑मित्यादौ संप्रसारणश्रुतौ यण्स्थानिक
इगुपस्थितो भवति । तत्रान्तरतम्याद्यस्य इकारः, वकारस्य उकारः, रेफस्य
ऋकारः लस्य लृकार इति ज्ञेयम् ।
