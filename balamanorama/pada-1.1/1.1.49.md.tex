\textless{}\textgreater{} - षष्ठी स्थानेयोगा । स्थानं प्रसङ्ग इति
वक्ष्यति । तस्मिन् वाचकतया योगो यस्याः सा स्थानेयोगा । निपातनात्सप्तम्या
अलुक् । स्थानेन योगो यस्या इति वा विग्रहः । निपातनादेत्वम् ।इको
यणची॑त्यादौ षष्ठी स्थानरूपसम्बन्धार्थिकेत्यर्थः । लोके तावदेकशतं
षष्टर्था आर्था यौना मौखाः रुआऔवाश्च । शब्दस्य शब्देन त्रय एव संबन्धाः
--- आनन्तर्यं सामीप्यं प्रसङ्गश्चेति । तत्रान्यतमार्थनिर्धारणार्थमिदं
सूत्रमिति भाष्यम् । ततश्च॒इको यणची॑त्यादौ ``इक'' इति षष्टआ स्थानमुच्यते
। तस्मिन् प्रकृत्यर्थ इक् निरूपकतयाऽन्वेति । अचि परत इकः प्रसङ्गे यण्
स्यादिति । विवरणवाक्ये त्वस्मिन् ``इक'' इति षष्ठी निरूपकतायामिति न
पुनरुक्तिः शङ्काः । यता --- ॒देवदत्तस्यावयवः पाणि॑रिति । ``ऊदुपधाया
गोहः'' इत्यत्र तु ``गोह'' इति षष्ठी न स्थानार्थिका,
उपपधापदसमभिव्याहारेणाऽवयवषष्ठीत्वनिर्धारणात्, परिभाषाणां चाऽनियमे
नियमनार्थमेव प्रवृत्तेः । तदेतदाह-अनिर्धारितेत्यादिना । अनिर्धारितः
सम्बन्धविशेषो यस्या इति विग्रहः । तदेवमुदाह्मतप्रकृतभाष्यरीत्या॒इको
यणची॑त्यादौ षष्ठी स्थानरूपसंबन्धविशेषार्थिकेति स्थितम् ।ष
मतुप्सूत्रभाष्ये त्वनन्तरादयो न षष्ठर्था इति स्थितम् । एवं सति स्थाने
इति सप्तम्यन्तपदेन योगो यस्या इति विग्रहे स्थाने इति
सप्तम्यन्तस्यानुकरणम् । षष्ठीश्रुतौ ``स्थाने'' इति सप्तम्यन्तं
पदमुपतिष्ठत इति फलति । स्थानेन स्थानपदार्थेन योगो यस्या इति
तृतीयान्तविग्रहे त्वध्याह्मतस्थानपदार्थनिरूपितसंबन्धार्थिकेत्यर्थः
।अस्तेर्भूर्भवतीति सन्देहः स्थाने अनन्तरे समीपे इति॑
इत्यादिप्रकृतसूत्रभाष्यस्य तु अस्तेरनन्तरे
इत्यध्याह्मतानन्तरादिपदार्थनिरूपितसम्बन्धे षष्ठीत्येवार्थः ।
अनन्तरादीनां षष्ठर्थत्वं तु नास्त्येवेति प्रौढमनोरमायां हलन्त्यमिति
सूत्रे स्थितम् । तद्व्याख्याने च शब्दरत्ने शब्देन्दुशेखरे च बहुधा
प्रपञ्चितम् । अनिर्धारितेति किम् । ऊदुपधाया॑ इत्यत्र ``गोह'' इति
षष्ठ्याः स्थानार्थकत्वं मा भूत् । सति तु तत्रापि स्थानार्थकत्वे,
गोहोऽन्त्यस्य उपधामात्रस्य च ऊत् स्यात् । ननु स्थानशब्द आधारवाची लोके
प्रसिद्धः । यथा शिवस्थानं कैलासः, विष्णुस्थानं वैकुण्ठमित्यादौ । एवं च
इको यणचीत्यादौ षष्ठ्याः स्थानार्थकत्वे प्रकृत्यर्थस्य तत्र
निरूपकत्वेऽभेदेन वा अन्वये सति इकोऽधिकरणे यण् स्यादिति इगधिकरणको यण्
स्यादिति वाऽर्थः स्यात् । तत इको निवृत्तिर्न स्यादित्यत आह --- स्थानं च
प्रसङ्ग इति । क्वचिदाभिचारेष्टौदर्भाणां स्थाने शरैःप्रस्तरितव्य॑मित्यत्र
स्थानशब्दस्य प्रसङ्गे दर्शनादिति भावः । एवं च तत्र यथा शरैर्दर्भा
निवत्र्यन्ते, तद्वदिको यणचीत्यादावपि यणादिभिरिगादयो निवत्र्यन्ते । तत्र
च यः प्रसक्तो निवर्तते स स्थानीति व्यवह्यियते, यो निवर्तयति स आदेश इति ।
