\textless{}\textgreater{} - नपुंसकवशात्शब्दरूपाणी॑ति विशेष्यमद्याहार्यं,
तदाह --- सर्वादीनीति । ननु बहुव्रीहेरन्यपदार्थप्रधानत्वात्सर्वंशब्दस्य च
समासवर्तिपदार्थत्वादन्यपदार्थत्वाबावाद्विआआदिशब्दानामेव सर्वादिशब्देन
बहुव्रीहिणावगमात्सर्वनामसंज्ञा स्यान्न तु सर्वशब्दस्यापीति चेत्, उच्यते
--- सर्व आदिर्यस्य समुदायस्येति विग्रहः । सर्वशब्दघटितः समुदायः
समासार्थः । समुदाये च प्रवर्तमाना सर्वनामसंज्ञा क्वचिदप्यप्रयुज्यमाने
तस्मिन् वैयथ्र्यादानर्थक्यात्तदङ्गेष्विति न्यायेनावयवेष्ववतरन्ती
अविशेषात्सर्वशब्देऽपि भवति । एवंचात्र सर्वशब्दस्य स्वरूपेण
वर्तिपदार्थता, समुदायरूपेण त्वन्यपदार्थप्रवेशः । नच
समुदायस्यान्यपदार्थत्वे सर्वादीनीति बहुवचनानुपपत्तिः शङ्क्याः,
सर्वशब्दघटितस्य विवक्षितावयवसङ्ख्यस्य समूहस्यान्यपदार्थत्वात्
।अद्भूतावयवभेदः समुदायः समासार्थ॑ इति कैयटोक्तेरप्ययमेवार्थः । अतो न
बहुवचनस्यानुपपत्तिः तदेवं व्याख्यानेहलि सर्वेषा॑मित्यादिनिर्देशः
प्रमाणम् । सर्वशब्दस्य सर्वनामत्वाऽभावे तु सर्वेषामित्यादौ
सर्वनामकार्याणि सुडादीनि न स्युः । तथाच सर्वादीनीति तद्गुणसंविज्ञानो
बहुव्रीहिः । तस्य=अन्यपदार्थस्य, गुण\#आ\#ः=विशेषणानि वर्तिपदार्थरूपाणि,
तेषां संविज्ञानं=क्रियान्वयितया विज्ञानं यत्र स तद्गुणसंविज्ञान इति
व्यत्पत्तिः । यत्र संयोगसमवायान्यतरसंबन्धेनान्यपदार्थे
वर्तिपदार्थान्वयस्तत्र प्रायेण तद्गुणसंविज्ञानो बहुव्रीहिः ।
यथा-॒द्विवासा देवदत्तो भुङ्क्ते॑,लम्बकर्णं भोजये॑त्यादौ । तत्पर हि
वाससोः कर्णयोश्च भुजिक्रियान्वयाऽभावेऽपि संनिहितत्वमात्रेण
तद्गुणसंविज्ञानत्वम् । प्रकृते च समुदाये ।ञन्यपदार्थे सर्वशब्दस्य
समवायान्तर्गतारोपितावयवावयविभावसंबन्धसत्त्वात्तद्गुणसंविज्ञानत्वम् ।
स्वस्वामिभावादिसम्बन्धेनान्यपदार्थे वर्तिपदार्थान्वये
त्वतद्गुणसंविज्ञानो बहुव्रीहिः । यथा --- ॒चित्रगुमानये॑त्यादावित्यलम्
।ननु सर्व विओत्येवं सर्वादिशब्दानां केवलानामेव सर्वादिगणे
पाठात्परमसर्वादिशब्दानां कथं सर्वनामतेत्यत आह --- तदन्तस्यापीति ।
द्वन्द्वे चेतीति ।द्वन्द्वे चे॑त्यनेन सर्वादिशब्दान्तद्वन्द्वस्य
सर्वनामसंज्ञा प्रतिषिध्यते --- वर्णाश्रमेतराणामित्यादौ । यदि केवलानामेव
सर्वादिगणपठितानां केवलानामेव सर्वादिशब्दानामस्तु सर्वनामता, मास्तु
तदन्तानामपि, ``सर्वनाम्नः स्मै''
इत्यादिसर्वनामकार्याणामङ्गाधिकारस्थत्वेनपदाङ्गाधिकारे तस्य च तदन्तस्य
चे॑ति परिभाषया ``परमसर्वस्मै'' इत्यादिषु । सिद्धेरित्यत आह --- तेनेति ।
तदन्तस्यापि संज्ञाबलेनेत्यर्थः । सिध्यतीत्यर्थः ।
चकारात्पञ्चम्यास्तसिलिति तसिल् च । नचावयवगतसर्वनामत्वेन तत्सिद्धिरिति
वाच्यं , कुत्सित \#इति सूत्रस्थभाष्यरीत्या सह्ख्याकारकाभ्यां
पूर्णार्थस्येतरान्वयेन सुबन्तादेव तद्धितोत्पत्त्यवगमेन
सर्वनामप्रकृतिकसुबन्तार्थगतकुत्सादिविवक्षायां सर्वनामावयवटेः
प्रागकजित्यर्थपर्यवसानात्तदन्तसंज्ञाऽभावे तदसिद्धेरिति भावः ।
