\textless{}\textgreater{} - संबुद्धौ शाकल्यस्येति । सम्बुद्धाविति
निमित्तसप्तमी, ओदित्यनुवृत्तेन सहाऽन्वेति ।प्रगृह्र॑मित्यनुवर्तते, स च
पुँल्लिङ्गतया विपरिणम्यते । ऋषिः=वेदः ।तदुक्तमृषिणे॑त्यादौ तथा दर्शनात्
। ऋषौ भवः --- आर्षः, न आर्षः-अनार्षः । अवैदिके इतिशब्दे परत इत्यर्थः ।
शाकल्यग्रहणाद्विकल्पस्तदाह --- संबुद्धिनिमित्तक इति । विष्णो इतीति ।
अत्र ओकारो ``ह्यस्वस्य गुण'' इति सम्बुद्धिनिमित्तकः । अत्र ओदन्तत्वेऽपि
निपातत्वाऽभावादप्राप्ते विभाषेयम् ।विष्ण॑वितीति प्रगृह्रत्वाऽभावे रूपम्
।
