\textless{}\textgreater{} - ग्रहणकसूत्रेऽण्सवर्णानां ग्राहक इति स्थितम्
। एवं सति अत् इत् उत् इत्यादितपराणामप्यणां स्वस्वसर्वसवर्णग्राहकत्वे
प्राप्ते इदमारभ्यते --- तपरस्तत्कालस्य । ``तपर'' इत्यावर्तते ।
प्रथमस्तावत्तपरशब्दः-तः परो यस्मादिति बहुव्रीहिः । द्वितीयस्तु तात् पर
इति पञ्चमीसमासः । ग्रहणकसूत्रादणित्यनुवर्तते । तस्य तपरत्वेन
उच्चार्यमाणवर्णस्य काल इव कालो यस्येति बहुव्रीहिः । ऊकालः उष्ट्रमुख
इत्यादिवत् समासः । एवं च ``अत्॒इत्'' इत्याद्यात्मकोऽण् तपरत्वेन
उच्चार्यमाणः स्वीयकालसदृशकालस्य संज्ञा स्यादित्यर्थः । तत्र अत् इत् उत्
ऋत् इत्येतेषां तपराणां ह्यस्वाकारादीनामणां
तत्तत्कालास्तत्तद्ध्रस्वप्रपञ्चाः । एत्, ऐत्, ओत्ौत् इत्येतेषां तु
तपराणामेकारादीनां तत्तत्कालास्तत्तद्दीर्घप्रपञ्चाः । तत्र
ह्यस्वाकारादीनां तपराणां तत्तद्ध्रस्वप्रपञ्चावाचकत्वस्य एकारादीनां च
दीर्घाणां तपराणां स्वस्वदीर्घप्रपञ्चवाचकत्वस्य लोकसिद्धत्वात्सिद्धे
सत्यारम्भो नियमार्थः॑ इति न्यायेन सूत्रमिदं नियमार्थं सम्पद्यते --- ॒अण्
तपरश्चेत्तत्कालस्यैव सवर्णस्य ग्राहको न त्वतत्कालस्ये॑ति । एवं च
अतत्कालनिवृत्त्यात्मकपरिसंख्यार्थंमिदं सूत्रम् । वैयाकरणस्तु
परिसंख्याविधिमेव नियमविधिरिति व्यवहरन्ति । तदिदं सर्वमभिप्रेत्य
व्याचष्टे --- तः परोयस्मादित्यादिना । नियमविधानस्य फलमाह ---
तेनेत्यादिना । तेन=नियमविधानेन । आदिना लृदित्यादिसंग्रहः । अत्, इत्,
उत्, लृत्, एत, ऐत्, ओत्, औत् --- इत्येते अष्टौ तपरा अणः
स्वस्वसमानकालानां षण्णां षण्णामेव संज्ञाः , न त्वतत्कालानामित्यर्थः ।
ऋदिति द्वादशानामिति । ऋलृवर्णयोरिति सावण्र्यविधानादिति भावः ।
नन्वेवम्लृदित्यपि द्वादशानां ग्रहणं स्यात्, तथा चपुषादिद्युताद्य्लृदितः॑
इत्यादिविधय ऋदित्सवपि प्रवर्तेरन्निति चेन्न, ऋदित्यनेन लृकारप्रपञ्चस्य
ग्रहणेऽपि क्वचित् लृदिद्ग्रहणबलेन लृ इत्यनेन ऋकारप्रपञ्चस्य
ग्रहणाऽभावात् । अन्यथाऋतो ङी॑त्यादौ क्वचित् ऋद्ग्रहणस्य
पुषादिद्युताद्य्लृदित इत्यादौ क्वचित् लृदिद्ग्रहणस्य च वैयथ्र्यापत्तेः ।
प्रथमातिक्रमणे कारणाऽभावेन सर्वत्र ऋदिद्ग्रहणस्यैव कर्तु शक्यत्वात् ।
इदमेवाभिप्रेत्य ग्रहणकसूत्रे ऋदिति द्वादशानामित्येवोक्तम्, नतु लृदंपीति
। अत्र च तः परो यस्मादिति बहुव्रीहेः-अत्, इत्, उत् इत्याद्युदाहरणम् ।
तात्पर इति पञ्चमीसमासस्य तु वृद्धिरादैजित्यैकार उदाहरणम् । आत् ईत् ऊत्
इत्यादि तु न तपरसूत्रस्योदाहरणम् , आकारादिषु हि
तपरसूत्रमपूर्वविधानार्थम्, उत नियमार्थम् , नाद्यः, तपरसूत्रे
अण्ग्रहणानुवृत्त्यभावेऽपि जातिपक्षे आकारादिभिर्दीर्घैः
स्वस्वसमानकालिकप्रपञ्चस्य वाच्यताया लोकत एव सिद्धत्वेन तेषु
तपरसूत्रप्रवृत्तेव्र्यर्थत्वात् । न द्वितीयः । उक्तरीत्या ग्रहणकसूत्रस्य
वार्णसमाम्नायिकवर्णमात्रविषयतया आकारादिषु तस्य प्रवृत्त्यसम्भवेन
तपरसूत्रस्य तेष्वतत्कालव्यावृत्तिफलकतन्नियमनार्थत्वानुपपत्तेः, सिद्धे
सत्यारम्भस्यैव नियमार्थत्वात् । एवं च आत्, ईदित्यादि
तपरकरणमसन्देहार्थमेवेत्यास्तां तावत् । तदेवं वृत्तः प्रत्याहारप्रपञ्चो
ग्रहणकशास्त्रप्रपञ्चश्च ।
