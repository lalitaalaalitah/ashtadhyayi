\textless{}\textgreater{} - अथ ओदन्ताः । प्रकृष्टा द्यौर्यस्येति
बहुव्रीहौ प्रद्योशब्दस्य ``ह्यस्वो नपुंसके'' इति ह्यस्वः प्राप्नुवन्नेचो
ह्यस्वाऽभावात्तेषां द्विस्थानत्वेन अ, इ, उ, ऋ, लृ, इत्येतेषां
ह्यस्वानामन्तरतमत्वाऽभावादन्तरस्थानसाम्याश्रयेण अवर्णादिषु यस्य
कस्यचिदनियमेन पर्यायेण वा प्राप्ताविदमारभ्यते --- एच इक् । आदिश्यते
इत्यादेशः । कर्मणि घञ् । तस्य ह्यस्वपदेन सह कर्मधारयः विशेष्यस्यार्षः
पूर्वनिपातः । आदेश इति निर्धारणसप्तमी । सौत्रमेकवचनम् । तदाह ---
आदिश्यमानेष्वित्यादिना । ``मध्ये'' इत्यपपाठः, तद्योगे षष्ठआ एवौचित्यात्
। इगेवेति । तेन अकारव्यावृत्तिः फलतीति भावः । यद्यपीकश्चत्वारः,
एचोऽप्येवं, तथापि स्थान्यादेशानां यथासङ्ख्यं न भवति । न ह्रयमपूर्वविधिः,
किन्तु नियमविधिः । यताप्राप्तमेव नियम्यते । एचां हि पूर्वभागोऽवर्णसदृशः
। उत्तरभागस्तु इवर्णोवर्णसदृशः । तत्पर पूर्वभागसादृश्यमवर्णस्यास्ति ।
तस्य च इग्ग्रहणेन निवृत्तौ इवर्णसादृश्यमात्रमादाय एकारस्य ऐकारस्य च
इवर्णः, उवर्णसादृस्यादोकारस्य औकारस्य च उवर्ण इति व्यवस्था न्यायप्राप्ता
। यताप्राप्तमेव च नियम्यत इति न यतासङ्ख्यम् । ततश्च प्रद्योशब्दे ओकारस्य
उकारो ह्यस्व इत्यभिप्रेत्योदाहरति --- प्रद्यु इत्यादि । ननुं
पुंनपुंसकयोः प्रकृष्टस्वर्गवत्त्वमेकमेव प्रवृत्तिनिमित्तमिति टादौ
पुंवत्त्वविकल्पः कुतो नेत्यत आह --- इह न पुंवदिति । कुत इत्यत आह ---
यदिगन्तमिति । प्रद्योशब्द ओदन्तः पुंसि । प्रद्युशब्दस्तु उदन्तो नपुंसके
। तथाच पुंसि प्रद्योशब्दस्य भाषितपुंस्कत्वेऽपि नपुंसके प्रद्युशब्दस्य
तदपेक्षया भिन्नत्वेन भाषितपुंस्कत्वाऽभावान्न पुंवत्वमित्यर्थः । केचित्तु
पुंसि यः प्रद्योशब्द ओदन्तः स एवेदानीं नपुंसकः, तस्य ह्यस्वान्तत्वेऽपि
एकदेशविकृतस्याऽनन्यत्वात्, अतः पुंवत्वविकल्पोऽस्त्येवेत्याहुः ।
एवमग्रेऽपीति । प्ररि, सूवु इत्यादावपीत्यर्थः । इत्योदन्ताः । अथ ऐदन्ताः
। एकारान्तस्योदाहरणं तु स्मृत इः कामो येन स स्मृतेः । सु=शोभनः
स्मृतेर्यस्य तत् ``सुस्मृति'' इत्यादि बोध्यम् । प्ररीति । प्रकृष्टो
रा=धनं यस्य इति बहुव्रीहौ प्ररैशब्दः । तस्य नपुंसकह्यस्वत्वेन इकारः ।
सुटि वारिवत् । सोर्लुप्तत्वात्रायो हली॑त्यात्वं न, टादावचि
पुंवत्त्वविकल्पः प्रद्युशब्दवत् । भ्यामादौ हलि विशेषमाह --- रायो
हलीत्यात्वमिति । ननु रैशब्दस्य ऐदन्तस्य बिहितमात्वं कथमिदन्तस्येत्यत आह
--- एकदेशविकृकतस्यानन्यत्वादिति । आमि विशेषमाह --- नुमचिरेति । नुटिरायो
हली॑त्यात्वे प्रराणामित्यन्वयः । ननु प्ररि-आमिति स्थिते नुटं बाधित्वा
परत्वान्नुमि तस्याङ्गभक्तत्वाद्धलादिविभक्त्यभावात्कथमात्वमित्यत आह ---
नुमचिरेति । पूर्वविप्रतिषेधान्नुमं बाधित्वा नुठआत्वं निर्बाधमिति भावः ।
संनिपातेति । ह्यस्वान्तत्वमुपजीव्य प्रवृत्तस्य नुटस्तद्विघातकमात्वं
प्रति निमित्तत्वाऽसंभवादिति भावः । ननु तर्हि ह्यस्वान्तत्वमुपजीव्य
प्रवृत्तस्य नुटस्तद्विघातकंनामी॑ति दीर्घं प्रति कथं निमित्तत्वमित्यत आह
--- नामीति दीर्घस्त्विति । इत्युक्तमिति । ``रामशब्दाधिकारे'' इति शेषः ।
इत्यैदन्ताः । अथ औदन्ताः । सु=शोभना नौर्यस्येति विग्रहे बहुव्रीहौ
``ह्यस्वो नपुंसके'' इति ह्यस्व उकार इति मत्वाह --- सुनु इति ।
इत्यौदन्ताः॥*****इति बालमनोरमायामजन्ता
नपुंसकलिङ्गाः*****अथाऽलुक्समासप्रकरणम् । --- --- --- --- --- --- --- ---
---
