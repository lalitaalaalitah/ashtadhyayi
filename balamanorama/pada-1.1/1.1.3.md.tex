\textless{}\textgreater{} - इको गुणवृद्धि । ``इक'' इति षष्टन्तशब्दः
स्वरूपपरो नपुंसकलिङ्गः प्रथमैकवचनान्तः । सोर्लुका
लुप्तत्वादत्वसन्तस्येति दीर्घो न ओ ।इकस्शब्द इत्यर्थः । ``उपतिष्ठते''
इति शेषः । ``वृद्धिरादैच्'' ``अदेङ्गुणः'' इत्यतो वृद्धिरिति गुण इति
चानुवर्तते । इतिशब्दोऽध्याहार्यः ।यत्र विधीयते तत्रे॑त्यप्यध्याहार्यम् ।
गुणो वृद्धिरित्युच्चार्य यत्र गुणवृद्धि विधीयेते तत्र इक इति षष्ठन्तं
पदमुपतिष्ठत इति योजना । तदाह --- गुणवृद्धिशब्दाभ्यामित्यादिना । उपतिष्ठत
इति । सङ्गतं भवतीत्यर्थः ।उपाद्देवपूचासङ्गतिकरणे॑त्यात्मनेपदम् । सोऽयं
पदोपस्थितिपक्षो भाष्यादौ सिद्धान्तितः ।सार्वधातुकार्धधातुकयोः॑,
मिदेर्गुणः॑ इत्याद्युदाहरणम् । इक इत्यस्यान्वयप्रकारस्तु तत्र तत्र
स्पष्टीभवष्यति । ``यत्र विधीयेते''
इत्युक्त्यावृद्धिर्यस्याचा॑मित्याद्यनुवादे इक इति नोपतिष्ठते । अनुवादे
परिभाषामनुपस्थितेः । ``त्यदादीनामः'' इत्यादावपि नेदमुपतिष्ठते, तत्र
गुणवृद्धिशब्दयोरश्रवणात् ।
