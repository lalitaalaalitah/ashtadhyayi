\textless{}\textgreater{} - हलोऽनन्तराः संयोगः ।अन्तर॑शब्दोऽत्र व्यवधाने
वर्तते ।अन्तरमवकाशावधिपरिधानान्तर्धिभेदतादर्थ्ये॑ इत्यमरः । व्यवधानं च
विजातीयेनैव । अविद्यमानमन्तरं व्यवधानं येषामिति विग्रहः
।नञोऽल्त्यर्थाना॑मिति विद्यमानपदस्य लोपः । तदाह --- अज्भिरित्यादिना ।
तत्र हलौ च हलश्च हल इत्येकशेषः । तेन द्वयोरपि संयोगसंज्ञा लभ्यते । ततश्च
शिक्षेत्यत्र ``गुरोश्च'' हलः इत्यप्रत्ययः सिध्यति । अत्र च समुदायस्यैव
संयोगसंज्ञा, महासंज्ञाकरणात्, व्याख्यानाच । नतु प्रत्येकम् । तथा सति
``सुदृषत्प्रासाद'' इत्यत्र पकारसन्निधौ तकारस्य संयोगत्वापत्तौ
संयोगान्तलोपापत्तेः । यत्र तु बहवो हलः श्लिष्टास्तत्रापि द्वयोद्र्वयोः
संयोग९संज्ञा न तु बहूनामेवेति शब्देन्दुशेखरे स्पष्टम् ।
