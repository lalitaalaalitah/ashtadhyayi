\textless{}\textgreater{} - अलोऽन्त्यस्य ।अ॑लिति प्रत्याहारो वर्णपर्यायः
। ``अल'' इति षष्ठन्तम् । ``षष्ठी स्थानेयोगा'' इत्यतःषष्ठी स्थाने
इत्यनुवर्तते । तच्च षष्ठीति प्रथमान्तं तृतीयान्ततया विपरिणम्यते
।निर्दिष्टस्ये॑ति शेषः । ``स्थाने'' इत्यनन्तरं ``विधीयमान'' इति शेषः ।
स्थाने विधीयमान आदेशः षष्ठीनिर्दिष्टस्य योऽन्त्योऽल् तस्य स्यादित्यर्थः
। तदाह --- षष्ठीत्यादिना ।त्यदादीनामः॑-यः सः । आदेश इति किम् ,
``आर्धधातुकस्येट्'' तृच ऋकारात् पूर्वो मा भूत् । अल इति किम् ।
पदस्येत्यधिकृत्य विधीयमानं वसुरुआंस्विति दत्वं
परमानडुभ्द्यामित्यत्राऽन्त्यस्य कृत्स्नस्य पदस्य मा भूत् ।
