\textless{}\textgreater{} - एवं प्राप्ते प्रतिषेधति --- नाज्झलौ । आसहितः
अच्-आच् । शाकपार्थिवादित्वात्सहितशब्दस्य लोपः । आ च हस्व-आज्झलौ ।
तुस्यास्यसूत्रात्सवर्णमित्यनुवर्तते । तच्च पुँल्लिङ्गाद्विवचनान्ततया
विपरिणम्यते । तदाह --- अकारसहितोऽजित्यादिना । ननु किमर्थोऽयं प्रतिषेध
इत्यत आह --- तेनेत्यादि यणादिकं नेत्यन्तम् । तेन=प्रतिषेधेन । आदिना
सवर्णदीर्घसङ्ग्रहः । दधीति इकारस्य हकारे षकारे सकारे च परे ``इको यणचि''
इति यणादेशः, शीतलमित्यत्र शकारे परे सवर्णदीर्घश्च न भवतीत्यर्थः ।
नन्वस्त्वकारहकारयोरिकारशकारयोरृकारषकारयोर्लृकारसकारयोश्च सावण्र्यं,
तथापिदधि षष्ठ॑मित्यादौ यणादिकं न प्रसक्तम्, अचपरकत्वाऽभावादित्यत आह ---
अन्यथेत्यादिना । अन्यथा=तेषां सावण्र्याभ्युपगमे, दीर्घादीनामिव
हकारादीनामप्यच्त्वं स्यादित्यन्वयः । ननु वर्णसमाम्नाये
हकारादीनामकारचकारमध्यगत्वाऽभावात्कथमच्त्वमित्यत आह-ग्रहणकशास्त्रबलादिति
। गृह्णन्त्यकारादयः स्वसवर्णान् येन तद्ग्रहणम् । करणे ल्युट् । स्वार्थे
कः । अणुदित्सूत्रादित्यर्थः । यद्यप्यच्छब्द०वाच्यत्वं
वार्णसमाम्नायिकानामेव वर्णानान्तथापीको
यणचीत्यादावच्छब्देनाऽकारादिषूपस्थितेषु तैरणुदित्सूत्रबलेन
स्वस्वसवर्णानामाकारादीनामुपस्थितिरस्ति ।
ततश्चाऽत्राच्पदवाच्याकारादिवाच्यत्वादाकारादीनामिव हकारादीनामपि लक्षणया
अच्छब्देन ग्रहणं स्यादित्यर्थः । न च इको यणचीत्यादौ
शक्यार्थमादायैवोपपत्तेर्न लक्षणासंभवः । अणुदित्सूत्रे तु अस्य
च्वावित्यादौ सावकाशमिति वाच्यं, स्वादिभ्य इत्यादिनिर्देशबलेन
प्रत्याहाराणां स्ववाच्यवाच्येषु लक्षणाऽवश्यंभावात् । तथा चअच्त्वं
स्या॑दित्यस्य अचपदबोध्यत्वं स्यादित्यर्थः । किं
तद्ग्रहणकशास्त्रमित्याकाङ्क्षायां तदुपपादनं प्रतिजानीते-तथा हीति ।
