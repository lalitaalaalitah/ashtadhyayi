\textless{}\textgreater{} - ङिच्च । ङकार इद्यस्य स-ङित्
।अलोऽन्त्यस्ये॑त्यनुवर्तते । तदाह --- अयमपीति । ङिदमपीत्यर्थः । अवङ्
तावङ् अनङित्यादिरादेश उदाहरणम् ।नन्वलोऽन्त्यस्येति पूर्वसूत्रेणैव सिद्धे
किमर्थमिदमित्यत आह-सर्वस्येति ।अनेकाल् शित् सर्वस्ये॑ति वक्ष्यमाणस्य
सर्वादेशत्वविधेरयं विधिरपवादः । अपोद्यते बाध्यते अनेनेति अपवादः ।
बाहुलकः करणे घञ् । येन नाप्राप्ते यो विधिरारभ्यते स तस्यापवाद
इत्यपवादलक्षणम् । अप्राप्ते इति भावे क्तः ।येने॑ति कर्तरि तृतीया । द्वौ
नञावावश्यकत्वं बोधयतः । यत्कर्तृकावश्यकप्राप्तौ सत्यां यो विधिरारभ्यते स
आरभ्यमाणविधिस्तस्याऽवश्यप्राप्तस्य अपवादो बाधक इति तदर्थः । अयं च
न्यायसिद्धः । अवङादयो हि ङित आदेशाः सर्वेऽनेकाल एव । तेषु
चानेकाल्विशेषेषु विधीयमानेन ङितामन्त्यादेशत्वेन स्वविषये अवश्यं
प्राप्तमनेकाल्सामान्येन विहितं सर्वादेशत्वं बाध्यते, विशेषविहितत्वात्,
निरवकाशत्वाच्च । विशेषशास्त्र हि विशेषेषु झटिति प्रवत्र्तते, विशेषाणां
स्वशब्देनोपात्तत्वात् । सामान्यशास्त्रं तु सामान्यमुखेन विशेषेषु
प्रवर्तत इति तस्य तेषु मन्दप्रवृत्तिः । अतो विशेषशास्त्रं प्रबलम् ।
उक्तं च भट्टवार्तिकेअवश्यमेव सामान्यं विशेषं प्रति गच्छति । गतमात्रं च
तत्तेन विशेषे स्थाप्यते ध्रुवम्॥॑ इति । किं च यदि ङिच्चेति
शास्त्रमनेक\#आल्विशेषेषु ङित्सु न प्रवर्तेत, तर्हि तदनर्थकमेव स्यात्
।अनेकाल्शित्सर्वस्ये॑त्यस्य तु ङित्सु अप्रवृत्तावपि नानर्थक्यम्,
तस्थस्थमिपां तान्तंताऽमः॑अस्तेर्भू॑रित्यादिष्वनेकाल्षु अङित्सु तस्य
सावकाशत्वात् । अतो विशेषशास्त्रं प्रबलमिति ।
