\textless{}\textgreater{} - अथअणुदित्सवर्णस्य चाप्रत्ययः॑ इति अ इ उ
इत्यादिसंज्ञां वक्ष्यन्सवर्णसंज्ञामाह --- तुल्यास्य । आस्यं = मुखम्,
तत्साम्यस्य सर्ववर्णेष्वविशिष्टत्वादव्यावर्तकत्वादास्यशब्दोऽत्र न
मुखमात्रपरः । किन्तु आस्ये मुखे भवमास्यं = ताल्वादिस्थानम् ।
``शरीरावयवाद्यत्'' इति भवार्थे यत्प्रत्ययः । ``यस्येति च'' इति
प्रकृत्यन्त्यस्य अकारस्य लोपः । प्रकृष्टो यत्नः प्रयत्नः । आस्यं च
प्रयत्नश्च-आस्यप्रयत्नौ, तुल्यौ आस्यप्रयत्नौ यस्य वर्णजालस्य
तत्तुल्यास्यप्रयत्नं परस्परं सवर्णसंज्ञकं स्यादिति भावः । तदाह ---
ताल्वादिस्थानमित्यादिना । मिथ इति । परस्परमित्यर्थः । कस्य किं
स्थानमित्याकाङ्क्षायां तद्व्यवस्थापकानि पाणिन्यादिशिक्षावचनान्यर्थतः
संगृह्णाति --- अकुहेत्यादिना । ``अ'' इत्यष्टादश भेदा गृह्रन्ते, ``कु''
इतदि कादिपञ्चकात्मकः कवर्गः । न चाणुदित्सूत्रस्येदानीमप्रवृत्तेः कथमत्र
``अ'' इत्यष्टादशभेदानां ग्रहणमिति वाच्यम् । लौकिकप्रसिद्धिमाश्रित्यैव
एतत्प्रवृत्तेः । एवमग्रेऽपि कथञ्चित्समाधानं बोध्यम् । अश्च कुश्च हश्च
विसर्जनीयश्चेति विग्रहः । विसर्जनीयशब्दोऽपि विसर्गपर्यायः । इचुयशेति । इ
इत्यष्टादश भेदाः । चु इति चवर्गः । इश्च चुश्च यश्च शश्चेति विग्रहः ।
तालु = काकुदम् ।ऋटुरषेति । ऋ इत्यष्टादश भेदाः । टु इति टवर्गः । आ च
टुश्च रश्च षश्चेति विग्रहः । ऋ शब्दस्य आ इति प्रथमैकवचनान्तं धाता इति
वत् । लृतुलशेति । लृ इत्यस्य द्वादश भेदाः । तु इति तवर्गः । आ च तुश्च
लश्च सश्चेति विग्रहः । लृशब्दस्यापि आ इत्येव प्रथमैकवचनान्तम् । आ अलौ
अलः । दन्तशब्देन दन्तमूलप्रदेशो विवक्षितः । अन्यथा भग्नदन्तस्य
तदुच्चारणाऽनुपपत्तेः । उपूपेति । उ इत्यष्टादश भेदाः । पु इति पवर्गः उश्च
पुश्च उपध्मानीयश्चेति विग्रहः । उपध्मानीयशब्दमनुपदमेव स्वयं
व्याख्यास्यति । ञमङणनेति । ञश्च मश्च ङश्च णश्च नश्चेति विग्रहः । चकारेण
स्वस्ववर्गीयस्थानसमुच्चयः ।एदैतोरिति । एच्च ऐच्च एदैतौ ।
तपरकरणमसंदेहार्थम् । नतु ``तपरस्तत्कालस्य'' इति तत्कालमात्रग्रहणार्थम् ।
तेन प्लुतयोरपि संग्रहः । कण्ठश्च तालु चेति
प्राण्यङ्गत्वात्समाहारद्वन्द्वः, एकवत्त्वं नपुंसकत्वं च । ओदौतोरिति ।
ओच्च औच्च ओदौतौ । तपरकरणं पूर्ववदसंदेहार्थमेव । कण्ठश्च ओष्टौ चेति
प्राण्यङ्गत्वात्समाहारद्वन्द्वः, एकवद्भावो नपुंसकत्वं च । वकारस्येति ।
दन्ताश्च ओष्ठौ चेति समाहारद्वन्द्वः । एकवत्त्वं नपुंसकत्वं च
जिह्वामूलीयस्येति । जिह्वामूलीयशब्दमग्रे व्याख्यास्यति ।
एवमनुस्वारशब्दमपि । इति स्थानानीति । इति = एवंप्रकारेण
वर्णाभिव्यक्तिस्थानानि प्रपञ्चितानीत्यर्थः ।ननु किमिह तुल्यास्यसूत्रे
यत्कचित्स्थानसाम्यं विवक्षितम् , उत यावत्स्थानसाम्यम् । न तावदाद्यः, तथा
सति इकारस्य एकारस्य च तालुस्थानकतया सावण्र्यापत्तौभवत्येवे॑त्यत्र
इकारादेकारे परे सवर्णदीर्घापत्तेः । न च एकारस्य वर्णसमाम्नाये
पाठसामथ्र्यादिकारेण न सावण्र्यमिति वाच्यम्, एकारपाठस्य अक् इक् उक् इति
प्रत्याहारेषु एकारग्रहणनिवृत्त्यर्थत्वसम्भवात् । किं च
वकारलकारयोर्दन्तस्थानसाम्येन सावण्र्यापत्तौ ``तोर्ल'' इत्यत्र लकारेण
वकारस्यापि ग्रहणात्तद्वानित्यत्र दकारस्य परसवर्णापत्तिः । ``यवलपरे यवला
वा'' इत्यत्र लकारग्रहणं तु यथासंख्यार्थं भविष्यति । न द्वितीयः, तथा सति
कङयोः कण्ठस्थानसाम्येऽपि ङकारस्य नासिकास्थानाधिक्येन
सावण्र्याभावापत्तौचोः कुः॑, क्विन्प्रत्ययस्य॑ इत्यत्र ङकारस्य
ग्रहणानापत्त्या प्राङित्यादौ नुमो नकारस्य ``क्विन्प्रत्ययस्य'' इति
कुत्वेन ङकाराऽनापत्तेः । तस्मात्स्थ्ानसाम्यं दुर्निरूपमिति चेत्, अत्र
ब्राऊमः --- यावत्स्थानसाम्यमेव सावण्र्यप्रयोजकम् । एवं च इकारस्य एकारस्य
च तालुस्थानसाम्येऽपि एकारस्य कण्ठस्थानाधिक्यान्न सावण्र्यम् । वलयोश्च न
सावण्र्यम् । वकारस्य ओष्ठस्थानाधिक्यात् । एवं च " तद्वानासाम्"
``यजुष्येकेषाम्'' इत्यादिनिर्देशा उपपन्नाः । ङकारस्य नासिका
स्थानाधिक्येऽपि ककारेण सावण्र्यमस्त्येव, आस्यमवस्थानसाम्यस्यैव
सावण्र्यप्रयोजकत्वात्, नासिकायाश्च आस्यानन्तर्गतत्वात् । उक्तं च
भाष्ये-किं पुनरास्यं, लोकिकमोष्ठात्प्रभृति प्राक्काकलकात् इति ।काकलको
नाम चुबुकस्याधस्तात् ग्रीवायामुन्नतप्रदेशः॑ इति कैयटः ।
तस्मादास्यभवयावत्स्थानसाम्यं सावण्र्यप्रयोजकमिति शब्देन्दुशेखरे विस्तरः
।ननु तुल्यास्यसूत्रे प्रयत्नशब्देन प्रशब्दबलादाभ्यन्तरयत्नो विवक्षित इति
स्थितम् । तत्राभ्यन्तरत्वविशेषणं किमर्थम्, व्यावर्त्त्याऽभावादित्यत
आह-यत्नो द्विधेति । यत्नानामाभ्यन्तरत्वं बाह्रत्वं च वर्णोत्पत्तेः
प्रागूध्र्वभावित्वमिति पाणिन्यादिशिक्षासु स्पष्टम् । आद्य इति ।
आभ्यन्तयत्न इत्यर्थः । कथं चातुर्विध्यमित्यत आह-स्पृष्टेति ।कस्य कः
प्रयत्न इत्याकाङ्क्षायां तद्व्यवस्थापकशिक्षावचनानि पठति --- तत्रेति ।
तेषु मध्य इत्यर्थः । प्रयतनमिति । प्रयत्न इत्यर्थः । स्पर्शादिशब्दानग्रे
व्याख्यास्यति । ह्यस्वस्यावर्णस्य संवृतमित्यन्वयः । एतावदेव शिक्षावचनम्
। नन्वेवं दण्ड-आढकमित्यत्र अकारस्य च विवृतसंवृतप्रयत्नभेदेन
सावण्र्याऽभावात्सवर्णदीर्घो न स्यादित्यत आह-प्रयोग इति ।
शास्त्रीयप्रक्रियाभिः परिनिष्ठितानां रामः कृष्ण इत्यादिशब्दानां प्रयोगे
क्रियमाण एव ह्यस्वस्याऽवर्णस्य संवृतत्वमित्यर्थः । प्रक्रियेति ।
शास्त्रीयकार्यप्रवृत्तिसमये तु ह्यस्वस्याप्यवर्णस्य विवृतत्वमेवेत्यर्थः
। शिक्षावचनसिद्धं स्वाभाविकं ह्यस्वावर्णस्य संवृतत्वं प्रच्याव्य
शास्त्रमूलभूते वर्णसमाम्नाये तस्य विवृतत्वेनैवोपदिष्टतया
कृत्स्नशास्त्रीयप्रक्रियासमये ह्यस्वस्याप्यवर्णस्य विवृतत्वेन
दण्डाठतमिच्यादौ सवर्णदीर्घादिकार्यं निर्बाधकमिति ऐउणिति सूत्रभाष्ये
स्पष्टम् । एवंचह्यस्वस्याऽवर्णस्य संवृत॑मिति शिक्षावचनं
परिशेषात्परिनिष्ठतदशायामेव पर्यवस्यतीति न तदानर्थक्यमिति भावः । अयं च
शिक्षावचनसङ्कोचः सूत्रकारस्यापि संमत इत्याह-एतच्चेति॥ तदेवोपपादयितुं
प्रतिजानीते-तथा हीति । यथा एतज्ज्ञापितं भवति तथा स्पष्टमुपपाद्यत
इत्यर्थः ।
