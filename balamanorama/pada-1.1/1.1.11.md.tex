\textless{}\textgreater{} - ईदूदेद्द्विचनम् । ईच्च ऊच्च एच्चेति
समाहारद्वन्द्वः । ईदूदेदिति द्विवचनविशेषणत्वात्तदन्तविधिः
।द्विवचन॑मित्यनेन तु प्रत्ययत्वे ।ञपि न तदन्तं गृह्रते,संज्ञाविधौ
प्रत्ययग्रहणे तदन्तग्रहणं नास्ती॑ति तन्निषेधात् । तदाह ---
ईदूदेदन्तमित्यादिना । हरी एताविति । अत्र ईकारस्य
परादिवत्त्वाश्रयणाद्द्विवचनत्वम् । प्रगृह्रत्वे सति ``प्लुतप्रगृह्रा''
इति प्रकृतिभावान्न यण् । विष्णू इमावित्यत्राप्येवम् । गङ्गे अमू इत्यत्र
त्वयादेशो न भवति ।ईदूदेदन्त॑मिति तदन्तविधेः प्रयोजनं दर्शयितुमाह-पचेते
इमाविति । तदन्तविध्यबावे ईदूदेदात्मकं द्विवचनं प्रगृह्रमिति लभ्येत । एवं
सति ``पचेते'' इत्यत्रइते इति द्विवचनस्य एद्रूपत्वाऽभावात्प्रगृह्रत्वं न
स्यादिति भावः ।ईदूदेदन्तं यद्द्विचनान्त॑मिति व्याख्याने तु कुमार्योरगारं
कुमार्यगारमित्यत्रातिप्रसङ्गः स्यात् ।ईदूदेदन्तं द्विवचन॑मिति व्याख्याने
तु नातिप्रसङ्गः, ओसो द्विवचनस्य ईदूदेदन्तत्वाऽभावात् । ननुमणीवोष्ट्रस्य
सम्बेते प्रियौ वत्सतरौ ममे॑ति भारतश्लोकेमणी इवे॑ति ईकारस्य प्रगृह्रत्वे
सति प्रकृतिभावे सवर्णदीर्घो न स्यादित्यत आह-मणीवोष्ट्रस्येति ।वं
प्रचेतसि जानीयादिवार्थे च तदव्यय॑मिति मेदिनी ।व वा यथा तथैवैवं
साम्ये॑इत्यमरः ।
