\textless{}\textgreater{} - तरप्तमपौ घः । प्रथमस्य प्रथमपादे सूत्रमिदम्
। आतिशायनिकप्रत्ययप्रकरणान्ते, ``पितौ घः'' तादी घः इति वा वक्तव्ये
प्रकरणान्तरे पृथग्गुरुसूत्रकरणमत्यन्तस्वार्थिकमपि तरपं ज्ञापयति । तस्य
आतिशायनिकप्रकरणबहुर्भूतस्य सत्त्वे तत्संग्रहणार्थं प्रकरणान्तरे
सूत्रकरणस्यावश्यकत्वादित्याहुः । तेन ``अल्पाच्तरं'' ``लोपस्च बलवत्तरः''
इत्यादि सिद्धम् । किमेत्तिङव्यय । ``आमु'' इति छेदः । उकार उच्चारणार्थः ।
किम्, एते, तिङ्, अव्यय --- एषां चतुर्णां द्वन्द्वः
।किमेत्तिङव्ययप्रकृतिको घः॑ इति मध्यमपदलोपी समासः । फलितमाह --- किम
एदन्तादित्यादिना । एभ्य इत्यर्थः । किन्तमामिति अत्यन्तस्वार्थिकोऽयं
तमप्, नत्वतिशायने । एषामतिशयेनाढ इतिवदेषामतिशयेन क इति
विग्रहस्याऽसंभवात् । जातिगुणक्रियासंज्ञाभि समुदायादेकदेशस्य पृथक्करणं हि
निर्धारणम् । प्राह्णे तमामिति । प्राह्ण=पूर्वाह्णः
।प्राह्णाऽपराह्णमध्याह्नाः त्रिसन्ध्य॑मित्यमरः । अतिशयिते पूर्वाह्णे
इत्यर्थः । पूर्वावयवगतप्रकर्षादह्नः प्रकर्षो बोध्यः । अत्र अहर्न
द्रव्यम्, सूर्योदयादारभ्य सूत्रास्तमयावधिकस्यैव कालस्य
अहन्शब्दार्थत्वात् । तस्य च उदयादिक्रियाघटितत्वान्न द्रव्यत्वमिति भावः ।
पचतितमामिति । अतिशयिनता पाकक्रियेत्यर्थः, तिङन्तेषु
क्रियाविशेष्यकबोधस्यैवप्रशंसायां रूप॑विति सूत्रभाष्ये प्रपञ्चितत्वात् ।
अतोऽत्र क्रियाया एवन प्रकर्षो नतु द्रव्यस्येति भावः । उच्चैस्तमामिति
।आशंसती॑त्यध्याहार्यम् । अतिशयेन उच्चैराशंशनादिक्रियेत्यर्थः । अत्रापि
क्रियाया एव प्रकर्षो न तु द्रव्यस्य । उच्चैस्तमस्तरुरिति । अतिशेन
उच्चैस्तरुरित्यर्थः । अत्र उच्चैस्त्वप्रकर्षस्य तरौ द्रव्ये
भानादाभ्यनेत्यर्थः ।किंतमा॑मित्यादौयस्येति चे॑ति लोपं परत्वाद्बाधित्वा
ह्यस्वान्तलक्षणनुटो निवृत्त्यर्थमामु इत्युकारोच्चारणम् । सति तु
तस्मिन्निरनुबन्धकग्रहणे न सानुबन्धकस्ये॑ति परिभाषया नुड्विधावस्य न
ग्रहणमित्यादि ``आमि सर्वनाम्नः'' इति सूत्रभाष्ये प्रपञ्चितम् ।
