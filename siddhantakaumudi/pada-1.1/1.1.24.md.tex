षान्ता नान्ता च सङ्ख्या षट्संज्ञा स्यात् । षड्भ्यो लुक्
\textless{}\{261\}\textgreater{} । पञ्च । पञ्च । सङ्ख्या किम् । विप्रुषः
। पामानः । शतानि सहस्त्राणीत्यत्र संनिपातपरिभाषया न लुक् ।
सर्वनामस्थानसंनिपातेन कृतस्य नुमस्तदविघातकत्वात् । पञ्चभिः । पञ्चभ्यः ।
पञ्चभ्यः । षट्चतुर्भ्यश्च \textless{}\{338\}\textgreater{}इति नुट् ॥
