आदिश्यमानेषु ह्रस्वेषु एच इगेव स्यात् । प्रद्यु । प्रद्युनी । प्रद्यूनि
। प्रद्युनेत्यादि । इह न पुंवत् । यदिगन्तं प्रद्यु इति तस्य
भाषितपुंस्कत्वाऽभावात् । एवमग्रेऽपि । प्ररि । प्ररिणी । प्ररीणि ।
प्ररिणा । एकदेशविकृतस्याऽनन्यत्वात् रायो हलि इत्यात्वम् । प्रराभ्यात् ।
प्रराभिः । नुमिचिरेति नुट्यात्वे प्रराणामिति माधवः । वस्तुतस्तु
संनिपातपरिभाषया नुट्यात्वं न । नामि \textless{}\{209\}\textgreater{} इति
दीर्घस्त्वारम्भसामर्थ्यात्परिभाषां बाधत इत्युक्तम् । प्ररीणाम् । सुनु ।
सुनुनी । सुनूनि । सुनुना । सुनुने । इत्यादि ॥। इति
अजन्तपुल्लिङ्गप्रकरणम्‌ ।
