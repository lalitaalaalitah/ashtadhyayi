बहुव्रीहौ चिकीर्षते सर्वनामसंज्ञा न स्यात् । त्वकं पिता यस्य स
त्वत्कपितृकः । अहकं पिता यस्य स मत्कपितृकः । इह समासात्प्रागेव
प्रक्रियावाक्ये सर्वनामसंज्ञा निषिध्यते । अन्यथां लौकिके विग्रहवाक्ये इव
त्राप्यकच् प्रवर्तेत । स च समासेऽपि श्रूयेत । अतिक्रान्तो
भवकन्तमतिभवकानितिवत् । भाष्यकारस्तु त्वकत्पितृको मकत्पितृक इति रूपे
इष्टापत्तिं कृत्वैतत्सूत्रं प्रत्याचख्यौ । यथोत्तरं मुनीनां प्रामाण्यम्
॥\textless{}!संज्ञोपसर्जनीभूतास्तु न सर्वादयः !\textgreater{}
(वार्तिकम्) ॥ महासंज्ञाकरणेन तदनुगुणानामेव गणे संनिवेशात् । अतः
संज्ञाकार्यमन्तर्गणकार्यं च तेषां न भवति । सर्वो नाम कश्चित्तस्मै सर्वाय
देहि । अतिक्रान्तः सर्वमतिसर्वस्तस्मै अतिसर्वाय देहि । अतिकतरं कुलम् ।
अतितत् ॥
