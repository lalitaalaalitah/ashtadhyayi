प्रतीयते विधीयत इति प्रत्ययः । अविधीयमानोऽण् उदिच्च सवर्णस्य संज्ञा
स्यात् । अत्राण् परेण णकारेण । कु चु टु तु पु एते उदितः । तदेवम् । अ
इत्यष्टादशानां संज्ञा । तथेकारोकारौ । ऋकारस्त्रिंशतः । एवम् लृकारोऽपि ।
एचो द्वादशानाम् । एदैतोरोदौतोश्च न मिथः सावर्ण्यम् । ऐऔजिति
सूत्रारम्भसामर्थ्यात् । तेनैचश्चतुर्विंशतेः संज्ञाः स्युरिति नापादनीयम्
। नाज्झलौ \textless{}\{13\}\textgreater{} इति निषेधो
यद्यप्याक्षरसमाम्नायिकानामेव तथापि हकारस्याऽऽकारो न सवर्णः ।
तत्राऽऽकारस्यापि प्रश्लिष्टत्वात् । तेन विश्वपाभिः इत्यत्र हो ढः
\textless{}\{324\}\textgreater{} इति ढत्वं न भवति ।
अनुनासिकाननुनासिकाभेदेन यवला द्विधा । तेनाननुनासिकास्ते द्वयोर्द्वयोः
संज्ञा ॥
