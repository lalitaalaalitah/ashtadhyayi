॥ अथ अव्ययप्रकरणम्‌ ॥

स्वरादयो निपाताश्चाव्ययसंज्ञाः स्युः । स्वर्, अन्तर्, प्रातर्, पुनर्,
सनुतर्, उच्चैस्, नीचैस्, शनैस्, ऋधक्, ऋते, युगपत्, आरात्, पृथक्, श्वस्,
दिवा, रात्रौ, सायम्, चिरम्, मनाक्, ईषत्, जोषम्, तूष्णीम्, बहिस्, अवस्,
समया, निकषा, स्वयम्, वृथा, नक्तम्, नञ्, हेतौ, इद्धा, अद्धा,
सामि(गणसूत्रम् -) वत्, ब्राह्मणवत्, क्षत्रियवत्, सना, सनत्, सनात्, उपधा,
तिरस्, अन्तरा, अन्तरेण, ज्योक्, कम्, शम्, सहसा, विना, नाना, स्वस्ति,
स्वधा, अलम्, वषट्, श्रौषट्, वौषट्, अन्यत्, अस्ति, उपांशु, क्षमा,
विहायसा, दोषा, मृषा, मिथ्या, मुधा, पुरा, मिथो, मिथस्, प्रायस्, मुहुस्,
प्रवाहुकम्, प्रवाहिका, आर्यहलम्, अभीक्ष्णम्, साकम्, सार्धम्, नमस्,
हिरुक्, धिक्, अथ, अम्, आम्, प्रताम्, प्रशान्, प्रतान्, मा, माङ् ।
आकृतिगणोऽयम् । च, वा, ह, अह, एव, एवम्, नूनम्, शश्वत्, युगपत्, भूयस्,
कूपत्, कुवित्, सूपत्स, कुवित्, नेत्, चेत्, चण्, कच्चित्, किंचित्, यत्र,
नह, हन्त, माकिः, माकिम्, नकिः, आकिम्, माङ्, नञ्, यावत्, तावत्, त्वै,
द्वै, रै, श्रौषट्, वौषट्, स्वाहा, स्वधा, तुम्, तथाहि, खलु, किल, अथो, अथ,
सुष्ठु, स्म, आदह । (गणसूत्रम् -) उपसर्गविभक्तिस्वरप्रतिरूपकाश्च ।
अवदत्तम्, अहयुः, अस्तिक्षीरा । अ, आ, इ, ई, उ, ऊ, ए, ऐ, ओ, औ, पशु, शुकम्,
यथाकथाच, पाट्, प्याट्, अङ्ग, है, हे, भोः, अये, द्य, विषु, एकपदे, युत्,
आतः । चादिरप्याकृतिगणः ॥
