ताल्वादिस्थानमाभ्यन्तरप्रयत्नश्चेत्येतद्द्वयं यस्य येन तुल्यं तन्मिथः
सवर्णसंज्ञं स्यात् । अकुहविसर्जनीयानां कण्ठः । इचुयशानां तालु ।
ऋटुरषाणां मूर्धा । लृतुलसानां दन्ताः । उपूपध्मानीयानामोष्ठौ । ञमङणनानां
नासिका च । एदैतोः कण्ठतालु । ओदौतोः कण्ठोष्ठम् । वकारस्य दन्तोष्ठम् ॥
जिह्वामूलीयस्य जिह्वामूलम् । नासिकानुस्वारस्य । इति स्थानानि । यत्नो
द्विधा - आभ्यन्तरो बाह्यश्च । आद्यश्चतुर्धा -
स्पृष्टेषत्स्पृष्टविवृतसंवृतभेदात् । तत्र स्पृष्टं प्रयत्नं स्पर्शानाम्
। ईषत्स्पृष्टमन्तस्थानाम् । विवृतमूष्मणां स्वराणां च ।
ह्रस्वस्याऽवर्णस्य प्रयोगे संवृतम् । प्रक्रियादशायां तु विवृतमेव । एतच्च
सूत्रकारेण ज्ञापितम् । तथाहि ॥
